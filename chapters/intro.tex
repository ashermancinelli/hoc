\chapter{Introduction}

While the modern programmer may consider the term \emph{compiler} to be a specific one, it is specific only in the
sense that users of the word tend to mean the same thing when they say it.
The details of what a compiler does are often obscure to even professional software engineers.
This is especially evident when they feel the need to use a term like \emph{transpiler} to distinguish between a
"true" compiler and one that emits code another programming language.
Of course, to someone more familiar with the workings of compilers, this distinction is not so useful.
Brian Kernighan described compilers in this seemingly general way\cite{new_history_of_modern_computing}:

\begin{quotation}
\textit{
A compiler is a program that translates something written in one language into something semantically equivalent in another language.
For example, compilers for high-level languages like C and Fortran might translate into assembly language for a particular kind of computer; some compilers translate from other languages such as Ratfor into Fortran.
}
\end{quotation}

However, this is not sufficiently general.
Two key counter examples are \tex and \metafont, which are compilers that
transform text into PDF documents and renderable fonts.
Transforming text into an executable program is not the same process as
transforming text into a document or font, yet both are considered compilation.

When students are first introduced to compilers, the first sort of program they are contrasted with is interpreters,
but again, this distinction is not necessarily meaningful either.
The conventional notion of an interpreter is a program that performs the actions
specified by the source code as chunks of the code are consumed;
perhaps this was the case with early \texttt{BASIC} interpreters, but one is hard-pressed
to find a modern interpreter that does not perform some sort of compilation step.
The most popular interpreters for today's most popular interpreted languages,
Python and Javascript, both use relatively sophisticated compilation techniques.
There are even many Python \textit{libraries} that perform
just-in-time (JIT) compilation, targetting CPUs, GPUs, and other specialized hardware
\cite{jax-compiler}\cite{lam_numba}\cite{numba_cuda}\cite{triton_tillet}.

\vspace{0.5em}

To capture the full spectrum of compiler technologies,
the definition we will use in this book is intentionally broad:
\begin{quotation}
\textit{
    A compiler analyzes, transforms, and produces code, based on source code.
}
\end{quotation}

In the case of \tex and \metafont, the source code
may not entail a \textit{program} in the conventional sense;
\tex source code describes a document rather than a sequence of
instructions to be performed.
The code being produced by the compiler may be the encoded PDF format, for instance.
So too are interpreters which produce code in some form during the interpretation process.
The CPython interpreter produces bytecode before executing the program,
meaning it is a compiler by our definition.

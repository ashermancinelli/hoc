\chapter{Introduction}
While the modern programmer may consider the term \emph{compiler} to be a specific one, it is still often misunderstood. Moreover, in the beginning, the term was more ambiguous than it is today.
% it is specific only in the% sense that users of the word tend to mean the same thing when they say it.
% The details of what a compiler does are often obscure to even professional software engineers.
% This is especially evident when they feel the need to use a term like \emph{transpiler} to distinguish between a
% "true" compiler and one that emits code another programming language.
% Of course, to someone more familiar with the workings of compilers, this distinction is not so useful.

Brian Kernighan described compilers in this seemingly general way
\cite{new-history-of-modern-computing}:

\begin{quotation}
    A compiler is a program that translates something written in one language 
into something semantically equivalent in another language. For example, 
compilers for high-level languages like C and Fortran might translate into 
assembly language for a particular kind of computer; some compilers translate 
from other languages such as Ratfor into Fortran.
\end{quotation}
However, this is not sufficiently general. Two counter examples are \tex and 
\metafont, which are compilers that transform text into PDF documents and 
renderable fonts, respectively. Transforming text into an executable program is 
not the same operation as transforming text into a document or font, yet both 
are considered compilation. When students are first introduced to compilers, the 
first sort of program they are contrasted with is interpreters. This distinction 
is not necessarily meaningful. The conventional notion of an interpreter is a 
program that performs the actions specified by the source code as chunks of the 
code are consumed;perhaps this was the case with early \texttt{BASIC} 
interpreters and it may be a useful mental model when introducing interpreters 
to new programmers, but one is hard-pressed to find a modern interpreter that 
does not perform a compilation step. The most popular interpreters for today's 
most popular interpreted languages, Python and Javascript, both use relatively 
sophisticated compilation techniques. There are even Python \textit{libraries} 
that perform just-in-time (JIT) compilation, targeting CPUs, GPUs, and other 
specialized 
hardware\cite{jax-compiler}\cite{lam-numba}\cite{numba_cuda}\cite{triton-tillet}.

\vspace{0.5em}

To capture the full spectrum of compiler technologies, the 
definition we will use in this book is intentionally broad:
\begin{quotation}
\textit{A compiler analyzes, transforms, and produces code, based on source code.}
% \caption{Definition of \textit{compiler}.}
\label{def:compiler}
\end{quotation}

In the case of \tex and \metafont, the source code may not entail a 
\textit{program} in the conventional sense; \tex source code describes a 
document rather than a sequence of instructions to be performed. The code being 
produced by the compiler may be the encoded PDF format, for instance. So too are 
interpreters which produce code in some form during the interpretation 
process. The CPython interpreter produces \gls{bytecode} before executing the 
program, meaning it is a compiler by our definition. Perhaps CPython it is a 
compiler the consumes Python code and produces CPython bytecode, which also 
happens to ship a CPython bytecode virtual machine which typically executes the 
bytecode as soon as it is produced. 

While teaching a course on compilers at 
Columbia University, one of Alfred Aho's students wrote a compiler called 
Up$\flat$eat which produces music based on input data; given input data in some 
format, Up$\flat$eat produces output code in the form of 
music\cite{aho_oral_history_2022}. The student's final presentation was to set 
the input data to the ticker for some symbol from the New York Stock Exchange, 
playing happy and upbeat music whenever the ticker went up, and sad depressed 
music when it went down. Another Bell Labs creation was the connection of two 
programs: a program that translated numbers into words (e.g. 123 to "one 
hundred twenty three"), and a program that turned text into data representing 
sound waves. Connecting these two compilers (on that compiled text with numbers 
into text with words sounding out the numbers, and one that compiled text into 
sound waves) allowedBell Labs employees to send messages that would be turned 
into sound waves and broadcast on a speaker in the computer room. The point of 
this section is not pedantry, but to establish a broad definition of compiler 
technology before embarking on the details of its development. We will primarily 
focus on compilers that have run on a machine. We largely exclude theoretical 
works like Ada King Lovelace's notes on calculating Bournoulli numbers and the 
development of automata theory, for instance. Each step in the development of 
compiler technology builds on the previous steps, and while the prior steps are 
important, our focus is on compiler programs and not the theoretical works that 
preceded them, except where especially relevant.

\chapter{Dawn, 1940-1960}

\section{Where Does it Start?}

There is significant debate about who created the first compiler,
in no small part due to ambiguous nomenclature.
As late as 1962, the term \textit{software} was not even in common use,
as \citeauthor{the_first_computers_2002} note:

\begin{quotation}
[E]xpressions such as "hardware", "software", "machine language", "compiler",
"architecture" and the like... were unknown in 1950. They only arrived a decade
later, but the underlying concepts were quite familiar to us.
\cite{the_first_computers_2002}
\end{quotation}

This Honeywell advertisement \textit{A Few Quick Facts on Software} sought to clarify these terms as well:

\begin{quotation}
Software is a new and important addition to the jargon of computer users and builders.
It refers to the automatic programming aids that simplify the task of telling the computer 'hardware' how to do its job....
% Generally there are three basic categories of software:
% 1) Assembly Systems, 2) Compiler Systems, and 3) Operating Systems.\cite[ch.5]{new_history_of_modern_computing}
\end{quotation}

At this time, hardware was the only piece that mattered to customers.
Software was an afterthought, if a thought at all.
The instruction set of the machine was important, because that was the
user's interface to the machine.
It should come as no surprise, then, that the origins of our modern understanding of the
term \textit{compiler} are similarly murky, especially considering the fact that
\textit{compiler} already carried meaning in English, and was repurposed for computing.

\bigskip

Once could argue that any of these efforts consituted the first compiler:
\begin{itemize}
    \item Konrad Zuse's run-programs for the Z4 at the ETH Zurich in 1950
    \item Laning and Zierler's compiler for the Whirlwind at MIT in 1950
    \item Grace Hopper's A-0 and A-1 compilers for the UNIVAC I at Remington Rand in 1951
\end{itemize}

We will discuss these in order.

\section{Development of the Z4}

Konrad Zuse, a German civil engineer, began work on the Z4 during World War II.
Funded partially by his family and partially by the Nazi government, his prior works
demonstrated significant creativity and ingenuity, and they were leveraged to build
precursors to modern cruise missiles and guided bombs.
Most of Zuse's machines prior to the Z4 were destroyed during the war.

\todo{
Zuse's Z4 was a strange machine with bespoke memory and instruction set.
This affected how the compilers for it were designed.
}

\section{The ETH's Aquisition of the Z4}

There were several early efforts to create programs that produced punch cards
which contained machine code instructions, which could then be fed back into the machine
as input punchcards.
The programs produced by these early compilers were called \textit{run-programs},
and the process of using them was called \textit{automatic programming}, a term later
coined by Grace Hopper.
The first of these programs was run on a machine called the Z4, designed by Konrad Zuse in Germany.

Professor Eduard Stiefel, shortly after establishing the Institute of Applied Mathematics
to study numerical analysis at the Swiss Federal Institute of Technology (ETH) in Zurich,
began searching for a computer for the institute.
He learned of the computing advancements in the United States, Great Britain, and Germany,
but none were readily available at the time.
He sent his assistants Heinz Rutishauser and Ambros P. Speiser to the US to study
the latest developments in computing; they spent most of 1949 with
Howard Aiken at Harvard and John von Neumann at Princeton.

\begin{quotation}
    Before we returned, that is, in the middle of 1949, Stiefel was informed about the existence of Konrad Zuse's
Z4. At that time Zuse was living in Hopferau, a German village near the Swiss border. Stiefel was told that
the machine might be for sale. He visited Zuse, inspected the device, and reviewed the specifications.
Despite the fact that the Z4 was only barely operational, he decided that the idea of transferring it to Zurich
should by all means be considered. Stiefel wrote a letter to Rutishauser and me (we were at Harvard at the
time), describing the situation and asking us to get Aiken's opinion. Aiken's reply was very critical - the
future belonged to electronics and, rather than spending time on a relay calculator, we should now
concentrate our efforts on building a computer of our own.\cite{konrad_zuses_z4_2000}
\end{quotation}

Stiefel was undeterred by Aiken's criticism, and convinced the ETH to purchase the Z4.
In 1950, Heinz Rutishauser at Switzerland's ETH obtains a Z4.

\begin{quotation}
    We also made some hardware changes. Rutishauser, who was exceptionally creative, devised a way of
letting the Z4 run as a compiler, a mode of operation which Zuse had never intended. For this purpose, the
necessary instructions were interpreted as numbers and stored in the memory. Then, a compiler program
calculated the program and punched it out on a tape. All this required certain hardware changes. Rutishauser
compiled a program with as many as 4000 instructions. Zuse was quite impressed when we showed him this
achievement.\cite{konrad_zuses_z4_2000}
\end{quotation}

Thus were the first run-programs produced.
This is what we will consider \textbf{the first compiler}, though it was not called that at the time.

\section{Introducing Grace Hopper}

While Grace Hopper may not have been the first to create a program that
punched out another program as its output, she pioneered the field of
compilation to the extent that many consider her the inventor of compilers.
Her innovations were also more aptly timed and readily adopted than those at the ETH.
Consider Figure \ref{fig:dawn-timeline}, and the pace of development in Hopper's time
compared to the years prior.
Note that while we have ample data on \textit{how} Hopper's compiler worked and how
she and her team developed it, the intuition behind those developments is foggy at best.
We have recollections from Hopper and her contemporaries, but only from long after the fact.
It was not understood at the time how important her work was, so we have
only to speculate and piece together oral histories.

Originally a mathematics professor at Vassar College, Hopper obtained waivers for her age and weight and
joined the U.S. Navy in 1943, eventually graduating first in her class from Midshipmen's School.
She was assigned, somewhat unexpectedly, to Commander Howard Aiken's Harvard Computation Laboratory in 1944
as the third programmer of the Automatic Sequence Controlled Calculator (Mark I), 
the world's first operational computer.

\begin{quotation}
    According to her friend and former Harvard colleague Edmund Berkeley, Hopper turned to
    alcohol during this period as a way to deal with the compounding pressures at the 
    Harvard Computation Laboratory. She had dedicated herself fully to the overwhelming task of
    bringing Howard Aikens machines to life.
    She used the machines to solve critical military problems, 
    including one that resulted in an explosion over Nagasaki. 
    As the psychological strains became increasingly pronounced, 
    alcohol seemed to serve as an effective outlet, 
    freeing Hopper to express emotions and to temporarily forget obstacles real and imagined. 
    According to Berkeley, the expiration of Hoppers Harvard research contract was the best thing 
    that could have happened to her, although in the short term unemployment added to the stress. 
    During the last week of May 1949, the 43-year-old programmer packed up her belongings, 
    headed to Philadelphia, and bet her future on two younger men who believed they could create 
    the first commercial computer company.\cite{grace_hopper_and_the_invention_of_the_information_age_2009}
\end{quotation}

After a brief stint of unemployment, Hopper joined a startup called the Eckert-Mauchly Computer Corporation (EMCC)
where she found a more congenial environment to continue her work on compilers.

\section{Hopper at Harvard and the Mark I}

\todo{
ENIAC, could not be programmed (had to be rewired to be programmed)
but was real fast. Aiken's Mark I.
Fall 1944, Aiken wants to record technical developments at Harvard Computation Laboratory (as book),
chose Hopper, to her disagreement. Finished 561 page manuscript, published 1946.
She believed she was chosen for "clear, fluid prose."
her time as a math teacher made her good at writing/explaining things.
would force students to write essays on math topics and would grade harshely.
"it was no use trying to learn math unless they could communicate with other people."
Aiken and Hopper knew a large/diverse audience would need to understand it for it to
be successful; high value on good writing+communication.
Aiken's strict heirarchy+meritocracy allowed her to compete with the men.
IBM tried to steal all the credit for the Mark I from Aiken.
aiken distaste for being assigned woman officer, but meritocracy.
}

\section{Hopper and the Mark I's Manual}

we will drift into some history, only because it was so formative for GH, who was so formative
in the development of compilers.

\todo{
Her manual began with an historical summary of of developments in computing.
One of earliest attempts to document computing history.
\begin{itemize}
    \item Blaise Pascal's counting machines, "foundation on which nearly all mechanical calculating machines since have been constructed."
    \item Leibniz; stepped wheels system for mul/divs.
    \item Charles Babbage; most significant part of the manual dedicated to him.
          Difference engine, idea for computing machine. Invented punchcard system to feed in information, made after textile looms.
          G H emphasized the machine would take 2 decks of cards, one for data, one for instructions (not von neumann).
    \item Ada King, Countess of Lovelace; series of essays on Babbage's machine.
          described possibly the first computer program. This could never run and would
          have to wait for the Mark I before the dream could come true.
    \item Aiken's Mark I
\end{itemize}
Babbage's autobio, Hopper exposed to Ada King; "she wrote the first loop. I will never forget; none of us ever will."
coworkers cast Aiken as Babbage, Hopper as Ada King.
}

\section{Postwar Collaboration}

\todo{
1947 Aiken grew in stature. Could relocate military staff to Harvard, which he did with Hopper and two others.
1944-1947 Presper Eckert, John Mauchly, and female operators proved valuable bc of how well they could use computers.
this generated a wave of new projects:
BINAC+UNIVAC at ECC, Mark II and Mark III at Harvard, SSEC at IBM, Whirlwind at MIT,
MANIAC at Inst for Advanced Study, ERA (engineering research associates).
everyone was isolated during the war, now aiken wanted to start sharing ideas.
Symposium on Large Scale Digital Calculating Machinery, June 1947.
}
\begin{quotation}
We'd all been isolated during the war, you see, classified contracts and 
everything under the sun. It was time to get together and exchange information 
on the state of the art, so that we could all go on from there.
\end{quotation}

\section{Hopper at EMCC}

The company Hopper joined was one of the earliest pure computer-focused ventures,
founded by J. Presper Eckert Jr. and John Mauchly
(designers of the ENIAC, or the Electronic Numerical Integrator and Computer).
This startup environment contrasted sharply with the academic rigor of Harvard and the industrial scale of IBM.
There she found an open-minded and welcoming environment to develop her ideas;
Mauchly, who was to become Hopper's boss, was characterized as 
"very broadminded, very gentle, very alive, very interested, very forward looking,"\cite{grace_hopper_and_the_invention_of_the_information_age_2009}
creating a tolerant, flexible company atmosphere in contrast to the pressure she experienced at Harvard.
A majority of their programming staff consisted of mathematically inclined women
who had served as ENIAC operators at the Moore School.

When Hopper arrived in 1949, EMCC had two major projects underway:
the BINAC (Binary Automatic Computer), which was close to completion,
and the UNIVAC I (Universal Automatic Computer), which would be running within a year.
The work environment and upcoming UNIVAC project excited Hopper and enticed her to join the company
after walking out of an interview with IBM.

While the organization was grounds for fruitful and innovative research and development team,
EMCC was under financial strain; they depended on partial payments for UNIVAC-I orders to stay afloat.
The unexpected death of EMCC's chairman Henry Straus forced Eckert and Mauchly to seek a buyer,
which they found in Remington Rand, a typewriter and office equipment manufacturer.

\section{Hopper and the A-series}

Hopper and her team at Remington Rand developed three "compilers"
in rapid succession, the A-0, A-1, and A-2, for the UNIVAC I.
I quote "compilers" because the A-0 and A-1 were not compilers in the modern sense.
Her work was grounded in intellectual openness, collaboration, and accessability;
she pioneered the debuggability of programming languages, compiler error reporting,
and new ways to share code and collaborate, for example.

\begin{quotation}
Hopper's recollections point to motivations
ranging from an altruistic desire to allow "plain, ordinary people"
to program to dealing with her own laziness. Naturally one must
be skeptical of such claims, for they were made years after the
fact. In 1951 it was difficult for even a visionary like Hopper to
imagine the eventual ubiquity of computer technology, and one
can be pretty confident that Hopper was not a lazy person.\cite{grace_hopper_and_the_invention_of_the_information_age_2009}
\end{quotation}

She began work on the A-0 in October 1951 in her spare time in order to address
the mounting crisis facing Remington Rand: they were unable to fully support their
customers, and their sales teams were, to put it kindly, incompetent with respect
to their product, and they supported their customers as well as one might expect.

\begin{quotation}
    Another Hopper programmer, Adele Mildred Koss, was
assigned to Commonwealth Edison when the utility approached
the Chicago sales office concerning a potential purchase of a
UNIVAC for billing and payroll. At the time, Koss was 7 months
pregnant and working part time. Since her pregnancy precluded
travel, Commonwealth Edison management was forced to come
to Philadelphia in order to discuss their billing needs. In the end,
the utility did not buy a UNIVAC, but instead purchased an IBM
701 when it became available. Koss recalled: "I remember Grace
Hopper's memo to management saying 'This is a multi-million
dollar client and you are not treating them like one. You have
only assigned a part time programmer to work with.'"
\cite[Adele Mildred Koss, interviewed by Kathy Kleiman]{grace_hopper_and_the_invention_of_the_information_age_2009}
\end{quotation}

Shortly after this fiasco, Hopper's team was tasked with supporting UNIVAC I customers
at the US Census Bureau, a task she thought no one in the company was prepared for.

\begin{quotation}
Inspired by Holberton's Sort-Merge Generator, Hopper conceived the idea of writing a 
program to create a program, or in modern day terms, building a compiler. 
The idea was to get commonly used subroutines automatically inserted into another program based on calculated offsets. 
Most people at the time considered this impossible. As Hopper recalled:\cite{women_in_computing_history_2002}
\end{quotation}

\begin{quotation}
The Establishment promptly told us, at least they told me, quite frequently that a
computer could not write a program; it was totally impossible; that all computers
could do was arithmetic, and that it couldn't write programs.\cite{hopl_keynote}
\end{quotation}

\todo{
the A-2 was a front-end translator to the A-0, more like a complete
toolchain with a front end to a language, a linker and a standard library.}

\section{After the A-2}

\todo{
- later developed the A-3 and AT-3/MATH-MATIC
- FLOW-MATIC, \FTN, COBOL
}

\begin{quotation}
B-0, as the business language compiler was designated, became available to 
UNIVAC customers at the start of 1958. Before its completion, Remington Rand 
merged with Sperry Gyroscope Corporation to form Sperry Rand. The marketing 
department of the new company renamed the business language FLOW-MATIC. (In 
addition, AT-3 was renamed MATH-MATIC.) The completed version of FLOW-MATIC had 
a rich library of oper-ational verbs that appeared to meet the application 
needs of most businesses.23 These verbs included editing commands so 
information could be formatted before output. Furthermore, FLOW-MATIC provided 
unparalleled flexibility in data designation, thus allowing file names to be 
given complicated descriptions.
\end{quotation}

\section{Laning and Zierler at MIT}

\todo{
Laning and Zierler, 1954. Early 1950s. Inspiration for Backus/\FTN.
Worked more like a modern compiler than Hopper's A-0 and A-1.
}
\begin{quotation}
The first programming system to operate in the sense of a modern compiler was 
developed by J. H. Laning and N. Zierler for the Whirlwind computer at the 
Massachusetts Institute of Technology in the early 1950s. They described their 
system, which never had a name, in an elegant and terse manual entitled "A 
Program for Translation of Mathematical Equations for Whirlwind I," distributed 
by MIT to about one-hundred locations in January 1954.26 It was, in John 
Backus's words, "an elegant concept elegantly realized." Unlike the UNIVAC 
compilers, this system worked much as modern compilers work; that is, it took 
as its input commands entered by a user, and generated as output fresh and 
novel machine code, which not only executed those commands but also kept track 
of storage locations, handled repetitive loops, and did other housekeeping 
chores. Laning and Zierler's "Algebraic System" took commands typed in familiar 
algebraic form and translated them into machine codes that Whirlwind could 
execute.27 (There was still some ambiguity as to the terminology: while Laning 
and Zierler used the word "translate" in the title of their manual, in the 
Abstract they call it an "interpretive program.")28
\cite{new_history_of_modern_computing}
\end{quotation}

\pagebreak
\begin{figure}[h]
\begin{luacode}
timeline.draw_timeline({ 
    start_year = 1942,
    end_year = 1957,
    marker_interval = 1,
    show_year = false,
    events = {
        {1942, "Zuse conceptualizes first high-level language \\textit{Plankalkül}"},
        {1944, "Automatic Sequence Control Calculator started working"},
        {1950, "Rutishauser ran the Z4 as a compiler at ETH", position=1950},
        {1951, "B{\"o}hm develops practical compiler for PhD thesis at ETH"},
        {1951.5, "Hopper begins work on the A-0 compiler for the UNIVAC I"},
        {1952, "Hopper presents \\textit{Automatic Programming} at ACM"},
        -- {1952.5, "Alick Glennie develops first compiler in the modern sense for the U of Manchester Mark I"},
        {1952.5, "Alick Glennie develops compiler for Mark I "},
        {1953, "Hopper's team completes the A-1"},
        {1953.3, "Hopper's team completes the A-2"},
        {1953.6, "Hopper gives presentation to UNIVAC users on the A-2"},
        {1953.9, "Numerous US government agencies have adopted the A-2"},
        {1954.5, "John Backous at IBM begins work on \\ftn{}"},
        {1954.9, "Nora Moser of US Army sends Hopper modifications to the A-2"},
        {1957, "John Backous at IBM released the first \\ftn{} compiler, the first commerical compiler"},
    },
})
\end{luacode}
\caption{Early compiler history, 1940--1960}
\label{fig:dawn-timeline}
\end{figure}


\chapter{Dawn, 1940-1960}

In the beginning, compilers were not important, and hardware was.
Customers cared about the instruction set,
and if the compiler could generate those instructions too,
it was a nice convenience, but not a nessecity.
Compilers were not a significant part of the value chain for customers of early computers.

As late as 1962, the term \textit{software} was not even in common use, as \citeauthor{the_first_computers_2002} note\cite{the_first_computers_2002}:

\begin{quotation}
[E]xpressions such as "hardware", "software", "machine language", "compiler",
"architecture" and the like... were unknown in 1950. They only arrived a decade
later, but the underlying concepts were quite familiar to us.
\end{quotation}

Again, this Honeywell advertisement \textit{A Few Quick Facts on Software} sought to clarify these terms as well \cite[ch.5]{new_history_of_modern_computing}:

\begin{quotation}
Software is a new and important addition to the jargon of computer users and builders.
It refers to the automatic programming aids that simplify the task of telling the computer 'hardware' how to do its job....
Generally there are three basic categories of software:
1) Assembly Systems, 2) Compiler Systems, and 3) Operating Systems.
\end{quotation}

Though the nascent notion of \textit{software} was obscure, it already entailed compilers.

\vspace{4em}

\pagebreak
\begin{luacode}
timeline.draw_timeline({ 
    start_year = 1940,
    end_year = 1960,
    marker_interval = 5,
    events = {
        {1942, "Zuse conceptualizes first high-level language \\textit{Plankalkül}"},
        {1944, "Automatic Sequence Control Calculator started working"},
        {1950, "Rutishauser ran the Z4 as a compiler at ETH"},
        {1951, "Hopper begins work on the A-0 compiler"},
        {1952, "Hopper presents \\textit{Automatic Programming} at ACM"},
    },
})
\end{luacode}


\vspace{4em}

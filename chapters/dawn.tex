\chapter{Dawn, 1940-1960}

In the beginning, compilers were not important.
Hardware was important.
Customers cared about the instruction set,
and if the compiler could generate those instructions too,
it was a nice convenience, but not a nessecity.
Compilers were not a significant part of the value chain for customers of early computers.

As late as 1962, the term \textit{software} was not even in common use, as \citeauthor{the_first_computers_2002} note\cite{the_first_computers_2002}:

\begin{quotation}
[E]xpressions such as "hardware", "software", "machine language", "compiler",
"architecture" and the like... were unknown in 1950. They only arrived a decade
later, but the underlying concepts were quite familiar to us.
\end{quotation}

Again, this Honeywell advertisement \textit{A Few Quick Facts on Software} sought to clarify these terms as well \cite[ch.5]{new_history_of_modern_computing}:

\begin{quotation}
Software is a new and important addition to the jargon of computer users and builders.
It refers to the automatic programming aids that simplify the task of telling the computer 'hardware' how to do its job....
Generally there are three basic categories of software:
1) Assembly Systems, 2) Compiler Systems, and 3) Operating Systems.
\end{quotation}

Though the nascent notion of \textit{software} was obscure, it already entailed compilers.

While GH may not have been the first to create a program that
punched out another program as its output, she was the first to
do so in order to support users' needs and she had a team to help her.

\pagebreak
\begin{figure}[h]
\begin{luacode}
timeline.draw_timeline({ 
    start_year = 1942,
    end_year = 1957,
    marker_interval = 1,
    show_year = false,
    events = {
        {1942, "Zuse conceptualizes first high-level language \\textit{Plankalkül}"},
        {1944, "Automatic Sequence Control Calculator started working"},
        {1950, "Rutishauser ran the Z4 as a compiler at ETH", position=1950},
        {1951, "B{\"o}hm develops practical compiler for PhD thesis at ETH"},
        {1951.5, "Hopper begins work on the A-0 compiler for the UNIVAC I"},
        {1952, "Hopper presents \\textit{Automatic Programming} at ACM"},
        -- {1952.5, "Alick Glennie develops first compiler in the modern sense for the U of Manchester Mark I"},
        {1952.5, "Alick Glennie develops compiler for Mark I "},
        {1953, "Hopper's team completes the A-1"},
        {1953.3, "Hopper's team completes the A-2"},
        {1953.6, "Hopper gives presentation to UNIVAC users on the A-2"},
        {1953.9, "Numerous US government agencies have adopted the A-2"},
        {1954.5, "John Backous at IBM begins work on \\ftn{}"},
        {1954.9, "Nora Moser of US Army sends Hopper modifications to the A-2"},
        {1957, "John Backous at IBM released the first \\ftn{} compiler, the first commerical compiler"},
    },
})
\end{luacode}
\caption{Early compiler history, 1940--1960}
\label{fig:dawn-timeline}
\end{figure}


\chapter{Dawn}

\vspace{4em}

\sffamily
\begin{chronology}[1]{1944}{1960}{\textwidth-5em}[100ex]
    % \eventpoint{1945}{}[black][.6]
    % \eventpoint{1960}{}[black][.6]
\end{chronology}
\normalfont

\vspace{4em}

In the beginning, compilers were not important, and hardware was.
Customers cared about the instruction set,
and if the compiler could generate those instructions too,
it was a nice convenience, but not a nessecity.
Compilers were not a significant part of the value chain for customers of early computers.

As late as 1962, the term \textit{software} was not even in common use, as
this Honeywell advertisement \textit{A Few Quick Facts on Software} sought to clarify \cite[ch.5]{new_history_of_modern_computing}:

\begin{quotation}
\textit{
Software is a new and important addition to the jargon of computer users and builders.
It refers to the automatic programming aids that simplify the task of telling the computer 'hardware' how to do its job....
Generally there are three basic categories of software:
1) Assembly Systems, 2) Compiler Systems, and 3) Operating Systems.
}
\end{quotation}

The quotation of \textit{hardware} should inform the reader that the term was not widely
used at the time.

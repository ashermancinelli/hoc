\chapter{Codesign, \textit{2000-2025}}

In each age, contemporaries have attempted to place their work in the
greater history of computing, and some have attempted to look forward and
guess at the field's future; in either case, they are often wrong. Computing is a
young, volatile industry, and this task will be much easier at a future time
looking back on the past.
Nonetheless, this is what I will attempt to do in this chapter.

\begin{quotation}
	But there are still lots of interesting open problems left and one of the most
	intriguing aspects of compiler design is can we use AI, machine learning, and
	large language models like GPT-3 to create code automatically from written or
	spoken specifications. That's still an unfolding story and I'm not willing to
	trust any program created by an AI program at this point. I wouldn't want it in
	my pacemaker. I wouldn't want it in my self-driving car or in my airplane. But
	maybe for a computer game, it's okay. This is what they're creating with these
	at this time. So even the area of programming language translation is
	undergoing new approaches and how successful they will be is yet to be
	determined. \parencite{aho_oral_history_2022}
\end{quotation}

\section{What Does Codesign Mean?}

I primarily focus on two issues in this period, and how compiler design
aims to address them:

\begin{enumerate}
	\item More so than in previous periods, software and
	      hardware must be designed together to give users the best performance.
	\item There is a diverse ecosystem of compiler tools and
	      programming languages that do not interoperate.
\end{enumerate}

Prior to multicore CPUs, software did not have to change much to get the best performance
from newer CPUs--programmers could rely almost entirely on hardware manufacturers to deliver
them the best performance simply by making the chips faster.
With the advent of multicore CPUs and \acrshort{simd} instructions in the early 2000s,
this stopped being the case.
For a programmer to get the best possible performance from a CPU that supports \acrshort{simd}
instructions, they must either rewrite and possibly restructure their application to explicitly use SIMD
instructions, or they must rely on \gls{autovec}, where the compiler infers parallelism
by analyzing the user's program and generating SIMD instructions automatically.

The latter approach was (and often still is) unreliable

\parencite{lattner_golden_age_compiler_design_2021}.
\parencite{lattner_minsky_why_ml_needs_new_programming_language_2025}.

\section{Chris Lattner}

Chris Lattner's impact on the landscape of compiler technology in the 21\textsuperscript{st}
century can't be overstated. His work on LLVM has more or less beat out every other
compiler technology.
We will now discuss the stories of the technologies that Chris developed contiguously,
with soliloquies for uses of those technologies that Chris was not involved in.

\subsection{Low-Level Virtual Machine}
\label{sec:lattner-llvm}

In December of 2000, Chris began work on LLVM with his advisor Vikram Adve
as part of his PhD research at the University of Illinois at Urbana-Champaign.
LLVM stood for Low-Level Virtual Machine
at the time but is no longer an acronym and is simply the name of the project.
At the time of the publication of his thesis \parencite{lattner_phd_thesis_pointer_intensive_programs_2005},
the umbrella project initially contained only an \acrlong{ir}, an optimizer for that \acrshort{ir},
\acrshort{ir}-level linking, and both \gls{offline-compilation} and \gls{online-compilation} for code
generation.
Today, these components are all part of the LLVM \textit{subproject} within the
LLVM \textit{umbrella project}, which contains other compiler libraries and tools.

\begin{quotation}
	This chapter describes LLVM — Low-Level Virtual Machine — a compiler framework that
	aims to make lifelong program analysis and transformation available for arbitrary software, and
	in a manner that is transparent to programmers. LLVM achieves this through two parts: (a) a
	code representation with several novel features that serves as a common representation for analysis,
	transformation, and code distribution; and (b) a compiler design that exploits this representation
	to provide a combination of capabilities that is not available in any previous compilation approach
	we know of. \parencite{lattner_phd_thesis_pointer_intensive_programs_2005}
\end{quotation}

While LLVM is perhaps best-known for some of the specific technologies contained
in the umbrella project, its most novel features lie in the compiler architecture.
The \acrshort{ir} was more flexible and language-agnostic than the other contemporary
\acrshort{ir}s, but the nature of LLVM as a set of \textit{libraries} that can
roughly be used independently of each other.
The tools developers know LLVM for (like Clang, LLD, and LLDB) are really thin main
programs that simply call into the libraries for parsing C code, optimizing a chunk of
LLVM IR, or generating machine code for that LLVM IR.
No prior art provided this level of flexibility, and LLVM's IR,
terminology, and interfaces have become the lingua-franca of the compiler world.

\begin{quotation}
	While LLVM provides some unique capabilities, and is known for
	some of its great tools (e.g., the Clang compiler, a C/C++/Objective-C compiler
	which provides a number of benefits over the GCC compiler), the main thing that
	sets LLVM apart from other compilers is its internal
	architecture.
	\parencite[Section 11. LLVM]{brown_wilson_lattner_aosa_vol1_2011}
\end{quotation}



\begin{table}[h!]
	\centering
	\begin{tabularx}{\linewidth}{llX}
		\toprule
		Subproject      & Date Added       & Notes                                                               \\
		\midrule
		LLVM Core       & 2000             & Project inception at UIUC, see \Cref{sec:lattner-llvm}.             \\
		Clang           & 2006             & Chris starts work on Clang, see \Cref{sec:llvm-at-apple-clang}.     \\
		compiler-rt     & $\sim$2009       & Compiler runtime libraries, drop-in replacement for \texttt{libgcc} \\
		Clang           & $\sim$2009       & Clang supports C and Objective-C.                                   \\
		LLDB            & $\sim$2010       & LLDB shipped as part of Xcode 4.                                    \\
		Clang++, libc++ & $\sim$2010       & Clang now supports C++; C++ standard library added as LLVM project. \\
		lld             & $\sim$2011--2015 & LLVM linker                                                         \\
		libclc          & $\sim$2012       & OpenCL standard library                                             \\
		OpenMP runtime  & $\sim$2013--2014 & Added as LLVM's OpenMP runtime                                      \\
		MLIR            & 2019             & \todo{see section on MLIR}                                          \\
		Flang           & 2020             & \todo{see section on Flang}                                         \\
		llvm-libc       & 2020s            &                                                                     \\
		\bottomrule
	\end{tabularx}
	\label{tab:lattner-llvm-project-timeline}
	\caption{Approximate dates major LLVM subprojects joined the umbrella project.}
\end{table}


\subsection{Architecture of the IR, Optimizer, and Code Generator}


\todo{examples of the IR and the C++ interfaces for building it, and how
	this is usable from other programming languages like OCaml.}

The first (and maybe only) interaction most non-compiler-engineers have with LLVM
is through the frontends, like Clang for C, C++, Objective-C, or Swift.
This betrays the complexity and elegance with which the pieces of LLVM fit together.
These frontends are typically wrappers around the LLVM command-line parsing library,
a library to perform the lexing, parsing, semantic analysis and whatever else the frontend
for that specific language is expected to do (like module dependency analysis in Swift, for example),
and then various other LLVM subprojects.

Most of the time, these other subprojects include the pieces from the LLVM subproject,
which does optimization and code generation.

\subsubsection{The IR}

Lattner et al refer to LLVM's \acrshort{ir} as \textit{C with vectors} in \parencite{lattner_amini_mlir_og_paper_2021}.
Type annotations are needed in more places than with C and many constructs are lower-level than C
(like control flow), but the approximations is accurate.
The IR was designed to preserve high-level information from diverse programming languages,
but it is apparent that it is designed especially for C and C++.

There are no physical registers in LLVM IR. It is an \acrfull{ssa} ir,
meaning there are infinite "registers" and each can be assigned to only once.
Typically, \acrshort{ssa} IRs use a \textit{phi} or \textphi instruction to merge
values from different paths, like the value \texttt{x} in the C expression \texttt{if (cond) x = 1; else x = 2;}.
The \acrfull{cfg} is also explicit; functions are made up of \gls{basicblock}s,
basic blocks are made up of instructions and terminated by branches or terminators,
and functions end in exactly one terminator (like a \texttt{return}):

\begin{minted}[linenos,frame=single]{llvm}
; Perhaps not the most efficient way to add two numbers.
; unsigned add2(unsigned a, unsigned b) {
;   if (a == 0) return b;
;   return add2(a-1, b+1);
; }

define i32 @add2(i32 %a, i32 %b) {
entry:
  %tmp1 = icmp eq i32 %a, 0
  br i1 %tmp1, label %done, label %recurse

recurse:
  %tmp2 = sub i32 %a, 1
  %tmp3 = add i32 %b, 1
  %tmp4 = call i32 @add2(i32 %tmp2, i32 %tmp3)
  ret i32 %tmp4

done:
  ret i32 %b
}
\end{minted}

There are few instructions and some of them are overloaded, making the IR
look very similar to \acrshort{risc} assembly languages.
The IR as it was when Lattner published his PhD thesis had only 31 opcodes
\parencite{lattner_phd_thesis_pointer_intensive_programs_2005}.

His thesis used this example for a simple addressing expression in LLVM IR:

% https://godbolt.org/z/aaea4vcaf
%
% \begin{lstlisting}[frame=single]
\begin{minted}[linenos,frame=single]{llvm}
; X[i].a = 1;
%p = getelementptr %xty* %X, int %i, ubyte 3;
store int 1 , int* %p;
\end{minted}

Structs are represented in an unsurprising way:
% \begin{lstlisting}[language=llvm,frame=single]
\begin{minted}[linenos,frame=single]{llvm}
; struct RT {char A; int B[10][20]; char C;};
; struct ST {int X; double Y; struct RT Z;};
; int *foo(struct ST *s) {
;     return &s[1].Z.B[5][13];
; }
%struct.ST = type { i32, double, %struct.RT }
%struct.RT = type { i8, [10 x [20 x i32]], i8 }
\end{minted}

LLVM IR was initially more strongly typed than it is today:

\begin{quotation}
	One of the fundamental design features of LLVM is the inclusion of a language-independent type
	system. Every SSA register and explicit memory object has an associated type, and all operations
	obey strict type rules.
	\parencite{lattner_phd_thesis_pointer_intensive_programs_2005}
\end{quotation}

This strictness has been eased over time.
The \texttt{getelementptr} instruction was used to compute address offsets on
base pointers, and explicit types were required. Pointer types were specified with
\texttt{i8*}, just as one would write in C. Now, with the shift to
\textit{opaque pointers} (or \textit{untyped pointers}), there is a simple \texttt{ptr} type
that does not specify the pointee type, and \texttt{ptradd} is used instead of \texttt{getelementptr}
with no required type annotation.

\subsubsection{The Interface}

As already discussed, a huge part of LLVM's value proposition is its ability
to be consumed by other applications by virtue of its modular design.
Another project can link against LLVM's libraries and use its interfaces
to procedurally build IR, or they can just write LLVM IR to a file and
pass it to the LLVM tools.
This means LLVM's interfaces for constructing IR are more important than in other
compiler projects; other developers \textit{outside of LLVM} depend on those interfaces.
The users of LLVM's interfaces far outnumber the developers contributing to LLVM
directly on a regular basis.

\cref{quote:ml-lattner-pattern-matching}

\todo{all compilers are sorta written in ML bc pattern matching}.

\subsection{Chris and LLVM at Apple}
\label{sec:llvm-at-apple}

\subsubsection{Clang}
\label{sec:llvm-at-apple-clang}

\subsubsection{Swift}

\parencite{lattner_minsky_why_ml_needs_new_programming_language_2025}:

\begin{quotation}
	\label{quote:ml-lattner-pattern-matching}
	\textbf{Chris}
	And so Clang has some really cool stuff that allowed it to scale and
	things like that, but I was also burned out. We had just shipped it. It was
	amazing. I’m like, there has to be something better. And so, Swift really came
	starting in 2010. It was a nights and weekends project.
	Turns out, programming languages are a mature space. It’s not like you
	need to invent pattern matching at this point. It’s embarrassing that C++
	doesn’t have good pattern matching.

	\textbf{Ron}
	We should just pause for a second, because I think this is like a small
	but really essential thing. I think the single best feature coming out of
	language like ML in the mid-seventies\dots
	having this pattern matching facility that lets you basically in a
	reliable way do the case analysis so you can break down what the possibilities
	are—is just incredibly useful. And very few mainstream languages have picked it
	up. I mean Swift again is an example, but languages like ML, SML, and Haskell,
	and OCaml.

	\textbf{Chris}
	I mean pattern matching, it is not an exotic feature. Here we’re
	talking about 2010.
	And so pattern matching, when I learned OCaml, it’s so beautiful. It
	makes it so easy and expressive to build very simple things.
\end{quotation}

\begin{table}[h!]
	\centering
	\begin{tabularx}{\linewidth}{l l X}
		\toprule
		Date       & Organization & Notes                                                              \\
		\midrule
		2022--     & Modular AI   & CEO/co-founder, Mojo programming language and compiler.            \\
		2020--2022 & SiFive       & Started LLVM CIRCT project applying MLIR to chip design.           \\
		2017--2020 & Google       & Started the MLIR project, worked on TPU compilers including XLA.   \\
		2017       & Tesla        &                                                                    \\
		2005--2017 & Apple        & Major expansion of the LLVM project. See \Cref{sec:llvm-at-apple}. \\
		2000--2005 & UIUC         & Designed foundations of LLVM.                                      \\
		\bottomrule
	\end{tabularx}
	\label{tab:lattner-career-timeline}
	\caption{Timeline of Chris Lattner's career \parencite{lattner_resume_work_history_2025}}
\end{table}


\subsection{Tablegen}

Perhaps the part of LLVM that is least-known among \textit{users} of LLVM is
its \textit{Tablegen} system. Tablegen is a program for declarative metaprogramming
inside LLVM.

\subsection{Chris and MLIR at Google}

\subsection{Mojo}

% \section{Godbolt}
% Huge jump in compiler accessability\cite{godbolt_happy_birthday_ce_2022}.
\section{MLIR}

In modern compilers, the \gls{ir} must be extremely flexible and generic
because every component of the compiler needs something different from it.

For example, the compiler's frontend typically cares very much about the
grammar of the source language, the semantic correctness of the \gls{ast},
and perhaps some early optimizations that might be performed at a very high level.
However, after that, the frontend really only cares about communicating the information
it has to the optimizer. It does not necessarily care about the details of the
optimizer, and simply wants a textual representation of the program.
The frontend would ideally emit an \gls{ir} that is simple with a level of abstraction
roughly matching that of the source language, minimizing the amount of work required to generate it.
The frontend may record that the user requested a particular region of code to
be inlined, unrolled, or offloaded, but writ large, the frontend does not want to deal with
the details of actually performing those transformations.

The optimizer, on the other hand, cares very much about the semantics the IR
because it is searching for patterns in the program that can be optimized.
The optimizer will want \textit{normal forms} of programs that make pattern matching
more straightforward, and it needs to be able to represent the artifacts of
optimization.
For example, after vectorization, the IR produced by the optimizer may look dramatically
different from the user's source code, with loops versioned for different vector lengths
based on aliasing information, inlined function calls, dead code eliminated, and operations
with operands known at compile time folded away.
For this phase of the compiler (and for certain optimizations more than others),
an IR capable of representing \textit{optimized} code is necessary, including
operations that may not be representable in the source language.
The optimizer does not care about how many registers the target machine has
nearly as much as the backend does, and it does not mind making drastic changes
to the representation of the program.

The backend has another set of desires; it must have information about the
machine being targeted, and it needs to map the semantics represented by the IR
to the machine's capabilities. It cares about the number of registers available,
the legality of operations on a particular machine, and how to perform those operations efficiently.
The optimizer may have produced an IR using very wide vectors, but the
backend will have to decide how to perform those operations on a specific machine
that may or may not have hardware that corresponds to vector operations of that length.

For these reasons, the different components of a compiler are constantly pulling
the IR in different directions, because all of their needs and wants must be
representable in their common format.
Now, imagine we have an IR that allows each component of the compiler to express
the semantics in a format and with semantics that \textit{they dictate}.
The frontend can define very high-level operations that are roughly equivalent
to the source language so the work it needs to do when converting the \gls{ast}
into the IR is minimal.
The vectorizer may define new operations on vectors, the semantics they
describe, and the process for converting them into another format.
The backend may define different operations for each machine it targets,
and it can progressively convert constructs from the frontend and optimizer
into operations suitable for the target machine, that it can convert into
machine code.
This is one of the key benefits of MLIR; each component can define its own
operations and semantics, the IR can be \textit{progressively} lowered from
one phase to the next, and each phase of the compiler can coexist with the others.

% Taking LLVM IR as an example, the Clang frontend will parse and analyze the source code
% and perform some optimizations, after which it would like to simple generate a valid
% representation of the program to let the optimizer do its job.


% \section{Accelerators}
% Lattner talks on mojo explaining motivation for languages and hw sw codesign.
% FPGAs. ASICs

% \section{Python DSLs}

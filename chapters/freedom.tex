\chapter{Freedom, 1980-2000}
\todo{gnu linux c llvm python; Facebook php->c++ compiler;}
\todo{lattner tried to get llvm in the gnu project; new licensing, permissive licensing.}
\begin{quotation}
The discussion of the GNU/Linux operating system and the "open source" software 
movement, discussed last, likewise has deep roots. Chapter 3 discussed the 
founding of SHARE, as well as the controversy over who was allowed to use and 
modify the TRAC programming language. GNU/Linux is a variant of UNIX, a system 
developed in the late 1960s and discussed at length in several earlier chapters 
of this book. UNIX was an open system almost from the start, although not quite 
in

the ways that "open" is defined now. As with the antitrust trial against 
Microsoft, the open source software movement has a strong tie to the beginnings 
of the personal computer's invention. Early actions by Microsoft and its 
founders played an important role here as well. We begin with the antitrust 
trial.
\cite{history_of_modern_computing_2003_ceruzzi}
\end{quotation}

\section{GNU}

\section{Adoption of Linux}

\section{Low-Level Virtual Machine}

Key innovation is that LLVM is a collection of compiler \textit{libraries} that have
thin programs wrapping them.
Other open-source compilers like GCC tend to be monolithic programs, which can
be harder to compose.
LLVM contains a collection of sub-projects that can be used independently as tools
\textit{or libraries}; key examples being the LLVM subproject of LLVM itself
(which contains LLVM IR, the intermediate representation, the optimizer, interpreter and code generator)
and Clang, a C/C++/Objective-C compiler front-end for LLVM.
Any other compiler project could emit LLVM IR and fully leverage the LLVM project after the front-end.
LLVM's intermediate representation has become the lingua-franca of the compiler world.

\begin{quotation}
From its beginning in December 2000, LLVM was designed as a set of reusable 
libraries with well-defined interfaces [LA04]. At the time, open source 
programming language implementations were designed as special-purpose tools 
which usually had monolithic executables. For example, it was very difficult to 
reuse the parser from a static compiler (e.g., GCC) for doing static analysis 
or refactoring. While scripting languages often provided a way to embed their 
runtime and interpreter into larger applications, this runtime was a single 
monolithic lump of code that was included or excluded. There was no way to 
reuse pieces, and very little sharing across language implementation projects.
\cite{aosa_vol1}
\end{quotation}

\begin{quotation}
The name "LLVM" was once an acronym, but is now just a brand for the umbrella 
project. While LLVM provides some unique capabilities, and is known for some of 
its great tools (e.g., the Clang compiler2, a C/C++/Objective-C compiler which 
provides a number of benefits over the GCC compiler), the main thing that sets 
LLVM apart from other compilers is its internal architecture.
\cite[LLVM]{aosa_vol1}
\end{quotation}

Lattner published his thesis on LLVM in 2002, and joined Apple in 2005.

\pagebreak
\section{Timeline}
\begin{figure}[h]
    \begin{luacode}
        local start = 1980
        local _end = 2020
        timeline.draw_timeline({    start_year = start,
            end_year = _end,
            marker_interval = 5,
            show_year = false,
            line_always = true,
            events = {        
                {1985, "??? gnu, gcc, stallman"},
                {1990, "??? torvalds, linux"},
                {1995, "??? IBM moves to linux"},
                {2004, "Development on LLVM begins", delta=0},
                {2012, "Compiler Explorer open-sourced by Matt Godbolt"},
        },})
        tex.sprint(string.format("\\caption{Open Source Compiler Development Timeline, %d--%d}", start, _end))
\end{luacode}
\label{fig:oss-timeline}
\end{figure}


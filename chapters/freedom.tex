\chapter{Freedom, 1980-2000}
\todo{gnu linux c llvm python; Facebook php->c++ compiler;}
\todo{lattner tried to get llvm in the gnu project; new licensing, permissive licensing.}
\begin{quotation}
The discussion of the GNU/Linux operating system and the "open source" software 
movement, discussed last, likewise has deep roots. Chapter 3 discussed the 
founding of SHARE, as well as the controversy over who was allowed to use and 
modify the TRAC programming language. GNU/Linux is a variant of UNIX, a system 
developed in the late 1960s and discussed at length in several earlier chapters 
of this book. UNIX was an open system almost from the start, although not quite 
in

the ways that "open" is defined now. As with the antitrust trial against 
Microsoft, the open source software movement has a strong tie to the beginnings 
of the personal computer's invention. Early actions by Microsoft and its 
founders played an important role here as well. We begin with the antitrust 
trial.
\cite{history_of_modern_computing_2003_ceruzzi}
\end{quotation}

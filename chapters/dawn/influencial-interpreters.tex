\section{Influential Interpreters}

Some early programming languages were initially interpreted,
but would go on to heavily influence compilers and programming language design
later on. We will introduce them here even though the payoff will not come until
later chapters.

\subsection{APL}

APL, standing for \textit{A Programming Language}, was developed by Kenneth E. Iverson in 1962
and partially implemented on the IBM System/360 as APL/360.
While this partial implementation was only an interpreter for a subset of the language Iverson
designed, it would spur on the development of a family of programming languages
called \textit{array programming languages} or \textit{Iversonian languages}.
\citetitle[Section X.4]{sammet_programming_languages_history_and_fundamentals_1969}.

\subsection{Lisp}

Lisp was developed primarily by John McCarthy in two bouts:
most of the key ideas were developed in 1956-1958,
and in 1958-1962 the language was actually implemented and applied to artificial
intelligence.

In terms of programming language design, Lisp's fundamental data structure is the \textit{list}--
hence the name (\textit{\textbf{LIS}t \textbf{P}rocessing}).
Symbolic computing (which was still novel at the time) was implemented using lists and basic
operations on them.
Lisp \textit{programs} were also represented as data; one of the key innovations of the language
the ability manipulate Lisp programs as any other data structure, with the \texttt{eval}
function serving as the bridge between the code as data and code as actions to be performed
by a Lisp program.
This concept of self-modifying programs was to serve as the basis for ALGOL Y,
the more ambitious successor to ALGOL 60,
which was abandoned due to the eventual scope and controversy of the ALGOL X project,
or ALGOL 68 as it came to be known.

\subsubsection{1956-1958}

Lisp was born out of John McCarthy's desire for a list-oriented language
for work on an IBM 704 at Dartmouth College during a summer research project
in 1956, which was the first organized study of artificial intelligence
\cite{mccarthy_history_of_lisp_1978}.

McCarthy was presented the list-processing programming language \textit{IPL 2},
written for the RAND Corporation's JOHNNIAC computer.
\footnote{The RAND Corporation was not the same company as Remington Rand/Sperry Rand, where Grace Hopper worked around this time.}
Dartmouth was soon to get access to an IBM 704 thanks to the New England Computation Center
at MIT, which IBM was in the midst of establishing.
McCarthy was to consult with a team at IBM developing a theorem-proving program for plane geometry,
and it was not clear at the time whether IBM's FORTRAN would be suitable for list-processing.

McCarthy was also independently working on artificial intelligence,
publishing his first paper in the field in 1958,
\citetitle{mccarthy_programs_with_common_sense_1958}.
This paper, also called the \textit{Advice Taker} proposal,
involved representing information in the form of sentances in a formal language,
and an accompanying program that would make inferences based on that information.
These sentences were to be structured as lists, so he naturally
needed a list-processing language to process these sentences.

McCarthy started by considering how list structures would be represented in
memory. The IBM 704 had an addressable word size of 36 bits with
a 15-bit address space, so the pointers would need to be 15 bits,
which allowed for two pointers in each word, plus 6 extra bits.
A list in Lisp was to be represented in a word like so:

\begin{table}[h]
	\centering
	\begin{tabular}{|c|c|c|c|c|}
		\hline
		               & \textbf{Tag} & \textbf{Decrement} & \textbf{Prefix} & \textbf{Address} \\
		\hline
		Width          & 3 Bits       & 15 Bits            & 3 Bits          & 15 Bits          \\
		\hline
		Purpose        & Type/Opcode  & Address of Tail    & Type            & Pointer to data  \\
		\hline
		Lisp primitive & \texttt{ctr} & \texttt{cdr}       & \texttt{cpr}    & \texttt{car}     \\
		\hline
	\end{tabular}
	\caption{Layout of a 36-bit word on the IBM 704 as used for Lisp list structures.}
\end{table}

Nathaniel Rochester, Herbert Gelernter and Carl Gerberich at IBM
took on the task of writing implementing this list-processing language
in FORTRAN, called FLPL for \textit{FORTRAN List Processing Language}.
This primitive version of the language lacked conditional expressions, recursion, and other
fundamental features.
McCarthy developed the conditional expressions in 1957-1958 while developing a chess program
in FORTRAN, though it was not the conventional notion of an if-statement since both arms
of the conditional expression were always evaluated; \texttt{XIF(C, E1, E2)} would return
\texttt{E1} if \texttt{C} was equal to 1 and \texttt{E2} otherwise, which is typically called
a \textit{merge} operation today.
This was motivated by the clumsy syntax and semantics of the \texttt{IF} statement in FORTRAN I and II.

Nathaniel Rochester invited McCarthy to join the IBM Information Research Department for the
summer of 1958 to implement differentiation of algebraic expressions in FLPL,
where he would go on to develop many more foundational concepts in Lisp programming
language design.

\todo{
	Started in 1958: \citetitle{mccarthy_history_of_lisp_1978}.
	MIT, doug mcqueen's collaboration, SMLNJ.
}

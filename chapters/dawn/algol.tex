\section{The IAL and the ALGOrithmic Language}

Before Hopper, Mayes and Phillips pulled together their committees for a common
business language, design-by-committee was already in the academic milieu,
especially in Europe.
While not commercially successful, ALGOL introduced a number of important concepts
like block scopes and the declaration of the types of variables, and it would
go on to be the standard language for describing algorithms in academia.
Originally called IAL (International Algebraic Language), it came to be called
ALGOL, or ALGOrithmic Language, and was designed by an international committee
with representatives from different organizations with the goal of a truly
machine-independent language.

After deliberation in numerous committees, representatives from the German
Association for Applied Mathematics and Machinery (GAMM) and the ACM met in
Zurich, Switzerland in the summer of 1958. They had both produced similar
reports and wanted to meet and agree on a unified language.
John Backus, Charles Katz, Alan Perlis and Joseph Wegstein from the ACM attended this meeting.
\todo{who were these people? Add some narrative. Maybe of Naur, then tie in with Backus.}
They arrived at the following objectives
\cite{perlis_samelson_1958_preliminary_report_ial}:

\begin{enumerate}
\item The new language should be as close as possible to standard mathematical 
      notation and be readable with little further explanation.
\item It should be possible to use it for the description of computing processes in publications.
\item The new language should be mechanically translatable into machine programs.
\end{enumerate}

Shortly thereafter, a large number of dialects and partial implementations sprung up around Europe
and the US such as BALGOL from Burroughs Corporation in Detroit, Michigan for the Burroughs 220.
Manufacturers such as Burroughs found the standard to be insufficient for their users:
"BAC-220 provides additions for the ALGOL reference language which are essential to
the operation of data-processing systems: input-output
facilities, conventions for inclusion of segments of machine-language coding,
and diagnostic features" \cite{burroughs1963bac220}.
This was intentional; the specifications of ALGOL (both the 1958 and the 1960 versions)
was solely for the purposes of
describing computation; no I/O or system libraries were specified.
Other dialects included CLIP, JOVIAL, MAD, and NELIAC.
The first issue of the \textit{ALGOL Bulletin} was issued in March of 1959 out of Copenhagen
with Peter Naur as the editor.

Jean Sammet describes the impact of ALGOL 58
\cite{sammet_programming_languages_history_and_fundamentals_1969}:
\begin{quotation}
Among the more intriguing technical features of ALGOL 58 were its essential 
simplicity; the introduction of the concept of three levels of language, namely 
a reference language, a publication language, and hardware representations; the 
\textbfit{begin\dots end} delimiters for creating a single (compound) statement from simpler 
ones; the flexibility of the procedure declaration and the \textbfit{do} statement for 
copying procedures with data name replacement allowed; and the provision for 
empty parameter positions in procedure declarations. While ALGOL 58 is not an 
exact subset of ALGOL 60, the only items of significance which are in the 
former but not the latter are the \textbfit{do} which was removed as a concept (although 
the word was used for something else) and the empty parameter positions. 
Because of this major carry-over, specific technical description of ALGOL 58 is 
not necessary. 
\end{quotation}

\section{ALGOL 60}

\todo{backstory of Peter Naur}

The International Conference on Information Processing was held in Paris
in June, 1959, where there were several key developments in ALGOL.
Firstly, John Backus presented the now-famous paper on Backus-Naur Form
for formal specification of programming languages\cite{backus_ial_1959}.
Although other accounts describe this paper as having garnered significant
attention in the IAL proceedings, Backus would recount that his paper
"was received with a silence that made it seem that precise syntax
description was an idea whose time had not yet come.
As far as I know that paper had only one reader, Peter Naur"
\cite{Backus_1980_Programming_in_America_in_1950s}.
Backus was involved in two programming languages at the time, \FTN{} and IAL,
having already moved on from Speedcoding.
Backus had been made aware of the logician Emil Post and her work on computability
in a course given by Martin Davis, in particular, Post's notion of a "production."
It was this concept that inspired Backus to write the 1959 paper, and Naur
would go on to extend it. This is part of Backus's specification of arithmetic expressions
in the paper before Naur's changes\cite{backus_ial_1959}:

\[
\langle \textit{factor} \rangle :\equiv
\langle \textit{number} \rangle \;\overline{\text{or}}\;
\langle \textit{function} \rangle \;\overline{\text{or}}\;
\langle \textit{variable} \rangle \;\overline{\text{or}}\;
\dots
\]

Naur would make the notation more readable by replacing $\overbar{or}$ with $|$
and $:\equiv$ with $::=$ and by fully spelling out names that backus had abbreviated.
Naur's edits and extensions would bring interest from more readers than Backus's original paper had.
This form would come to be known as Backus-Naur Form, or BNF.
Naur would submit this in a paper to the subsequent ALGOL meeting in 1960,
titled \citetitle{naur_backus_algol_1960}\cite{naur_backus_algol_1960}.
Various commitees had established some shortfalls of the original design of the AIL,
and the 1960 meeting set out to address them with the help of Backus and Naur's
specification.
This report "represents the union of the Committee's concepts and the
intersection of its agreements."
The IAL had been renamed ALGOL 58 and then finally ALGOL 60 in Naur's report.

The final report specified expressions to have the following syntax:

\[
\langle \textit{expression} \rangle ::=
\langle \textit{arithmetic expression} \rangle
| \langle \textit{Boolean expression} \rangle
| \langle \textit{designational expression} \rangle
\]

\section{Adoption of ALGOL}

The \textit{ACM Communications} started publishing an \textit{Algorithms} section;
ALGOL 58 was initially used, but ALGOL 60 was adopted once the updated report
was in wide circulation.
\FTN was not considered acceptable for publication until 1966, and even then
it did not see widespread use in the journal.
Because ALGOL had been designed from scratch by committee, there were
no reference compilers for people to use; the only way to verify the correctness
of an algorithm described in ALGOL was to rewrite it in another language
and run test against test cases.

In early 1960, the \textit{ACM ALGOL Maintenance Group} formed: an informal working
group for the purpose of discussing the implementation of an ALGOL compiler.
They discussed ambiguities in the specification and potential changes to the ALGOL 60 report.
They mostly corresponded by mail and reports of their discussions were synthesized in
the \textit{ALGOL Bulletin}\cite{sammet_programming_languages_history_and_fundamentals_1969}.
The 14 issue of the bulletin in 1962 contained a questionnaire from Peter Naur
about some ambiguities in the 1960 report, and the philosophy of some proposed changes and
enhancements to the language.
\footnote{The reader may find it interesting that in this edition of the bulletin,
Jean Sammet submitted a paper on \textit{A Method of Combining ALGOL and COBOL};
\todo{follow up on this?}}
This questionnaire was used as a guide for another committee meeting in Rome in April of 1962
the primary result of this meeting is detailed in Naur's
\citetitle{naur_revised_report_algol_1962}.

\section{Peculiarities of ALGOL}

The reader may find it interesting that ALGOL had a number of peculiar features.
Namely, the distinction between \textit{call by name} and \textit{call by value}.
Naur's 1962 revised report explains it best, I think\cite[Section 4.7.3]{naur_revised_report_algol_1962}:

\begin{quotation}
\noindent
\textbf{4.7.3.1 Value assignment (call by value)}

of the corresponding actual parameters\dots the effect is as though an 
additional block embracing the procedure body were created in which these 
assignments were made to variables local to this fictitious block\dots

\noindent
\textbf{4.7.3.2 Name replacement (call by name)}

Any formal parameter not quoted in the value list is replaced, throughout the 
procedure body, bu the corresponding actual parameter, after enclosing this 
latter in parentheses wherever syntactically possible.
\end{quotation}

While neither Naur's revised report nor Edgar Dijkstra's
\citetitle{dijkstra_primer_of_algol_programming_1962}
point out the implications of this distinction, it is important to note
the subtle implications.
Call-by-name is not common in modern programming languages, and readers
may find it more similar to macro expansion
(though it is not, in fact, simply textual replacement).
The following example illustrates the difference:

\lstinputlisting[language=python,frame=single]{chapters/dawn/callbyname1.py}

Most implementations of call-by-name use zero-argument thunks to delay
the evaluation of arguments until they are used in the body.
The arguments to function calls become zero-argument closure functions
returning the value of the argument expression and capable of being
\textit{l-values}, or values in the left-hand side of an assignment expression.
Then, each instance of the argument in the callee's body is replaced
with an invocation of this function such that it is evaluated anew
each time it appears.
The prior Python example may be rewritten to emulate call-by-name as follows:

\lstinputlisting[language=python,frame=single]{chapters/dawn/callbyname2.py}

Call-by-value semantics demand that the argument to \texttt{double} be evaluated
before entering the function, so \texttt{get\_5} is called once and
\textit{the result} is passed to \texttt{double}.
With call-by-name semantics, \texttt{get\_5} is passed the expression itself,
which is the substituted into the body of the callee \textit{before}
being evaluated.

Donald Knuth proposed the \textit{man or boy} test in the \textit{ALGOL Bulletin}
to distinguish between
conformant ALGOL compilers capable of handling recursion and non-local references
using call-by-name semantics\cite{Knuth1964ManOrBoy}.

\begin{quotation}
There are quite a few ALGOL60 translators in existence which have been designed 
to handle recursion and non-local references properly, and I thought perhaps a 
little test-program may be of value. Hence I have written the following simple 
routine, which may separate the man-compilers from the boy-compilers.

This uses nothing known to be tricky or ambiguous. My question is: What should 
the answer be? Unfortunately, I don't have access to a "man-compiler" myself, 
asn do I was forced to try hand calculations. My conjecture (probably wrong) is 
that the answer will be

\[
73 - 119 - 177 + 102 = -121 .
\]

I'd be very glad to know the right answer.
\end{quotation}

The program is as follows:

\lstinputlisting[language=algol,frame=single]{chapters/dawn/knuth-man-boy-test.a68}

There are some useful applications of call-by-name semantics as well;
it is not \textit{solely} an avenue for confusion.
Jensen's Device, named in honor of Mr. J. Jensen of Regnecentralen, Copenhagen,
is a well-known application of call-by-name semantics.
This allows the user of a function to describe an expression to be evaluated
in a loop body inside the function, permitting a very general summation function.
There is not an especially elegant way to do this in a call-by-value language,
but as Dijkstra points out\cite{dijkstra_defense_of_algol_60_1961},
it can be implemented with this ALGOL program:

\lstinputlisting[language=algol,frame=single]{chapters/dawn/jensens-device.a68}

To simply sum the values of the array \texttt{V} from 1 to 100, one would call:
\texttt{Sum(i, 1, 100, V[i])}.
The real power of this device is its generality;
if a user were to need to compute $\sum_{i=1}^{100} V_{i}\times i$,
they could call \texttt{Sum(i, 1, 100, V[i]*i)}.
The first and final arguments to this function are not evaluated
until they are used in the body of the loop in \texttt{Sum}.
Others have also pointed out numerous easy-to-make mistakes
that arise because of this feature.
For example, a naive implementation of a swap function may easily
cause unintended side effects:

\lstinputlisting[language=algol,frame=single]{chapters/dawn/swap-error.a68}

While calling \texttt{Swap(i, j)} may work correctly, calling
\texttt{Swap(i+1, j+1)} or \texttt{Swap(A[i], A[j])} would not.

Among the now-unusual argument evaluation semantics, ALGOL also lacked any I/O 
facilities as we have already discussed.  The language also made string 
variables available, but provided no means of manipulating them.  ALGOL 
permitted \textit{own variables}, which are akin to static variables in C or 
SAVE variables in \FTN{}; they retain their values between entrances to the 
block.  Dijkstra also points out nowhere is the evaluation order of 
subexpressions specified\cite{dijkstra_defense_of_algol_60_1961}.

While the declared intent of defining a language detached from any particular
manufacturer or machine was laudable (and profitable in the case of portable 
languages to follow), ALGOL was so far removed from the machine to be useless without
supplementary libraries (for I/O, string processing, etc).

\section{ALGOL 68 (and X, Y, W, \dots)}

For years after the 1962 Rome meeting, there were discussions amongst the ALGOL
community about potential changes and improvements to the language.
The 1960 and 1962 meetings restricted themselves to clarifying ambiguities
and rectifying true errors in the prior reports, but not extensions or
significant changes.
In the March of 1964 as follow up to Rome meeting, they finally agreed to
start on ALGOL X and ALGOL Y, drafts of the language that would consider
significant changes and additions to the language.
ALGOL X would be intended to solve the issues of ALGOL 60 in the short-term,
compared to the "radically reconstructed future ALGOL Y"
\cite{cleaning_up_algol60_duncan_wijngaarden_1964}.

In the spring of 1968, a draft report of such an extension to the language 
was released with Adriaan van Wijngaarden as the editor.
The first programming language designed by international committee received
significant interest and is still used in academia today, however it never
gained significant traction.
Outside of small snippets in academic papers, it was only ever in use by
the Burroughs Corporation\cite{burroughs1963bac220},
and their implementation required significant library
development to provide their programers with an environment capable of doing
much of anything useful.

The computer and aerospace historian Paul Ceruzzi describes the result of
the ALGOL 68 report as follows:

\begin{quotation}
Whereas ALGOL-60 was based on a formal structure
and was very lean, ALGOL-68 was burdened by an attempt to do too
much, with the effects that some features interfered with the clean
implementation of others. It was hard to understand. In an attempt to
satisfy a broad range of users worldwide, the committee produced
something that satisfied few.
\end{quotation}

Charles Lindsey, British computer scientist and editor of
\citetitlecite{revised_report_on_the_algorithmic_language_algol_68_1976}
(and numerous other texts on ALGOL 68) describes popular sentiment about the
report\cite{a_history_of_algol_68_1993}:
\begin{quotation}
The world seems to have a rather negative perception of ALGOL 68. The language has been said to
be "too big," to be defined by an "unreadable Report" produced by a committee which "broke up in
disarray," to have no implementations, and to have no users. 
\end{quotation}

This was not his opinion, however\cite{a_history_of_algol_68_1993}:

\begin{quotation}
I should point out that my own involvement with the project came after the basic design of the
language, and of its original Report, were complete\dots
It is only now, in the course of studying the minutes and other
documents from that time, that I have come to see what the real fuss was about, and I hope that all
this has enabled me to take a dispassionate view of the events. The reader of this paper will certainly
see discord, but I believe he will see also how good design can win through in the end. 
\end{quotation}

Lindsey was appointed Lecturer in Computer Science at Manchester University
in 1967. Aside from his standardization efforts, he wrote a research implementations of
ALGOL 68 for MU5, one of the Manchester computers, and he maintained an
implementation of a subset of ALGOL 68, \textit{ALGOL 68S}.
It is difficult to take Lindsey's opinions about ALGOL 68 entirely at face
value because so much of his career was dedicated to the language.
If we look at those involved with the design of ALGOL 68 who
had programming language design and compiler experience elsewhere,
we find his depiction above to be more accurate than not.
% We will use Lindsey's portrayal of the language to illustrate the
% good parts of ALGOL 68 before looking to others' criticisms.

The development of ALGOL 68 was fraught with disagreement and controversy;
in 1965, Naur and Brian Randall of IBM even suggested that the incremental
ALGOL X be dropped entirely in favor of the more ambitious ALGOL Y.
Perhaps the most significant divergence ALGOL 68 took from ALGOL 60
and all programming language specifications since is its two-level grammar.
In the October 1965 meeting in St. Pierre de Chartreuse, three draft reports
were proposed for ALGOL X, one of which was Adriaan van Wijngaarden's
\citetitlecite{orthogonal_design_and_description_of_a_formal_language_wijngaarden_1965}.
This report introduced a few interesting concepts that would in large part determine
the trajectory of ALGOL 68:

\begin{enumerate}
      \item A two-level grammar now known as a \textit{W Grammar} after Wijngaarden.
      \item "The combination of a minimal number of language concepts in an orthogonal way"
            \cite{a_history_of_algol_68_1993}.
      \item Expression orientation; everything is an expression and has a value.
\end{enumerate}

The meeting concluded that this report would be the basis for ALGOL 68.
Wijngaarden was asked to produce a more complete report based on
\citetitle{orthogonal_design_and_description_of_a_formal_language_wijngaarden_1965}
and find consensus with the authors of the other two draft reports.
At nearly every step, the committee ran significantly past their schedules:
"Throughout the whole project, the WG in general, and Van Wijngaarden in particular, consistently
underestimated the time it would take by substantial factors. Recall that ALGOL 60 was put together
in six days"\cite{a_history_of_algol_68_1993}.

Another meeting in May 1967 in Zandvoort, Netherlands was intended to develop a direction for ALGOL Y,
however nearly all the discussion was about ALGOL X.
From the outset, the only language aspect the designers knew they wanted from ALGOL Y
was the ability for an ALGOL Y program to modify itself.
Lindsey remarks:

\begin{quotation}
They even spent an afternoon on ALGOL Y, from which it emerged that no one had 
very concrete ideas of what it was about, and that the only role model was 
LISP, on account of the fact that that language could construct its own 
programs and then \textit{eval} them. 
\end{quotation}

The largest single outpouring of criticism of ALGOL 68 came from a report
\citetitlecite{mr93_draft_report_on_algol_68_1968}, also known as MR93.
Several members of the committee (including Peter Naur) resigned
after this report's circulation.
It was first circulated to the \textit{ALGOL Bulletin} in February 1968,
and "was the cause of much shock, horror, and dissent, even
(perhaps especially) among the membership of WG 2.1"\cite{a_history_of_algol_68_1993}.
Numerous papers were written in response to MR93, including Peter Naur's \textit{scathing}
critique \citetitle{successes_and_failures_of_the_algol_effort_naur_1968}.
His report was especially critical of the International Federation for Information Processing
(IFIP), which had formed a working-group (WG 2.1) responsible for specifying the language.

Also in response to MR93, Lindsey joined the language effort and
distributed his \citetitlecite{algol68_with_fewer_tears_1968}
in which he attempted to synthesize the nicer language hidden within the earlier reports
\footnote{For those unaware, this title is an homage to the English textbook \textit{French without Tears}.}.
He clarified a number of terms which appear to be specific to ALGOL, and
have well-understood meanings under different names in other languages.
For example, what ALGOL calls a \textit{name} is more commonly known as a reference or pointer;
numerous differences in nomenclature contributed to the difficulty of reading
the original reports.

Van Wijngaarden had circulated another 
report\cite{penultimate_draft_report_on_algol_68_1968} in October of 1968 as he 
had earlier promised the committee; at the same time, other members who 
disagreed with the direction but had not resigned were also developing several 
minority reports, include Dijkstra, Hoare, and Randell.
Sparing the reader the full details and political history of the ensuing committee
meetings, the final report \citetitle{report_on_the_algorithmic_language_algol_68_mailloux_1969}
was published in \textit{Numerische Mathematik} and \textit{Kybernetika}
in 1969, thus specifying what we know today to be ALGOL 68.

\section{Concepts of ALGOL 68}

Setting aside the political machinery that resulted in the final report,
we will now look to the technical details of the language, its specification,
and therefore its compilers.

Firstly, ALGOL 68 is an expression-oriented language;
there is fundamentally no distinction between statements and expressions,
allowing for constructs like the following:

\begin{lstlisting}[language=algol,frame=single]
x := (real a = p*q;
      real b = p/q;
      if a>b then a else b fi)
      + (y := 2*z); 
\end{lstlisting}

There were numerous extensions to arrays, including higher-dimensional arrays
and complicated slicing mechanisms, such as the following:
\begin{lstlisting}[language=algol,frame=single]
loc[1:4, 1:5] int a45; 
a45[2  ,    ] # row 2 #
a45[   , 3  ] # column 3 #
a45[2:3, 3  ] # part of column 3 #
a45[2:3, 2:4] # a little square in the middle.  #
\end{lstlisting}

\todo{W Grammar complexity}
\todo{look at peck's \citetitlecite{the_algol_68_story_peck_1978}}

His work on ALGOL 68S was not an outlier amongst would-be implementers of the
language; because the report was so large and complex, most
implementers ended up restricting themselves to a subset of the language.

\section{Criticism of ALGOL 68}

\todo{Defined in Wijngaarden Grammar by Adriaan van Wijngaarden.
Contains parsing and things which in other langauges are called semantics.}

\todo{'68 critcized by Hoare and Dijkstra for abandoning simplicity of '60.
In 1970, ALGOL 68-R became the first working compiler for ALGOL 68.}

\section{Legacy of ALGOL 68}

\todo{ALGOL 68 also had a notable influence in the Soviet Union, details of which can be found in Andrey Terekhov's 2014 paper: "ALGOL 68 and Its Impact on the USSR and Russian Programming."}

\todo{Pascal, Ada}
\todo{Lindsey: "So here are my recommendations to people who essay to design programming languages."
\citetitlecite{a_history_of_algol_68_1993}}

\begin{quotation}
An interesting observation is the fact that the two-level formalism used
in the description of ALGOL 68 by itself led to a number of programming
languages, including PROLOG (Colmerauer) and my own CDL family (Compiler
Description Language).
\citetitle{a_shorter_history_of_algol68_1994}
\end{quotation}

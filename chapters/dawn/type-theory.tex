\section{Early Type Theory}

A surprising number of the foundations for type theory were laid before
many familiar programming concepts were developed.
Namely, Alonzo Church's \lambdacalc and Haskell Curry's combinatory logic
would set the foundation for type theory long before they would be implemented in
a proper compiler.

\subsection{Alonzo Church and the \Lambdacalc}

We will not given an exhaustive look at Church's life and work outside of the
\lambdacalc--I encourage readers to consult \citetitle{stanford_encyclopedia_church_2025}
for this.
He is renowned for more than just the \lambdacalc though we will not cover much more
than that here.

All throughout the 1930s, Church developed the \textit{\lambdacalc},
a formal definition of a \textit{higher-order, functional} programming language
built on only three concepts: functions, variables, and application.
The language was \textit{higher-order} in that functions could be passed
as arguments to other functions, and functions could return other functions as results.
It was \textit{functional} in that function definition and application was the
primary abstraction in the language.

In the \lambdacalc, functions are given by $\lambda x. E$ where $x$ is the function's argument
and $E$ is the resulting expression. The result of the function is given by substituting the
argument for all occurances of $x$ in $E$. Function application is given by juxtaposition,
so $F x$ applies the function $F$ to the argument $x$.

Function definitions and applications are all assumed to take exactly one argument,
though there are syntaxes given for more:

\begin{align}
	\lambda x . (\lambda y . (\lambda z . E)) & \equiv \lambda x y z . E
	\tag{function definition}
	\\
	M N P x                                   & \equiv M (N (P x))
	\tag{function application}
\end{align}

In 1940, he published \citetitle{church_simple_theory_of_types_1940}
which would become the foundation for type theory.

The \lambdacalc

\subsection{Haskell Curry}

\todo{not sure how much we want to cover here}...
\citetitle{curry_functionality_in_combinatory_logic_1934}.

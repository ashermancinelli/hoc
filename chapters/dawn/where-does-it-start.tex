\section{Where Does it Start?}

There is significant debate about who created the first compiler,in no small
part due to ambiguous nomenclature. The term \textit{software} came into use
sometime between 1959 and 1962, as \citeauthor{the-first-computers-2002} note:

\begin{quotation}
	[Expressions] such as "hardware", "software", "machine language",
	"compiler", "architecture" and the like... were unknown in 1950.
	They only arrived a decade later, but the underlying concepts were
	quite familiar to us.
	\cite{the-first-computers-2002}
\end{quotation}

This Honeywell advertisement \textit{A Few Quick Facts on Software} sought to
clarify these terms as well:

\begin{quotation}
	Software is a new and important addition to the jargon of computer
	users and builders. It refers to the automatic programming aids that
	simplify the task of telling the computer 'hardware' how to do its
	job.
	\cite[ch.5]{new-history-of-modern-computing}
\end{quotation}

At this time, hardware was the only piece that mattered to customers. Software
was an afterthought, if a thought at all. The instruction set of the machine was
important, because that was the user's interface to the machine. It should come
as no surprise, then, that the origins of our modern understanding of the term
\textit{compiler} are similarly murky, especially considering the fact that
\textit{compiler} already carried meaning in English, and was repurposed for
computing. John Backus even pointed out how the ambiguity around the term
\textit{compiler} makes computing history circa 1950s especially difficult to
untangle \cite{Backus_1980_Programming_in_America_in_1950s}:

\begin{quotation}
	There is an obstacle to understanding, now, developments in
	programming in the early 1950s. There was a rapid change in the
	meaning of some important terms during the 1950s. We tend to assume
	that the modern meaning of a word is the same one it had in an early
	paper, but this is sometimes not the case. Let me illustrate this
	point with examples concerning the word "compiler."
\end{quotation}

\bigskip
Once could argue that any of these efforts constituted the first compiler:
\begin{itemize}
	\item Konrad Zuse's run-programs for the Z4 at the ETH Zurich in 1950
	\item Grace Hopper's A-0 and A-1 compilers for the UNIVAC I at
	      Remington Rand in 1951
	\item Laning and Zierler's algebraic compiler for the Whirlwind at MIT in 1950
	\item John Backus's \FTNI{} compiler at IBM
\end{itemize}

I attempt to discuss these in order, however their efforts overlap
significantly in time. I try to tell their stories as a whole, though each story
contains references to the others;if you find yourself confused by names
introduced out of context, please finish the chapter or search for the content
within this chapter before giving up.

\section{Development of the Z4}

Konrad Zuse, a German civil engineer, began work on the Z4 during World War
II. Funded partially by his family and partially by the Nazi government, his
prior works demonstrated significant creativity and ingenuity, and they were
leveraged to build precursors to modern cruise missiles and guided bombs. Most of
Zuse's machines prior to the Z4 were destroyed during the war. \todo{Zuse's Z4
	was a strange machine with bespoke memory and instruction set. This
	affected how
	the compilers for it were designed.}

In a turn of events Konrad Zuse was made
aware of Aiken's Mark I through his daughter
\cite{howard_aiken_and_the_dawn_of_the_computer_age_2000}:

\begin{quotation}
	Konrad Zuse told me an amusing anecdote about how he first
	encountered the work of Aiken. The occasion of our conversation was
	a luncheon in Zuse's honor, hosted by Ralph Gomory at the Watson
	Research Laboratory of IBM before a lecture given by Zuse to the
	staff of the lab. When Zuse learned that I was gathering materials
	for a book on Aiken, he told me that he had come across Aiken and
	Mark I in an indirect manner, through the daughter of his
	bookkeeper. She was working for the German Secret
	Service (Geheimdienst) and knew through her father of Zuse's work on
	a large scale calculator. According to Zuse,the young woman never
	learned any details about his machine, which was shrouded in
	war-time secrecy. But she knew enough about Zuse's machine to
	recognize that the material filed in a certain drawer related to
	advice that seemed somewhat like Zuse's. She reported this event
	to her father, giving the file number of the drawer, and the father
	at once informed Zuse of her discovery. Zuse, of course, could not
	go to the Secret Service and ask for the document since that would
	give away the illegal source of his information. Zuse was well
	connected, however, and was able to send two of his assistants to
	the Secret Service, armed with an official demand for information
	from the Air Ministry, requesting any information that might be in
	the files concerning a device or machine in any way similar
	to Zuse's. Zuse's assistants were at first informed that no such
	material existed in the files, but they persisted and eventually got
	to the right drawer. There they found a newspaper clipping (most
	likely from a Swiss newspaper), containing a picture of Mark I and a
	brief description about Aiken and the new machine. But there was not
	enough technical information to enable Zuse to learn the machine's
	architecture.
\end{quotation}

\section{The ETH's Acquisition of the Z4}

There were several early efforts to create programs that produced punch
cards which contained machine code instructions, which could then be fed back
into the machine as input punchcards. The programs produced by these early
compilers were called \textit{run-programs}, and the process of using them was
called \textit{automatic programming}, a term later coined by Grace Hopper. The
first of these programs was run on a machine called the Z4, designed by Konrad
Zuse in Germany.

Professor Eduard Stiefel, shortly after establishing the
Institute of Applied Mathematics to study numerical analysis at the Swiss
Federal Institute of Technology (ETH) in Zurich, began searching for a computer
for the institute. He learned of the computing advancements in the United
States, Great Britain, and Germany,but no machines were readily available at
the time. He sent his assistants Heinz Rutishauser and Ambros P. Speiser to the
US to study the latest developments in computing; they spent most of 1949
with Howard Aiken at Harvard and John von Neumann at Princeton.

\begin{quotation}
	Before we returned, that is, in the middle of 1949, Stiefel was informed
	about the existence of Konrad Zuse's Z4. At that time Zuse was living in
	Hopferau, a German village near the Swiss border. Stiefel was told that the
	machine might be for sale. He visited Zuse, inspected the device, and reviewed
	the specifications. Despite the fact that the Z4 was only barely
	operational, he
	decided that the idea of transferring it to Zurich should by all means be
	considered. Stiefel wrote a letter to Rutishauser and me (we were at Harvard at the time),
	describing the situation and asking us to get Aiken's opinion.
	Aiken's reply was very critical - the future belonged to electronics
	and, rather
	than spending time on a relay calculator, we should now concentrate our efforts
	on building a computer of our own.
	\cite{konrad-zuses-z4-2000}
\end{quotation}

The Z4 was a bespoke machine with unique components; the computational logic
was wired together with telephone relays and the memory was entirely mechanical.

\begin{quotation}
	The Z4 could be used as a kind of manually triggered calculator: the
	operator could enter decimal numbers through the decimal keyboard, these
	were transformed into the floating point representation of the Z4, and were
	loaded to the CPU registers, first to OR-I, then to ORII. Then, it was possible
	to start an operation using the "operations keyboard" (an addition, for
	example). The result was held in OR-I and the user could continue loading
	numbers and computing. The result in OR-I could be made visible in decimal
	notation by transferring it to a decimal lamp array (at the push of a button).
	It could also be printed using an electric typewriter.
	\cite{architecture-of-konrad-zuses-z4-computer-2021}
\end{quotation}

It notably featured instructions for conditional branching and subroutine
calling, which both proved essential for the compiler development that would
follow at the ETH. Stiefel was undeterred by Aiken's criticism, and convinced
the ETH to purchase the Z4. In 1950, Heinz Rutishauser at Switzerland's ETH
obtains a Z4.

\begin{quotation}
	We also made some hardware changes. Rutishauser, who was exceptionally
	creative, devised a way of letting the Z4 run as a compiler, a mode
	of operation
	which Zuse had never intended. For this purpose, the necessary
	instructions were
	interpreted as numbers and stored in the memory. Then, a compiler
	program calculated the program and punched it out on a tape. All this required
	certain hardware changes. Rutishauser compiled a program with as many as 4000
	instructions. Zuse was quite impressed when we showed him this achievement.
	\cite{konrad-zuses-z4-2000}
\end{quotation}

Thus were the first run-programs produced. This is what we will
consider \textbf{the first compiler}, though it was not called that
at the time. Shortly after Stiefel's assistants' stints in the US and
correspondence with Aiken,one of Aiken's engineers would find
considerably more success exploring related ideas.

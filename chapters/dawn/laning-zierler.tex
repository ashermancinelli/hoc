

\section{Laning and Zierler at MIT}
\label{sec:laning-zierler}

Of the early compiler efforts, Laning and Zierler's team is perhaps the most overshadowed.
Their contemporaries were very impressed by their work, and they inspired a number
of innovations at Remington Rand and IBM.
Backus described the pseudocode compilers of the time (such as the A-2)
as merely providing an instruction set
slightly different from the machine's actual code, but not providing any real abstraction;
writing the pseudocode still tedious and unproductive.
Laning and Zierler's work was different.

John Backus was hugely supportive of their work, though he was not directly invluenced by it
prior to starting work on \GLS{ftn}.
In Backus's 1980 paper on the programming landscape of the 1950s
\citetitlecite{Backus_1980_Programming_in_America_in_1950s}, he recalls:

\begin{quotation}
	Very early in the 1950s, J. Halcombe Laning, Jr., recognized that
	programming using algebraic expressions would be an important improvement.
	As a result of that insight he and Neal Zierler had the first
	algebraic compiler running on WHIRLWIND at MIT in January 1954.
	(A private communication from the Charles Stark Draper Laboratory indicates
	that they had demonstrated algebraic compiling sometime in 1952!) The priesthood
	ignored Laning's insight for a long time. A 1954 article by Charles W. Adams
	and Laning (presented by Adams at the ONR symposium) devotes less than
	3 out of 28 pages to Laning's algebraic system; the rest are devoted to other
	MIT systems. The complete description of the system's method of operation
	as given there is the following
\end{quotation}

In this quote, the \textit{priesthood} Backus is referring to is the group of programmers
who rejected the idea that programming should or could be done in a higher level language,
and that programming in machine code is an art that would need to be preserved.
On his own side of the argument, Backus found Laning and Zierler's work on compiling
algebraic expressions into programs to be more accessible and efficient.
Grace Hopper was on this side of the argument as well.

Donald Knuth and Trabb Pardo also recall the importance of their work in
the development of programming languages and compilers
in \citetitlecite{Knuth_TrabbPardo_1976_Early_Development}:

\begin{quotation}
	In retrospect, the biggest event of the 1954 symposium on automatic
	programming was the announcement of a system that J. Halcombe Laning, Jr. and Niel Zierler
	had recently implemented for the Whirlwind computer at M.I.T.  However, the
	significance of that announcement is not especially evident from the published
	proceedings [NA 524-], 97\% of which are devoted to enthusiastic description s
	of assemblers, interpreters, and 1954-style "compilers".
\end{quotation}

Clearly, their work was highly influential.

\begin{figure}[h!]
	\centering
	\includegraphics[width=0.5\linewidth]{resource/knuth_pardo_on_laning_zierlers_algebraic_compiler.png}
	\caption{Knuth and Trabb Pardo on Laning and Zierler's Algebraic Compiler}
	\label{fig:knuth-pardo-on-laning-zierler}
\end{figure}

\begin{quotation}
	The first programming system to operate in the sense of a modern compiler was
	developed by J. H. Laning and N. Zierler for the Whirlwind computer at the
	Massachusetts Institute of Technology in the early 1950s. They described their
	system, which never had a name, in an elegant and terse manual entitled "A
	Program for Translation of Mathematical Equations for Whirlwind I,"
	distributed
	by MIT to about one-hundred locations in January 1954.26 It was, in John
	Backus's words, "an elegant concept elegantly realized." Unlike the UNIVAC
	compilers, this system worked much as modern compilers work; that is, it took
	as its input commands entered by a user, and generated as output fresh and
	novel machine code, which not only executed those commands but also kept track
	of storage locations, handled repetitive loops, and did other housekeeping
	chores. Laning and Zierler's "Algebraic System" took commands typed
	in familiar
	algebraic form and translated them into machine codes that Whirlwind could
	execute.27 (There was still some ambiguity as to the terminology: while Laning
	and Zierler used the word "translate" in the title of their manual, in the
	Abstract they call it an "interpretive program.")28
	\cite{new-history-of-modern-computing}
\end{quotation}


\todo{Backus was a huge fan of thier work. Look at \citetitle{hopl_backus_history_of_fortran}.}
\todo{ Laning and Zierler, 1954. Early 1950s. Inspiration for Backus/\FTN{}.
	Worked more like a modern compiler than Hopper's A-0 and A-1.  }

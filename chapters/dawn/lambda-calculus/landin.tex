\subsection{Peter Landin on the Lambda Calculus}
\label{sec:landin-lambda-calculus}

Exemplary of the varied approaches to formalizing the design of ALGOL 60,
Peter Landin sought to express the language's semantics in the
\lamc in \parencite{landin_algol_lambda_1965}
\footnote{See \cref{sec:algol60} for more details on the development of ALGOL 60.}.
A few years earlier, McCarthy adopted some components of the \lamc in
Lisp, however McCarthy's language broke with \lam in a few key areas,
namely \gls{dynamic-binding}
\footnote{ For a more complete treatment of Lisp, see \cref{sec:lisp}. }.
ALGOL's semantics provided a much cleaner relationship to \lam, thanks to
its block structure and lexical scoping rules, thus Landin
made it possible to look at \lam as a programming language in and of itself
in a more complete way than McCarthy had done with Lisp.

In parallel with his translation of ALGOL, Landin developed in
\parencite{landin_eval_of_expressions_1964} an abstract machine for
\lam called the \textit{SECD-machine}.

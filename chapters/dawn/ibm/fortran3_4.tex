
\subsection{FORTRAN III}

While Lois was carrying out Ziller, Backus and Nelson's design for FORTRAN III,
Ziller was already working on a more advanced version of FORTRAN, FORTRAN III.
Ziller's design incorporated a form of \gls{inline-assembly} that allowed
704 instructions to take the addresses of FORTRAN variables as arguments.
Modern forms of inline assembly allow the user to write assembly inside \textit{text}
in the host program--Ziller made the unfortunate decision of making IBM 704
instructions available \textit{at the language level}, which, as Backus remarked
\cite{hopl_backus_history_of_fortran}, doomed FORTRAN III to die whenever the IBM 704
was replaced.

Ziller and the team also introduced boolean expressions and the capability to
pass functions and subroutines as arguments and FORMAT codes for printing
alphanumeric strings.
This version of FORTRAN was never widely used and was only distributed to
20 or so installations in the winter of 1958.
It was in operation until the 1960s using the IBM 709's compatibility mode,
which kept FORTRAN III's machine-specific IBM 704 features alive for a bit longer
than the 704 itself.

\subsection{FORTRAN IV}

By 1958, most of the original FORTRAN team had moved on to other work and
some members of the original team, especially John Backus, were unhappy with
the direction they took.
The following edition, FORTRAN IV, was more of a successor to FORTRAN II than
it was to FORTRAN III given the latter's machine-specific features and
lack of adoption.

To address the long compilation times that users of FORTRAN II dealt with,
the team Programming Research Department responsible for FORTRAN IV
tried to deliver the best of both worlds by introducing a new compiler
that was both faster and generated more efficient code, instead of adding
different compilation modes. Modern compilers let the user choose to enable
optimizations via command-line flags. For example, most C, C++ and FORTRAN compilers
provide the \texttt{-ON} flag, where \texttt{N} is a number
between 0 and 3 specifying how aggressive the compiler ought to be.

As a result, IBM's FORTRAN IV compiler was slower than other fast FORTRAN compilers
like the \textit{WATFOR} and \textit{WATFIV} compilers developed at the University of Waterloo
(the \textbf{WAT}erloo \textbf{FOR}tran compiler\cite{cress_dirksen_graham_watfor_fortran_iv_1970}),
and in general it did not produce machine code as fast as IBM's FORTRAN II compiler.

\todo{fortran iv was revamped ii compiler, introduced common blocks.}
\todo{See 1964 backus and heising paper}.
\cite{backus_heising_fortran_1964}.

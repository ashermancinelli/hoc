\subsection{SHARE}
\label{subsec:share}

At this point in FORTRAN's history, there was such significant adoption of
many of IBM's innovations that a volunteer user's group, SHARE, had grown to
significant membership.
SHARE was founded in 1955 and gained traction shortly after \cite{akera_voluntarism_ibm_share_2001},
made up of IBM customers managing installations of the 704,
and collectively aimed to deduplicate the effort of managing the system
\cite{armer_share_eulogy_1980}.

This group was an early precursor to the open-source movement.
Prior to this period, most machines were used for accounting and were for
highly specialized machines that constituted hardly more than a calculator.
Once more advanced machines were introduced, the task of writing software
and maintaining the hardware became increasingly difficult.

Out of this difficulty, organized user groups like SHARE emerged alongside
ad-hoc collaboration, like Grace Hopper's collaboration with
\todo{some air force group I think? where they would send her fixes and enhancements
	to the compilers.}

\subsection{FORTRAN III}

While Lois was carrying out Ziller, Backus and Nelson's design for FORTRAN III,
Ziller was already working on a more advanced version of FORTRAN, FORTRAN III.
Ziller's design incorporated a form of \gls{inline-assembly} that allowed
704 instructions to take the addresses of FORTRAN variables as arguments.
Modern forms of inline assembly allow the user to write assembly inside \textit{text}
in the host program--Ziller made the unfortunate decision of making IBM 704
instructions available \textit{at the language level}, which, as Backus remarked
\cite{hopl_backus_history_of_fortran}, doomed FORTRAN III to die whenever the IBM 704
was replaced.

Ziller and the team also introduced boolean expressions and the capability to
pass functions and subroutines as arguments and FORMAT codes for printing
alphanumeric strings.
This version of FORTRAN was never widely used and was only distributed to
20 or so installations in the winter of 1958.
It was in operation until the 1960s using the IBM 709's compatibility mode,
which kept FORTRAN III's machine-specific IBM 704 features alive for a bit longer
than the 704 itself.

\subsection{FORTRAN IV}

By 1958, most of the original FORTRAN team had moved on to other work and
some members of the original team, especially John Backus, were unhappy with
the direction they took.
The following edition, FORTRAN IV, was more of a successor to FORTRAN II than
it was FORTRAN III given the latter's machine-specific features and
lack of adoption.
FORTRAN's adoption had grown so large and the set of features so restrictive
that in 1961 the team at IBM decided to create a new compiler conceptually
based on the FORTRAN II compiler, but completely rewritten.
They used the opportunity not only to introduce new features but also to
correct some design decisions made in the FORTRAN II compiler.

To address the long compilation times that users of FORTRAN II dealt with,
the team Programming Research Department responsible for FORTRAN IV
tried to deliver the best of both worlds by introducing a new compiler
that was both faster and generated more efficient code, instead of adding
different compilation modes. Modern compilers let the user choose to enable
optimizations via command-line flags. For example, most C, C++ and FORTRAN compilers
provide the \texttt{-ON} flag, where \texttt{N} is a number
between 0 and 3 specifying how aggressive the compiler ought to be.

As a result, IBM's FORTRAN IV compiler was slower than other fast FORTRAN compilers
like the \textit{WATFOR} and \textit{WATFIV} compilers developed at the University of Waterloo
(the \textbf{WAT}erloo \textbf{FOR}tran compiler\cite{cress_dirksen_graham_watfor_fortran_iv_1970}),
and in general it did not produce machine code as fast as IBM's FORTRAN II compiler.

The team introduced the labeled COMMON block in FORTRAN IV, which allowed
seperably compiled modules to share data with each other,
and required that each module agree on the size and layout of the data.
Programmers found this useful because they could share data between modules
without needing to recompile the entire program after every change
(unless the layout of the common block changed!),
but there are lots of mistakes that are hard to avoid with COMMON blocks.

For example, take the following subroutines, and assume they are
defined in different files (and thus different \gls{translation-unit}s):

\begin{lstlisting}[language=fortran,frame=single]
! f.F
subroutine f()
  real n
  real A(10)
  common /lab/ A, n
  ! uses of A and n...
end subroutine

! g.F
subroutine g()
  real n
  real A(5)
  common /lab/ n, A
  ! uses of A and n...
end subroutine
\end{lstlisting}

Notice how the declarations of \texttt{A} and \texttt{n} differ between the two
subroutines. The array \texttt{A} is declared with a different size in each subroutine,
and the order of the variables is different!
If the programmer of the subroutine \texttt{g} assigns to \texttt{n}, they
will be assigning to the same memory referred to by the first element of \texttt{A}
in the subroutine \texttt{f}.

The team went to great lengths to make FORTRAN IV a superset of FORTRAN II
so all the existing users could keep compiling their code with the newest
compilers, but there were some features they just couldn't support.
The IBM user's group \textit{SHARE}\cref{subsec:share} came up with a
translation program \textit{SIFT}, or
\textit{S}HARE \textit{I}nternal \textit{F}ORTRAN \textit{T}ranslator,
to help users upgrade their FORTRAN II code to FORTRAN IV.
\footnote{
	Jean Sammet discusses this in \citetitle{sammet_programming_languages_history_and_fundamentals_1969}
	and cites the article \textit{SHARE Internal FORTRAN Translator} in
	Volume 9 of \textit{Datamation} in 1963, but I'm unable to track this
	edition down myself, so we take Sammet's word for it.}
The term \gls{sift} grew to become a more general term for this sort of
translation--there were many efforts in this direction in the Python 2 to
3 upgrading process.

The first instance of this process was before the term \gls{sift}
existed or was used to mean translation more generally, with
the ALTAC to FORTRAN II translator \cite{olsen_altac_fortranii_translator_1965}.

\todo{fortran iv was revamped ii compiler, introduced common blocks.}
\todo{See 1964 backus and heising paper}.
\cite{backus_heising_fortran_1964}.


\subsection{The IAL and the ALGOrithmic Language}

Before Hopper, Mayes and Phillips pulled together their committees for a common
business language, design-by-committee was already in the academic milieu,
especially in Europe.
While not commercially successful, ALGOL introduced a number of important concepts
like block scopes and the declaration of the types of variables, and it would
go on to be the standard language for describing algorithms in academia.
Originally called IAL (International Algebraic Language), it came to be called
ALGOL, or ALGOrithmic Language, and was designed by an international committee
with representatives from different organizations with the goal of a truly
machine-independent language.

After deliberation in numerous committees, representatives from the German
Association for Applied Mathematics and Machinery (GAMM) and the ACM met in
Zurich, Switzerland in the summer of 1958. They had both produced similar
reports and wanted to meet and agree on a unified language.
John Backus, Charles Katz, Alan Perlis and Joseph Wegstein from the ACM attended this meeting.
\todo{who were these people? Add some narrative. Maybe of Naur, then tie in with Backus.}
They arrived at the following objectives
\cite{perlis_samelson_1958_preliminary_report_ial}:

\begin{enumerate}
	\item The new language should be as close as possible to standard mathematical
	      notation and be readable with little further explanation.
	\item It should be possible to use it for the description of computing processes in publications.
	\item The new language should be mechanically translatable into machine programs.
\end{enumerate}

Shortly thereafter, a large number of dialects and partial implementations sprung up around Europe
and the US such as BALGOL from Burroughs Corporation in Detroit, Michigan for the Burroughs 220.
Manufacturers such as Burroughs found the standard to be insufficient for their users:
"BAC-220 provides additions for the ALGOL reference language which are essential to
the operation of data-processing systems: input-output
facilities, conventions for inclusion of segments of machine-language coding,
and diagnostic features" \cite{burroughs1963bac220}.
This was intentional; the specifications of ALGOL (both the 1958 and the 1960 versions)
was solely for the purposes of
describing computation; no I/O or system libraries were specified.
Other dialects included CLIP, JOVIAL, MAD, and NELIAC.
The first issue of the \textit{ALGOL Bulletin} was issued in March of 1959 out of Copenhagen
with Peter Naur as the editor.

Jean Sammet describes the impact of ALGOL 58
\cite{sammet_programming_languages_history_and_fundamentals_1969}:
\begin{quotation}
	Among the more intriguing technical features of ALGOL 58 were its essential
	simplicity; the introduction of the concept of three levels of language, namely
	a reference language, a publication language, and hardware representations; the
	\textbfit{begin\dots end} delimiters for creating a single (compound) statement from simpler
	ones; the flexibility of the procedure declaration and the \textbfit{do} statement for
	copying procedures with data name replacement allowed; and the provision for
	empty parameter positions in procedure declarations. While ALGOL 58 is not an
	exact subset of ALGOL 60, the only items of significance which are in the
	former but not the latter are the \textbfit{do} which was removed as a concept (although
	the word was used for something else) and the empty parameter positions.
	Because of this major carry-over, specific technical description of ALGOL 58 is
	not necessary.
\end{quotation}

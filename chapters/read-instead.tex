\chapter{What to Read Instead}

This work is highly derivative.
Titans of the computing and information sciences have written extensively
about the history of computing and programming languages.
While this collection has a unique focus on compilers, there are a number
of fantastic related works that readers may want to consider instead.

John Backus's \citetitlecite{Backus_1980_Programming_in_America_in_1950s}
and \citeauthor{Knuth_TrabbPardo_1976_Early_Development}'s
\citetitlecite{Knuth_TrabbPardo_1976_Early_Development} are particularly instructive
about the very early days of programming languages.
Backus has a unique voice--he "talks like a regular guy"--and he wrote many of
his thoughts on the average programmer's experience in the 1950s.
Similar works on the early days of programming languages tend to reference
Knuth and Pardo's work extensively; I found that after reading their book,
many of those other works covering the same time period felt derivative.

Jean Sammet's \citetitlecite{sammet_programming_languages_history_and_fundamentals_1969}
should not be overlooked. If this work is at all interesting to the reader, they
will no doubt find tremendous value in Sammet's works, this one in particular.

In general, when discussing topics where I feel another work may interest or
inform the reader more than this one, I attempt to point it out.

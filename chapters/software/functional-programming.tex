\section{Functional Programming}

One framework for understanding trends in programming language design
goes like this:

\begin{enumerate}
	\item Something happens in the world that results in lots of money being spent on
	      programming languages and software.
	\item Lots of academics use this money to think about how to design programming languages.
	\item They converge on mathematical (in particular, algebraic)
	      approaches to programming languages design,
	      and these ideas gain traction.
	\item That money dries up, and software developers generally move
	      away from algebraic approaches to programming languages design,
	      and sometimes go in the \textit{opposite} direction towards
	      practicality at all costs.
\end{enumerate}

I don't necessarily believe this and macro trends are hard to prove one
way or another, but sometimes it's a useful lens.

Perhaps the first instance of this trend was Laning and Zierler's algebraic compiler
which formed during and immediately after the Second World War, when computing was
in its infancy and there was lots of government funding for research into programming languages.
This culminated in the development of ALGOL with it's principle of orthogonality of language
features.
This line of thinking eventually gave way to more practical and less principled approaches to
language design found in Fortran and C.

Another example might be the development of ML\dots \todo{...}

\section{The Next 700 Programming Languages}

Published in \citeyear{landin_next_700_prog_langs_1966},
Peter Landin's seminal paper \citetitlecite{landin_next_700_prog_langs_1966} lays out his vision for
the future of programming language design, emphasizing the importance of the programmer's intent
uncluttered by details of the hardware.
He argued that the programmer ought to only consider their intent, and the compiler ought to
consider the operations that would be needed to carry out their intent.

The paper describes a new programming language called IYSWIM, for \textit{If You See What I Mean}.
\todo{dig into ML languages \citetitlecite{hopl_history_of_ml_2020}}.
\todo{type systems, type inference, Hindley Milner, SML.}
\todo{similar vein to Laning and Zierler's algebraic compiler.}

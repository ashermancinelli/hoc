
\section{Commercialization}

Once IBM split off a new group to sell software, it started to become
evident that entire business models could be built off of software.
Though few understood it at the time, the software business model
would become among the most lucrative in the world due to the nature of software.

If a carpenter designs and crafts a chair, then they can sell that one chair,
and one person gets value from it.
When a customer goes to a resturant and pays the waitstaff and cooks for their
time and the ingredients, they get one meal's value out of their labor.
Software is fundamentally different.
Rather than providing a good or a service, the process of writing software
is far closer to writing sheet music.
A programmer can write out a blueprint, or a script, which defines actions
from which someone may derive value, and that blueprint may be replicated
once, twice, or a trillion times.

Of course there are maintenance costs associated with software, and at some point,
someone has to manage the hardware as well.
But fundamentally, the potential value someone can derive from software
has a very high upside relative to the labor it takes a programmer to produce
it. It is in this time period that a few companies started to discover the
commercial value of selling software.

\subsection{Microsoft}

Microsoft would come to be so astronimically large that it is almost easy
to lose sight of their impact. Because they constitute an entire
developer ecosystem, their influence sometimes does not escape the walls
of their garden.

\todo{Bill Gates and Paul Allen (Microsoft) | Microsoft BASIC (1975) |
	Developed the first critical piece of commercial software for personal
	computers,establishing the doctrine that software should be a purchased,
	proprietary commodity. Sun microsystems, each part of the company needed to sell
	to all the others,reason why their compiler was paid; proprietary Unix;}

\subsection{Sun Microsystems}

Sun had three core folks that made their business special: a brilliant engineer, software architect, and marketing...
Bill Joy Spark and Solaris.

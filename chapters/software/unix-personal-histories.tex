
\section{Personal Histories}

The following sections will feature many of the same figures;
here we will introduce them so their interleaving stories may be told
uninterrupted in the following sections.
Readers unfamiliar with the history of Unix may find unfamiliar terms in these
personal accounts; these will be discussed in the following sections after
our characters have been introduced.

\subsection{Doug McIlroy, Joined 1958}

Douglas McIlroy was born in Fishkill, New York.
His father worked in electrical engineering and spent time at
MIT and ended his career at Cornel, having contributed to the RADAR effort in World War II.
His father collected maps, and Doug grew an interest in maps himself.
One day, while ill in bed with chicken pox, his father assigned him the task of drawing
a dam on a USGS map, and showing which regions would be inundated by the dam
\cite{doug_mcilroy_oral_history_2019}.
His mother, too, had a master's degree in physics from the University of Rochester,
which was very unusual at the time. She was forced to audit some of her classes there,
because "women can't take this class! But you can sit in on it, if you want."
Thus he was raised in an engineering-minded household.

Barbara, his wife, also studied mathematics and faced discouragement similar to his mother's.
They ended up meeting at the laboratory, where Doug recalled that some figures at the labs sought
to improve the situation for women, specifically Bernie Holbrook.

Doug joined Bell Labs in 1958 and shortly thereafter earned his
PhD in applied mathematics from MIT.
McIlroy became head of the Computing Techniques Research department in 1965, and he's regarded
as one of the most brilliant members of the staff by key members that readers may be more familiar with.

He is described by nearly everyone at the Bell Labs Computing Research Center
(who wrote or spoke about their time there)
as the most brilliant member of that team that no one has heard of
\cite{kernighan_interviews_thompson_2019}; Ken Thompson described him as
"the smartest one of all of us and the least remembered or written down of all of us."

He was relatively hands-off as a manager, preferring to peek into his employees' offices
with suggestions or interesting problems and wait for the employees to let \textit{him} know
what needed to be done, rather than assigning tasks directly.
His employees respected his tasted in technical and personal matters,
as Brian Kernighan describes\cite{kernighan_unix:_2020}:

\begin{quotation}
	Unix might not have existed, and certainly
	would not have been as successful, without Doug's good taste and sound
	judgment of both technical matters and people.
\end{quotation}

\begin{figure}
	\centering
	\includegraphics[width=0.7\textwidth]{resource/software/unix/doug-1964-pipes.png}
	\caption{Doug McIlroy's 1964 memo proposing Unix pipes\cite{doug_mcilroy_origin_of_unix_pipes_1964}.}
	\label{fig:unix-pipes-mcilroy-memo}
\end{figure}

In 1964, he circulated a memo which led to Unix pipes\cite{doug_mcilroy_origin_of_unix_pipes_1964}
(see \ref{fig:unix-pipes-mcilroy-memo}) though it took him quite a while to convince
Ken Thompson that he really ought to implement them.
Aside from pipes, he wrote many common Unix utilites we still use today, like
\texttt{diff}, \texttt{sort}, \texttt{join}, \texttt{tr}, \texttt{echo}, \texttt{tee}, and \texttt{spell}.

He hired Alfred Aho in 1967\cite{aho_oral_history_2022}:

\begin{quotation}
	I was interviewed by a department head by the name of Doug McIlroy. He was an applied
	mathematician from MIT. He had been at Bell Labs for a few years before me. Amongst other things, he
	had co-invented macros for programming languages and he's also in this class of one of the smartest
	people I've ever met.
\end{quotation}

\subsection{Ken Thompson, Joined 1966}



\subsection{Dennis Ritchie, Joined 1967}


\subsection{Jeffrey Ullman, Joined 1967}


\subsection{Alfred Aho, Joined 1967}

Alfred grew up in a Finnish household, and when he showed up to Kindergartne,
he couldn't speak any English.
In his first report card, the teacher told his parents that he couldn't speak and English,
and in the next report card, the teacher said he spoke \textit{too much} English;
so he learned it relatively late and became quite a social kid.
Alfred played music all throughout his childhood, and the schools he went to
as a kid had quite good musical programs.
He went to University of Toronto for undergraduate school and continued playing
the violin all through his years at Bell Labs. He was an only child. He had a
proclivity not only for music but for mathematics and reading science fiction.

After finishing his undergraduate degree in Toronto in engineering physics,
he got a masters and then PhD
from Princeton University in electrical engineering and computer science,
completing the latter in 1967.
At Princeton took a course from John Hopcroft on computer science theory with a heavy emphasis on
automata and language theory, which sparked his long-lasting interest in formal languages.
His PhD thesis was on extending the theory of context-free grammars, an area of
expertise that would serve him well at the Labs.

One of the fist people he met at Princeton was Jeffrey Ullman, who had also recently
started there, beginning a long-term friendship and collaboration between the two
\cite{aho_oral_history_2022}:

\begin{quotation}
	One of the first people that I met at Princeton was a Columbia graduate by the name of Jeffrey
	Ullman. He had just gotten his undergraduate degree from Columbia University and also had come to
	study digital systems in the EE department at Princeton. So, he and I became close friends. When we
	graduated from Princeton, we both joined the newly formed Computing Sciences Research Center at Bell
	Labs. There we developed a lifelong collaboration on subjects ranging from algorithms, programming
	languages, to the very foundations of computer science. I was very fortunate to have met some of the
	greatest people in the field and to have gotten to know them and work with them.
\end{quotation}

His PhD advisor was John Hopcroft, whom he would also continue to work with for a long time.
He told Alfred to "find his own research problem," and he turned to his interest in
programming languages and compilers.
He was always very keen on precision; he wanted to use precise terms and he would
press even his own friends to be very precise in their informal discussions, and
this acuity for precision extended to his thesis, which was on the very precise
understanding of the \textit{syntax} and the \textit{semantics} of programming languages.
Alfred would contribute his deep understanding of the theory of automata and programming
languages to the Labs, where there were many other brilliant people who were more practical
and lacked familiarity with the literature.
In 1967 his good friend Jeffrey Ullman had started working at Bell Labs, and a few months
after starting there, Alfred interviewed there.
He was interviewed by Doug McIlroy, who hired him.
For some time, he and Jeffrey worked together at the Labs, but Jeffrey had wanted to
spend more time in academia than Alfred, so he returned to Princeton's electrical
engineering department, but continued to consult at Bell Labs one day a week.
Thus the two continued to work together.
Alfred described their collaboration at Bell Labs:

\begin{quotation}
	He stayed at Bell Labs for a few years and went to Princeton University where he
	joined the faculty of the electrical engineering department, but he would come and spend one day a
	week consulting at Bell Labs. His consulting stint, he would come Fridays and sit in my office
	all day. The conversations that we'd have would range over all sorts of topics, and sometimes he'd
	mentioned that he was working on a problem with a colleague at Princeton, and after describing the
	problem, I might say, "You're kidding," and he said, "Oh, you're right. The solution is obvious,
	isn't it?" I don't know whether I would say dynamic programming or whatever, but several papers
	came out of this intense collaboration, and we got to the point where we could communicate with just
	a few words. We had a very large, shared symbol table.
	\cite{aho_oral_history_2022}
\end{quotation}

Ken Thompson joined the labs several months before the two of them to
start working on Multics, and had developed a plethora of tools based on
regular expressions, such as \texttt{grep}.
This combined with the \texttt{ED} and \texttt{QED} editors that came from MIT
and were shipped with Multics sparked Alfred's interest in regular expressions.

Aho is probably best known for the textbooks he co-authored with his friend Jeffrey,
colloquially known as \textit{The Dragon Book} (\citetitlecite{the_dragon_book_aho_ullman_sethi_1986})
and their book with John Hopcroft on algorithms.

\begin{quotation}
	\textbf{Collaboration with Ullman}
	Aho is best known for the textbooks he wrote with Ullman, his co-awardee.
	The two were full time colleagues for three years at Bell Labs, but after
	going back to Princeton as a faculty member Ullman continued to work one day a
	week for Bell. They retained an interest in the intersection of automata theory
	with formal language. In an early paper, Aho and Ullman showed how it was
	possible to make Knuth's LR(k) parsing algorithm work with simple grammars that
	technically did not meet the requirements of an LR(k) grammar. This technique
	was vital to the Unix software tools developed by Aho and his colleagues at Bell
	Labs. That was just one of many contributions Aho and Ullman made to formal
	language theory and to the invention of efficient algorithms for lexical
	analysis, syntax analysis, code generation, and code optimization. They
	developed efficient algorithms for data-flow analysis that exploited the
	structure of "gotoless" programs, which were at the time just becoming the norm.
	\cite{aho_turing_award_2020}
\end{quotation}

\subsection{Compiler-Compilers}


\todo{Yacc and Lex made with Aho's help. then everyone started making mini languages.AWK.
	"Kernighan and Cherry developed a little language for specifyingmathematics called EQN using these
	tools"}


\begin{quotation}
	People started using the Kernighan and Lorinda Cherry EQN tool to specify mathematics in their
	documents and in the research papers that they were writing. They would feed the EQN specification
	into the typesetting program roff
	\dots
	Knuth adopted the EQN language to include in the TeX typesetting system, and in LaTeX. It's
	basically Kernighan and Cherry's way of specifying mathematics. These software tools had a great
	deal of influence, and Kernighan and Cherry enjoyed the fruits of parsing theory and formal language
	theory in using the tools Lex and Yacc to create their EQN typesetting language. Knuth has this
	saying that the best theory is motivated by practice and the best practice by theory. I internalized
	that with my early experience in the Computing SciencesResearch Center because I found that the
	theory that we were developing in computer science could be applied to document preparation systems,
	programming languages, compilers, and so on. It was really avery productive environment. I taught
	courses on compiler design at local universities, and then when I went to Columbia, I would teach
	the course on programming languages and their translators
	\dots
	I might point out that the first Fortran compiler developed by IBM in the 1950s took 18 staff years
	to create. In my programming languages and compilers course, I organized the students into teams of
	four or five. Each team had to create their own programming language, and then write a translator
	for it, and in all the time that I taught the course for almost 25 years at Columbia to thousands of
	students, never did a team failed to deliver a working compiler in the 15-week course, and I
	attribute that to the abstractions
	\cite{aho_oral_history_2022}
\end{quotation}
\begin{quotation}
	Aho: Okay. AWK is a programming language that was created by me, Brian
	Kernighan, and Peter Weinberger.

	Hsu: And it's your three initials that are in.

	Aho: Yes. I'm the A in AWK. Weinberger is the W in AWK and Kernighan is the
	K in AWK.We thought that it was just a throwaway tool for us, nobody really
	would be interested in it. But it's amazing how much routine data processing
	there is in the world.The reason the language got to be known as AWK was because
	when our colleagues would see the three of us in one office or another, and when
	they'd walk past the open door, they'd say, AWK, AWK, AWK as they were going
	down the corridor. So we had no choice but to call it AWK because of the
	good-natured ribbing we got from our colleagues, and because at some Unix
	conference, they passed out t-shirts that had AWK,and the error message saying
	"bailing out on or near line five" on them.
\end{quotation}



\subsection{Brian Kernighan, Joined 1969}

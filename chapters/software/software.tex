
\section{The Software Crisis}
% \todo{DEC PDP-8 and PDP-11; IBM System/360 and OS/360; Multics; Unix; C}As the 1960s progressed,
% the notion of
% \textit{software} became more established,and the programs being written served the authors to a
% much greater extent.
% Prior to the 1960s, programs were often tailor-made for a specific
% machine. There was no hope of re-using the program on another machine. For most of the users, their
% organization had spent a considerable portion of their budget on their system, and they were
% expected to use it for a long time. Retargetable compilers did not exist.

Note that we will discuss some giants of computing history in this chapter who cast long
shadows over the field; however, we will primarily focus on those involved in the
development of compilers and programming languages.
In particular, we will not do justice to key figures from Bell Labs, such as
Brian Kernighan, Ken Thompson, Dennis Ritchie, Claude Shannon, and Doug McIlroy.
I can recommend \textit{The Idea Factory} for a more comprehensive account.

As Hopper and Backus had put it numerous times, programming was still exceedingly
painful at this point in time, and the relative cost of programming kept increasing.
As machines became cheaper and more powerful, the share of the cost of building and
running a program devoted to programming increased, and thus the importance of
ease of programming increased as well.

Michael Mahoney, doctor history and history of science, goes so far as to describe programming
and computer design in this way as early as 1945
\cite[The Structures of Computation]{the-first-computers-2002}:
\begin{quotation}
	The kinds of computers we have designed since 1945 and the kinds of programs we have written for
	them reflect not the nature of the computer but the purposes and aspirations of the groups of people
	who made those designs and wrote those programs, and the product of their work reflects not the
	history of the computer but the histories of those groups, even as the computer in many
	cases fundamentally redirected the course of those histories.
\end{quotation}

While this description was correct about the \textit{direction} compiler engineering
and programming language design were going, the statement was far too strong to make
about the year 1945.
Automatic programming was still scarcely more than synthetic machine code so the programmer
did not have to concern themselves with the details of the particular machine's instruction set
to the same extent, but they were still very much writing programs by writing specific instructions
instead of concerning themselves with the problem they were trying to solve and allowing the
compiler to turn that into proper machine code.

American historian of computing and aerospace Paul Ceruzzi describes the situation in 1968:

\begin{quotation}
	Despite great strides in software, programming always seemed to be in a state of crisis and
	always seemed to play catch-up to the advances in hardware. This crisis came to a head in 1968, just
	as the integrated circuit and disk storage were making their impact on hardware systems. That year,
	the crisis was explicitly acknowledged in the academic and trade literature and was the subject of a
	NATO-sponsored conference that called further attention to it. Some of the solutions proposed were a
	new discipline of software engineering, more formal techniques of structured programming, and new
	programming languages that would replace the venerable but obsolete COBOL and FORTRAN
	\footnote{Readers may find it entertaining that Ceruzzi describes COBOL and FORTRAN
		as obsolete \textit{by 1968}; both are still heavily used in some industries,
		and one of the very first compilers to adopt MLIR (to be discussed in a later chapter)
		is a Fortran compiler\cite{spickett_flang_levels_up_2025}.}. Although
	not made in response to this crisis, the decision by IBM to sell its software and services separately
	from its hardware probably did even more to address the problem. It led to a commercial software
	industry that needed to produce reliable software in order to survive. The crisis remained, however,
	and became a permanent aspect of computing. Software came of age in 1968; the following decades would
	see further changes and further adaptations to hardware advances.
	\cite{history_of_modern_computing_2003_ceruzzi}
\end{quotation}

Backus describes this as part of the motivation for beginning work on \ftn{}
\cite[\textit{The Economics of Programming}]{hopl_backus_history_of_fortran}:
\begin{quotation}
	Another factor that influenced the development of Fortran was the economics
	of programming in 1954. The cost of programmers associated with a computer
	center was usually at least as great as the cost of the computer itself\dots
	Thus, programming an debugging accounted for as much as three quarters of the
	cost of operating a computer; and obviously, as computers got cheaper, this
	situation would get worse.
\end{quotation}

Thus this chapter is about the efforts to make programming easier and more productive
to address this crisis. Software rose in importance in industry, it became a field of
study in academia, and true research on compilers and programming languages began.

\subsection{Peter Landin's ISWIM}

In \cref{sec:intro-lambda-calc} we established the \lamc and its early
applications to compilers in Lisp, and in \cref{sec:algol60} we discussed Landin's
use of \lam to formalize the semantics of ALGOL 60.
Landin's work continued to be highly influential, in particular his
\parencite{landin_next_700_prog_langs_1966}.

In this paper, he describes a programming language \textit{ISWIM}, standing for
\textit{If You See What I Mean}, which conveyed the semantics of \lam with
a particularly elegant syntax.
This language represented his vision for the future of programming languages
with an emphasis on expressing the programmer's intent uncluttered by
the details of the machine running the program.
\todo{was dynamically typed, much like lisp but with semantics closer to \lam.}
He argued that the programmer ought to only consider their intent, and the compiler ought to
consider the operations that would be needed to carry out their intent.

\parencite{hopl_history_of_ml_2020} described the origins of the Meta Language in Landin's paper
like so:

\begin{quotation}
	The basic framework of the language design was inspired by Peter Landin’s ISWIM language
	from the mid 1960s, which was, in turn, a sugared syntax for Church’s lambda calculus. The
	ISWIM framework provided higher-order functions, which were used to express proof tactics and
	their composition. To ISWIM were added a static type system that could guarantee that programs
	that purport to produce theorems in the object language actually produce logically valid theorems
	and an exception mechanism for working with proof tactics that could fail. ML followed ISWIM in
	using strict evaluation and allowing impure features, such as assignment and exceptions.
\end{quotation}

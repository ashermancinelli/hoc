\subsection{Peter Landin's ISWIM}

In \cref{sec:intro-lambda-calc} we established the \lamc and its early
applications to compilers in Lisp, and in \cref{sec:algol60} we discussed Landin's
use of \lam to formalize the semantics of ALGOL 60.
Landin's work continued to be highly influential, in particular his
\parencite{landin_next_700_prog_langs_1966}.

In this paper, he describes a general framework for programming languages
called \textit{ISWIM}, standing for
\textit{If You See What I Mean}
\footnote{ISWIM is sometimes pronounced \textit{eye-swim}.},
which conveyed the semantics of \lam with
a particularly elegant syntax over \lam constructs.
This language represented his vision for the future of programming languages
with an emphasis on expressing the programmer's intent uncluttered by
the details of the machine running the program.

\begin{quotation}
	Most programming languages are partly a way of
	expressing things in terms of other things and partly a
	basic set of given things. The ISWIM (If you See What I
	Mean) system is a byproduct of an attempt to disentangle
	these two aspects in some current languages.

	ISWIM is an attempt at a general purpose system for
	describing things in terms of other things, that can be
	problem-oriented by appropriate choice of "primitives."
	So it is not a language so much as a family of languages,
	of which each member is the result of choosing a set of
	primitives.
\end{quotation}

Landin described the grammar of ISWIM in informal English, which was perhaps
a step backwards from John Backus and Peter Naur's formal grammars (see \parencite{naur_backus_algol_1960})
which were also being developed for the specification of ALGOL.

\subsection{Landin and Strachey}

\parencite{landin_my_years_w_strachey_2000}.
\parencite{scott_strachey_math_semantics_for_computer_languages_1971}.

\parencite{hopl_history_of_ml_2020} notes how most of the characters in this story
were in part brought together by a shared interest and an interested amateur
who noticed them all reading similar materials in the library.
In this way, many of the early meetings formalizing the language were
informal and poorly recorded.

\begin{quotation}
	It is interesting to note that most of the central personalities first met through an unofficial reading group formed by an
	enthusiastic amateur named Mervin Pragnell, who recruited people he found reading about topics like logic at bookstores or
	libraries. The group included Strachey, Landin, Rod Burstall, and Milner, and they would read about topics like combinatory
	logic, category theory, and Markov algorithms at a borrowed seminar room at Birkbeck College, London. All were self-taught
	amateurs, although Burstall would later get a PhD in operations research at Birmingham University. Rod Burstall was
	introduced to the lambda calculus by Landin and would work for Strachey briefly before moving to Edinburgh in 1965.
	Milner had a position at Swansea University before spending time at Stanford and taking a position in Edinburgh in 1973.
\end{quotation}

\todo{was dynamically typed, much like lisp but with semantics closer to \lam.}
He argued that the programmer ought to only consider their intent, and the compiler ought to
consider the operations that would be needed to carry out their intent.

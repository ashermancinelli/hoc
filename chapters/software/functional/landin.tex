\subsection{Peter Landin}

\subsubsection{Landin and ALGOL}

\parencite{landin_algol_lambda_1965}

\subsubsection{The Next 700 Programming Languages}

Published in \citeyear{landin_next_700_prog_langs_1966},
Peter Landin's seminal paper \citep{landin_next_700_prog_langs_1966} lays out his vision for
the future of programming language design, emphasizing the importance of the programmer's intent
uncluttered by details of the hardware.
This language, IYSWIM, or \textit{If You See What I Mean}, was semantically
equivalent to the \lambdacalc but introduced programming language design features
that were novel and innovative above that of the \lambdacalc.
He argued that the programmer ought to only consider their intent, and the compiler ought to
consider the operations that would be needed to carry out their intent.

\parencite{hopl_history_of_ml_2020} described the origins of the Meta Language in Landin's paper
like so:

\begin{quotation}
	The basic framework of the language design was inspired by Peter Landin’s ISWIM language
	from the mid 1960s [1966b], which was, in turn, a sugared syntax for Church’s lambda calculus. The
	ISWIM framework provided higher-order functions, which were used to express proof tactics and
	their composition. To ISWIM were added a static type system that could guarantee that programs
	that purport to produce theorems in the object language actually produce logically valid theorems
	and an exception mechanism for working with proof tactics that could fail. ML followed ISWIM in
	using strict evaluation and allowing impure features, such as assignment and exceptions.
\end{quotation}

\section{GNU}

There were loads of small companies that made money leasing out their compilers,
and GNU put them all out of business essentially overnight.
Every hardware company wrote their own compilers, but once it became clear
that just adding a backend for the GNU compiler would be significantly easier
than writing a new compiler from scratch, many of the hundreds of small compiler
companies went out of business, and the compiler teams inside large companies
began using and contributing to the GNU compilers.

In \cref{chap:software}, we discussed how the C programming language permeated
the software industry via Unix.
Hardware vendors would then develop their own C compilers so their users
could use the programming language they were familiar with, similar to how
hardware vendors were expected to provide Fortran compilers.
This led to a diverse ecosystem of C compilers from different vendors targeting
their respective hardware with an unfortunately diverse set of supported language features.
This meant users could not rely on a consistent experience with C compilers when
developing software for different hardware platforms.

The \textit{automake} and \textit{autoconf} tools were developed to address this
issue with macros, but buy-and-large, this experience was unpleasant for users.

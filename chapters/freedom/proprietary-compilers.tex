
\section{Proprietary Compilers}

While GCC decimated the market for small proprietary compilers
in the early 1990s, many proprietary compilers persisted in other
markets, especially from larger hardware vendors.
There were many proprietary \acrfull{hpc} compilers in particular,
because the \acrshort{hpc} market is relatively large by market capitalization,
but small in the number of customers, and the applications are often niche.
Government laboratories and universities are the most common customers,
and they are usually willing to pay for compilers and support if it means
their applications will better utilize the hardware, which tends to be
very, very expensive.

\subsection{Cray}
\subsection{IBM XL Compilers}
\subsection{The Portland Group}

\url{https://docs.nvidia.com/hpc-sdk/pgi-compilers/legacy.html}

\citeauthor{von_hagen_definitive_guide_gcc_2011} summarizes 
\begin{quotation}
    Some third-party vendors exist that provide stand-alone compiler suites. One such vendor is The Portland
    Group (\url{http://www.pgroup.com/}), which markets a set of high-performance, parallelizing compiler
    suites supporting Fortran, C, and C++
\end{quotation}

\subsection{Absoft}
\subsection{Borland}
\subsection{Intel}
\subsection{Microsoft Visual Compilers}

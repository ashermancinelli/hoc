
\section{Short History of Open Source}

The most significant feature of this time period is the collaboration brought about by
the advent of the free software movement.
In the beginning of software development, much of it \textit{was already} given
away to customers of hardware companies like IBM.
When the software was already in machine code, there was not
much of an alternative to giving your source code away to your customers.
You had a deck of punchcards with your software, and to get that to your customers,
you had to punch out a new copy and mail it to them; other than the accompanying documentation,
you \textit{had} to ship the code for your programs, and there was nothing in between your
programs and the runnable program.

Hardware companies would give the software away for free with the hardware because
the cost of the software was negligible relative to the cost of the hardware.
The turning point in this model of software distribution came in 1969 when IBM
broke some of their software off to be sold separately.
Prior to this, perhaps the first instance of free software was in Grace Hopper's development of
the A-2 compiler (see Section \ref{sec:a1-a2-compilers}), for which the Remington Rand company
accepted changes and suggestions their customers sent them.
When software demanded a market of its own, businesses were able to support themselves
solely by selling software, and it became the core business value of the company.
This worked against the free distribution of software to a large extent.

Shortly after Unix was developed at Bell Labs in the early 1970s, its source code
was distributed under a license that would today be considered similar to \GLS{foss}.
While the source code of Unix was only given to universities for academic use and
only users covered by the license were technically allowed to see the source,
large communities began to form around it, namely USENIX.
Some entirely new distributions and rewrites of Unix formed, most notably the
Berkeley Software Distribution (BSD), which was developed at the University of California,
Berkeley.

\todo{gnu linux c llvm python; Facebook php->c++
	compiler;}

\todo{lattner tried to get llvm in the gnu project; new licensing,
	permissive licensing.}

\begin{quotation}
	The discussion of the GNU/Linux
	operating system and the "open source" software movement, discussed last,
	likewise has deep roots. Chapter 3 discussed the founding of SHARE, as well as
	the controversy over who was allowed to use and modify the TRAC programming
	language. GNU/Linux is a variant of UNIX, a system developed in the late 1960s
	and discussed at length in several earlier chapters of this book. UNIX was an
	open system almost from the start, although not quite in the ways that "open" is
	defined now. As with the antitrust trial against Microsoft, the open source
	software movement has a strong tie to the beginnings of the personal computer's
	invention. Early actions by Microsoft and its founders played an important role
	here as well. We begin with the
	antitrust trial.\cite{history_of_modern_computing_2003_ceruzzi}
\end{quotation}

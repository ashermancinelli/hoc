
\begin{figure}[h]
\begin{luacode}
local start = 1942
local _end = 1957
timeline.draw_timeline({ 
    start_year = start,
    end_year = _end,
    marker_interval = 1,
    show_year = false,
    line_always = true,
    events = {
        {1942, "Zuse conceptualizes first high-level language \\textit{Plankalkül}"},
        {1944, "Automatic Sequence Control Calculator started working"},
        {1949.3, "Hopper's contract at Harvard Computation Lab ends; she joins EMCC the same year",delta=-.2},
        {1949.5, "Stiefel at the ETH learns about Z4"},
        {1950, "Rutishauser ran Zuse's Z4 as a compiler at ETH"},
        {1951, "B{\"o}hm develops practical compiler for PhD thesis at ETH",delta=-.2},
        {1951.5, "Hopper begins work on the A-0 compiler for the UNIVAC I",delta=-.2},
        {1952, "Hopper presents \\textit{Automatic Programming} at ACM",delta=-.2},
        -- {1952.5, "Alick Glennie develops first compiler in the modern sense for the U of Manchester Mark I"},
        {1952.5, "Alick Glennie develops compiler for Mark I",delta=-.3},
        {1953, "Hopper's team completes the A-1",delta=-.4},
        {1953.3, "Hopper's team completes the A-2 (featuring \\textit{psuedocode}, closer to modern notion of \\textit{compiler})",delta=-.2},
        {1953.6, "Hopper gives presentation to UNIVAC users on the A-2"},
        {1953.9, "Numerous US government agencies have adopted the A-2"},
        {1954.5, "John Backous at IBM begins work on \\FTN{}"},
        {1954.9, "Nora Moser of US Army sends Hopper modifications to the A-2"},
        {1957, "John Backous at IBM released the first \\FTN{} compiler, the first comerical compiler"},
    },
})
tex.sprint(string.format("\\caption{Early compiler history, %d--%d}", start, _end))
\end{luacode}
\label{fig:dawn-timeline}
\end{figure}

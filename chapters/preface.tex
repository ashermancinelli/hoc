\chapterstar{Preface}

The history of compilers is rich and deeply connected to the broader history of computing,
however, I believe that no comprehensive work has tied together the threads of this history
with a specific focus on compilers.
There are wonderful tellings of the very first compilers on Konrad Zuse's Z4 computer at the ETH Zurich
and Grace Hopper's pioneering work on the A-series compilers for the UNIVAC I,
and John Backus' work on the first commercial compiler for \FTN{} at IBM,
but these are often isolated stories.
When they are woven together, they are often not connected to the
\textit{next} developments in compiler technology at Bell Labs:
Aho and Ullman's \textit{Principles of Compiler Design}, the development of Lex and Yacc,
the C programming language, Bjarne Stroustrup's first C++ compiler \texttt{cfront}
(inspired by Alan Kay's vision of object-oriented programming).
The subsequent decades of open-source compiler development, advances in optimization techniques,
and the explosion of new programming languages and compilation paradigms
(e.g. just-in-timem compilation) are then followed by the
rise and necessity of hardware-software codesign and domain-specific languages
seen most evidently in projects based on MLIR and LLVM.
The threads between these points in history offer a deeper understanding of
each individual piece and context that motivates modern compiler development.

% There have been several monumentous works on the history of computing and a few on
% the history of programming languages, but in the two and a half decades since the turn
% of the century there have been ample developments in compiler technology that
% no comprehensive work has covered thus far.
% This book aims to fill that gap to a small degree; it is not exhaustive, but
% contextualizes many of the most important recent developments in the larger
% narrative of compiler history.

% The introduction chapter contains a brief version of the entire book; the reader is encouraged to read it first.
This work intends to weave these threads together.
My hope is that each chapter stands on its own, but that reading them in order will give a more complete picture.
The reader ought to be able to read a particular chapter that suits their needs at the time.
This book does not assume the reader is deeply familiar with compiler engineering or computer science.
The book is structured as a chronological narrative of the history of compilers,
intending to keep the focus on compiler technologies and the people behind them
without focusing on any particular company or product.

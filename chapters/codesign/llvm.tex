\section{LLVM}
\label{sec:llvm-in-detail}

LLVM was introduced in \cref{sec:lattner-llvm} as part of Chris Lattner's
history, but LLVM is a large and diverse project
and it deserves discussion outside of the context of Chris Lattner.


\begin{table}[h!]
	\centering
	\begin{tabularx}{\linewidth}{llX}
		\toprule
		Subproject      & Date Added       & Notes                                                               \\
		\midrule
		LLVM Core       & 2000             & Project inception at UIUC, see \Cref{sec:lattner-llvm}.             \\
		Clang           & 2006             & Chris starts work on Clang, see \Cref{sec:llvm-at-apple-clang}.     \\
		compiler-rt     & $\sim$2009       & Compiler runtime libraries, drop-in replacement for \texttt{libgcc} \\
		Clang           & $\sim$2009       & Clang supports C and Objective-C.                                   \\
		LLDB            & $\sim$2010       & LLDB shipped as part of Xcode 4.                                    \\
		Clang++, libc++ & $\sim$2010       & Clang now supports C++; C++ standard library added as LLVM project. \\
		lld             & $\sim$2011--2015 & LLVM linker                                                         \\
		libclc          & $\sim$2012       & OpenCL standard library                                             \\
		OpenMP runtime  & $\sim$2013--2014 & Added as LLVM's OpenMP runtime                                      \\
		MLIR            & 2019             & \todo{see section on MLIR}                                          \\
		Flang           & 2020             & \todo{see section on Flang}                                         \\
		llvm-libc       & 2020s            &                                                                     \\
		\bottomrule
	\end{tabularx}
	\label{tab:lattner-llvm-project-timeline}
	\caption{Approximate dates major LLVM subprojects joined the umbrella project.}
\end{table}


\subsection{Architecture of the Project}

LLVM is an \textit{umbrella project} containing many \textit{subprojects}
that can be used independently.

The first (and maybe only) interaction most non-compiler-engineers have with LLVM
is through the frontends, like Clang for C, C++, Objective-C, or Swift.
This betrays the complexity and elegance with which the pieces of LLVM fit together.
These frontends are typically wrappers around the LLVM command-line parsing library,
a library to perform the lexing, parsing, semantic analysis and whatever else the frontend
for that specific language is expected to do (like module dependency analysis in Swift, for example),
and then various other LLVM subprojects.

Most of the time, these other subprojects include pieces from the LLVM subproject,
which does optimization and code generation (among other things).

\todo{refer to \parencite[11.4.2. LLVM is a Collection of Libraries]{brown_wilson_lattner_aosa_vol1_2011}}

\subsection{The IR}

Lattner et al refer to LLVM's \acrshort{ir} as \textit{C with vectors} in \parencite{lattner_amini_mlir_og_paper_2021}.
Type annotations are needed in more places than with C and many constructs are lower-level than C
(like control flow), but the approximations is accurate.
The IR was designed to preserve high-level information from diverse programming languages,
but it is apparent that it is designed especially for C and C++.

There are no physical registers in LLVM IR. It is an \acrfull{ssa} ir,
meaning there are infinite "registers" and each can be assigned to only once.
Typically, \acrshort{ssa} IRs use a \textit{phi} or \textphi instruction to merge
values from different paths, like the value \texttt{x} in the C expression \texttt{if (cond) x = 1; else x = 2;}.
The \acrfull{cfg} is also explicit; functions are made up of \gls{basicblock}s,
basic blocks are made up of instructions and terminated by branches or terminators,
and functions end in exactly one terminator (like a \texttt{return}):

\begin{minted}[linenos,frame=single]{llvm}
; Perhaps not the most efficient way to add two numbers.
; unsigned add2(unsigned a, unsigned b) {
;   if (a == 0) return b;
;   return add2(a-1, b+1);
; }

define i32 @add2(i32 %a, i32 %b) {
entry:
  %tmp1 = icmp eq i32 %a, 0
  br i1 %tmp1, label %done, label %recurse

recurse:
  %tmp2 = sub i32 %a, 1
  %tmp3 = add i32 %b, 1
  %tmp4 = call i32 @add2(i32 %tmp2, i32 %tmp3)
  ret i32 %tmp4

done:
  ret i32 %b
}
\end{minted}

There are few instructions and some of them are overloaded, making the IR
look very similar to \acrshort{risc} assembly languages.
The IR as it was when Lattner published his PhD thesis had only 31 opcodes
\parencite{lattner_phd_thesis_pointer_intensive_programs_2005}.

Structs are represented in an unsurprising way:
% \begin{lstlisting}[language=llvm,frame=single]
\begin{minted}[linenos,frame=single]{llvm}
; struct RT {char A; int B[10][20]; char C;};
; struct ST {int X; double Y; struct RT Z;};
%struct.ST = type { i32, double, %struct.RT }
%struct.RT = type { i8, [10 x [20 x i32]], i8 }
\end{minted}

Memory operations are represented by the \texttt{load} and \texttt{store}
instructions.
The access \texttt{s[1].Z.B[5][13];} in the snippet above looks roughly like
this in modern LLVM IR:

% https://godbolt.org/z/oozebnznf
% \begin{lstlisting}[frame=single]
\begin{minted}[linenos,frame=single]{llvm}
%0 = ; pointer to value of type ST
%1 = getelementptr i8, ptr %0, i64 1296
%2 = load i32, ptr %2
\end{minted}

LLVM IR was initially more strongly typed than it is today:

\begin{quotation}
	One of the fundamental design features of LLVM is the inclusion of a language-independent type
	system. Every SSA register and explicit memory object has an associated type, and all operations
	obey strict type rules.
	\parencite{lattner_phd_thesis_pointer_intensive_programs_2005}
\end{quotation}

This strictness has been eased over time.
The \texttt{getelementptr} instruction was used to compute address offsets on
base pointers, and explicit types were required. Pointer types were specified with
\texttt{i8*}, just as one would write in C. Now, with the shift to
\textit{opaque pointers} (or \textit{untyped pointers}), there is a simple \texttt{ptr} type
that does not specify the pointee type, and \texttt{ptradd} is used instead of \texttt{getelementptr}
with no required type annotation.

\subsection{The Interface}

As already discussed, a huge part of LLVM's value proposition is its ability
to be consumed by other applications by virtue of its modular design.
Another project can link against LLVM's libraries and use its interfaces
to procedurally build IR, or they can just write LLVM IR to a file and
pass it to the LLVM tools.
This means LLVM's interfaces for constructing IR are more important than in other
compiler projects; other developers \textit{outside of LLVM} depend on those interfaces.
The users of LLVM's interfaces far outnumber the developers contributing to LLVM
directly on a regular basis.

\inputminted[linenos,frame=single]{cpp}{chapters/codesign/llvm-example.cpp}

The above snippet of C++ constructs the function in LLVM IR below:

\begin{minted}[linenos,frame=single]{llvm}
define i32 @add_ints(i32 %0, i32 %1) {
entry:
  %2 = add i32 %0, %1
  ret i32 %2
}
\end{minted}

The modular interfaces for working with IR do not stop with the \textit{construction}
of IR;
there are also rich interfaces for constructing compiler passes and pattern-matching.
The example instruction-simplification pass below from \parencite{brown_wilson_lattner_aosa_vol1_2011}
demonstrates how the pattern-matching interfaces might be used in a \gls{peephole} optimization:

\begin{minted}[linenos,frame=single]{cpp}
// X - 0 -> X
if (match(Op1, m_Zero()))
  return Op0;
// X - X -> 0
if (Op0 == Op1)
  return Constant::getNullValue(Op0->getType());
// (X*2) - X -> X
if (match(Op0, m_Mul(m_Specific(Op1), m_ConstantInt<2>())))
  return Op1;
// $\dots$
return nullptr;  // Nothing matched, return null to indicate no transformation.
\end{minted}

This function might be called in a simplification pass that traverses
all the instructions in a \gls{basicblock}, and if a simpler version
of the instruction is found, the simpler version replaces the original instruction like so:

\begin{minted}[linenos,frame=single]{cpp}
for (auto &I : BB)
  if (auto *NewValue = simplifyInstruction(I))
    I.replaceAllUsesWith(NewValue);
\end{minted}

\subsubsection{Aside: Influence of ML on LLVM}

As a brief (and opinionated) aside: in \parencite{brown_wilson_lattner_aosa_vol1_2011},
Lattner makes this remark about the pattern matching features in LLVM:

\begin{quotation}
	LLVM is implemented in C++, which isn't well known for its
	pattern matching capabilities (compared to functional languages like Objective Caml),
	but it does offer a very general template system that allows us to
	implement something similar. The match function and the \texttt{m\_} functions allow
	us to perform declarative pattern matching operations on LLVM IR code.
	For example, the \texttt{m\_Specific} predicate only matches if the left hand
	side of the multiplication is the same as \texttt{Op1}.
\end{quotation}

And in this interview
\parencite{lattner_minsky_why_ml_needs_new_programming_language_2025},
Chris elaborates further:

\begin{quotation}
	\textbf{Chris}
	And so Clang has some really cool stuff that allowed it to scale and
	things like that, but I was also burned out. We had just shipped it. It was
	amazing. I’m like, there has to be something better. And so, Swift really came
	starting in 2010. It was a nights and weekends project.
	Turns out, programming languages are a mature space. It’s not like you
	need to invent pattern matching at this point. It’s embarrassing that C++
	doesn’t have good pattern matching.

	\textbf{Ron}
	We should just pause for a second, because I think this is like a small
	but really essential thing. I think the single best feature coming out of
	language like ML in the mid-seventies\dots
	having this pattern matching facility that lets you basically in a
	reliable way do the case analysis so you can break down what the possibilities
	are—is just incredibly useful. And very few mainstream languages have picked it
	up. I mean Swift again is an example, but languages like ML, SML, and Haskell,
	and OCaml.

	\textbf{Chris}
	I mean pattern matching, it is not an exotic feature. Here we’re
	talking about 2010.
	And so pattern matching, when I learned OCaml, it’s so beautiful. It
	makes it so easy and expressive to build very simple things.
\end{quotation}

It appears to me that \textit{every} compiler is written in ML to some degree--for
whatever reason, the language features that Landin envisioned in \parencite{landin_next_700_prog_langs_1966}
and were expanded upon in LCF/ML \todo{needs citation, refer to the ML section}
happen to be particularly useful
when designing compilers.

This quote is often attributed to Tony Hoare:
"I don't know what the language of the year 2000
will look like, but I know it will be called Fortran."
Well, I don't know what the language of choice for writing compilers
will be in the year 2100, but I know it will look like ML.

\subsection{Tablegen}

Perhaps the part of LLVM that is least-known among \textit{users} of LLVM is
its \textit{Tablegen} system. Tablegen is a program for declarative metaprogramming
inside LLVM.

\parencite[11.6.2. Unit Testing the Optimizer]{brown_wilson_lattner_aosa_vol1_2011}.

\subsection{Testing}

\todo{lit, filecheck, generate-test-case.py, other tools.}

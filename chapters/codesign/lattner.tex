\section{Chris Lattner}

Chris Lattner's impact on the landscape of compiler technology in the 21\textsuperscript{st}
century can't be overstated. His work on LLVM has more or less beat out every other
compiler technology.
We will now discuss the stories of the technologies that Chris developed contiguously,
with soliloquies for uses of those technologies that Chris was not involved in.

\subsection{Low-Level Virtual Machine}
\label{sec:lattner-llvm}

In December of 2000, Chris began work on LLVM with his advisor Vikram Adve
as part of his PhD research at the University of Illinois at Urbana-Champaign.
LLVM stood for Low-Level Virtual Machine
at the time but is no longer an acronym and is simply the name of the project.
At the time of the publication of his thesis \parencite{lattner_phd_thesis_pointer_intensive_programs_2005},
the umbrella project initially contained only an \acrlong{ir}, an optimizer for that \acrshort{ir},
\acrshort{ir}-level linking, and both \gls{offline-compilation} and \gls{online-compilation} for code
generation.
Today, these components are all part of the LLVM \textit{subproject} within the
LLVM \textit{umbrella project}, which contains other compiler libraries and tools.

\begin{quotation}
	This chapter describes LLVM — Low-Level Virtual Machine — a compiler framework that
	aims to make lifelong program analysis and transformation available for arbitrary software, and
	in a manner that is transparent to programmers. LLVM achieves this through two parts: (a) a
	code representation with several novel features that serves as a common representation for analysis,
	transformation, and code distribution; and (b) a compiler design that exploits this representation
	to provide a combination of capabilities that is not available in any previous compilation approach
	we know of. \parencite{lattner_phd_thesis_pointer_intensive_programs_2005}
\end{quotation}

While LLVM is perhaps best-known for some of the specific technologies contained
in the umbrella project, its most novel features lie in the compiler architecture.
The \acrshort{ir} was more flexible and language-agnostic than the other contemporary
\acrshort{ir}s, but the nature of LLVM as a set of \textit{libraries} that can
roughly be used independently of each other.
The tools developers know LLVM for (like Clang, LLD, and LLDB) are really thin main
programs that simply call into the libraries for parsing C code, optimizing a chunk of
LLVM IR, or generating machine code for that LLVM IR.
No prior art provided this level of flexibility, and LLVM's IR,
terminology, and interfaces have become the lingua-franca of the compiler world.

\begin{quotation}
	While LLVM provides some unique capabilities, and is known for
	some of its great tools (e.g., the Clang compiler, a C/C++/Objective-C compiler
	which provides a number of benefits over the GCC compiler), the main thing that
	sets LLVM apart from other compilers is its internal
	architecture.
	\parencite[Section 11. LLVM]{brown_wilson_lattner_aosa_vol1_2011}
\end{quotation}



\begin{table}[h!]
	\centering
	\begin{tabularx}{\linewidth}{llX}
		\toprule
		Subproject      & Date Added       & Notes                                                               \\
		\midrule
		LLVM Core       & 2000             & Project inception at UIUC, see \Cref{sec:lattner-llvm}.             \\
		Clang           & 2006             & Chris starts work on Clang, see \Cref{sec:llvm-at-apple-clang}.     \\
		compiler-rt     & $\sim$2009       & Compiler runtime libraries, drop-in replacement for \texttt{libgcc} \\
		Clang           & $\sim$2009       & Clang supports C and Objective-C.                                   \\
		LLDB            & $\sim$2010       & LLDB shipped as part of Xcode 4.                                    \\
		Clang++, libc++ & $\sim$2010       & Clang now supports C++; C++ standard library added as LLVM project. \\
		lld             & $\sim$2011--2015 & LLVM linker                                                         \\
		libclc          & $\sim$2012       & OpenCL standard library                                             \\
		OpenMP runtime  & $\sim$2013--2014 & Added as LLVM's OpenMP runtime                                      \\
		MLIR            & 2019             & \todo{see section on MLIR}                                          \\
		Flang           & 2020             & \todo{see section on Flang}                                         \\
		llvm-libc       & 2020s            &                                                                     \\
		\bottomrule
	\end{tabularx}
	\label{tab:lattner-llvm-project-timeline}
	\caption{Approximate dates major LLVM subprojects joined the umbrella project.}
\end{table}


\subsection{Architecture of the IR, Optimizer, and Code Generator}


\todo{examples of the IR and the C++ interfaces for building it, and how
	this is usable from other programming languages like OCaml.}

The first (and maybe only) interaction most non-compiler-engineers have with LLVM
is through the frontends, like Clang for C, C++, Objective-C, or Swift.
This betrays the complexity and elegance with which the pieces of LLVM fit together.
These frontends are typically wrappers around the LLVM command-line parsing library,
a library to \textit{do} the lexing, parsing, semantic analysis and whatever else the frontend
for that specific language is expected to do (like module dependency analysis in Swift, for example),
and then various other LLVM subprojects.

Most of the time, these other subprojects include the pieces from the LLVM subproject,
which does optimization and code generation.

\subsection{Chris and LLVM at Apple}
\label{sec:llvm-at-apple}

\subsubsection{Clang}
\label{sec:llvm-at-apple-clang}

\subsubsection{Swift}

\parencite{lattner_minsky_why_ml_needs_new_programming_language_2025}:
\begin{quotation}
	\textbf{Chris}
	And so Clang has some really cool stuff that allowed it to scale and
	things like that, but I was also burned out. We had just shipped it. It was
	amazing. I’m like, there has to be something better. And so, Swift really came
	starting in 2010. It was a nights and weekends project.
	Turns out, programming languages are a mature space. It’s not like you
	need to invent pattern matching at this point. It’s embarrassing that C++
	doesn’t have good pattern matching.

	\textbf{Ron}
	We should just pause for a second, because I think this is like a small
	but really essential thing. I think the single best feature coming out of
	language like ML in the mid-seventies\dots
	having this pattern matching facility that lets you basically in a
	reliable way do the case analysis so you can break down what the possibilities
	are—is just incredibly useful. And very few mainstream languages have picked it
	up. I mean Swift again is an example, but languages like ML, SML, and Haskell,
	and OCaml.

	\textbf{Chris}
	I mean pattern matching, it is not an exotic feature. Here we’re
	talking about 2010.
	And so pattern matching, when I learned OCaml, it’s so beautiful. It
	makes it so easy and expressive to build very simple things.
\end{quotation}

\begin{table}[h!]
	\centering
	\begin{tabularx}{\linewidth}{l l X}
		\toprule
		Date       & Organization & Notes                                                              \\
		\midrule
		2022--     & Modular AI   & CEO/co-founder, Mojo programming language and compiler.            \\
		2020--2022 & SiFive       & Started LLVM CIRCT project applying MLIR to chip design.           \\
		2017--2020 & Google       & Started the MLIR project, worked on TPU compilers including XLA.   \\
		2017       & Tesla        &                                                                    \\
		2005--2017 & Apple        & Major expansion of the LLVM project. See \Cref{sec:llvm-at-apple}. \\
		2000--2005 & UIUC         & Designed foundations of LLVM.                                      \\
		\bottomrule
	\end{tabularx}
	\label{tab:lattner-career-timeline}
	\caption{Timeline of Chris Lattner's career \parencite{lattner_resume_work_history_2025}}
\end{table}


\subsection{Tablegen}

Perhaps the part of LLVM that is least-known among \textit{users} of LLVM is
its \textit{Tablegen} system. Tablegen is a program for declarative metaprogramming
inside LLVM.

\subsection{Chris and MLIR at Google}

\subsection{Mojo}

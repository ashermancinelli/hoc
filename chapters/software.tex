\chapter{Software, 1960-1970}

As the 1960s progressed, the notion of \textit{software} became more established,
and the programs being written served the authors to a much greater extent.
Prior to the 1960s, programs were often tailor-made for a specific machine.
There was no hope of re-using the program on another machine.
For most of the users, their organization had spent a considerable portion
of their budget on their system, and they were expected to use it for a long time.
Retargetable compilers did not exist.

Michael Mahoney made a strong statement to this end rather early on \cite[The Structures of Computation]{the_first_computers_2002}:
\begin{quotation}
The kinds of computers we have
designed since 1945 and the kinds of programs we have written for them reflect not the nature of the
computer but the purposes and aspirations of the groups of people who made those designs and wrote those
programs, and the product of their work reflects not the history of the computer but the histories of those
groups, even as the computer in many cases fundamentally redirected the course of those histories.
\end{quotation}

1945 was too early for this strong of a statement; how can one argue that software
reflected the authors 

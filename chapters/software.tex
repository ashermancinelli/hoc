
\chapter{Software, 1960-1980ish}
\todo{shit got crazy; became important, needs more focus/attention.}
\section{The Software Crisis}
\begin{quotation}
    Despite great strides in software, programming always seemed to be in a state of crisis and 
always seemed to play catch-up to the advances in hardware. This crisis came to a head in 1968, just 
as the integrated circuit and disk storage were making their impact on hardware systems. That year, 
the crisis was explicitly acknowledged in the academic and trade literature and was the subject of a 
NATO-sponsored conference that called further attention to it.Some of the solutions proposed were a 
new discipline of software engineering,more formal techniques of structured programming, and new 
programming languages that would replace the venerable but obsolete COBOL and FORTRAN. Although 
not made in response to this crisis, the decision by IBM to sell its software and services separately 
from its hardware probably did even more to address the problem. It led to a commercial software 
industry that needed to produce reliable software in order to survive. The crisis remained, however, 
and became a permanent aspect of computing. Software came of age in 1968; the following decades would 
see further changes and further adaptations to hardware advances.
\cite{history_of_modern_computing_2003_ceruzzi}
\end{quotation}
\todo{DEC PDP-8 and PDP-11; IBM System/360 and OS/360; Multics; Unix; C}As the 1960s progressed, 
the notion of 
\textit{software} became more established,and the programs being written served the authors to a 
much greater extent.Prior to the 1960s, programs were often tailor-made for a specific 
machine.There was no hope of re-using the program on another machine.For most of the users, their 
organization had spent a considerable portion of their budget on their system, and they were 
expected to use it for a long time. Retargetable compilers did not exist.Michael Mahoney made a 
strong statement to this end rather early on 
\cite[The Structures of Computation]{the-first-computers-2002}:
\begin{quotation}
    The kinds of computers we have designed since 1945 and the kinds of programs we have written for 
them reflect not the nature of the computer but the purposes and aspirations of the groups of people 
who made those designs and wrote those programs, and the product of their work reflects not the 
history of the computer but the histories of those groups, even as the computer in many 
cases fundamentally redirected the course of those histories.
\end{quotation}
\todo{1945 was too early for this strong of a statement;how can one argue that software reflected 
the authors when it was so dependent on the hardware?
They were still very dependent on the specific machine they were working on and
could not solely focus on their own problems.}
\section{Seymore Cray}
\begin{quotation}
    The CDC 160 and the Origins of the MinicomputerThe Whirlwind (a computer prototype built at 
MIT) had a word length of only 16 bits, but the story of commercial minicomputers really begins with 
an inventor associated with very large computers: Seymour Cray. While at UNIVAC Cray worked on the 
Navy Tactical Data System (NTDS), a computer designed for navy ships and one of the first 
transistorized machines produced in quantity. Around 1960 Control Data, the company founded in 1957 
that Cray joined, introduced its model 1604, a large computer intended for scientific customers. 
Shortly thereafter CDC introduced the 160, designed
\cite{nothing_new_since_von_neumann_2000}
\end{quotation}
\section{The DEC VAX and the IBM System/360}
\begin{quotation}
    Through the 1980s the dominant mainframe architecture continues to be a descendent of the IBM 
System/360, while the dominant mini was the DEC VAX, which evolved as a 32 bit extension of the 
16-bit PDP-11.
\cite{nothing_new_since_von_neumann_2000}
\end{quotation}
\section{Aho Before Bell Labs}
\section{Aho, Ullman, and Bell Labs}
\todo{Software (and compilers!) starts to become a real discipline!Ullman was older and further 
along than Aho, and Hopcraft came to Princeton and became Aho's advisor.}
\begin{quotation}
    One of the first people that I met at Princeton was a Columbia graduate by thename of Jeffrey 
Ullman. He had just gotten his undergraduate degree fromColumbia University and also had come to 
study digital systems in the EEdepartment at Princeton. So, he and I became close friends. When we 
graduatedfrom Princeton, we both joined the newly formed Computing Sciences ResearchCenter at Bell 
Labs. There we developed a lifelong collaboration on subjectsranging from algorithms, programming 
languages, to the very foundations ofcomputer science. I was very fortunate to have met some of the 
greatest peoplein the field and to have gotten to know them and work with them. You learn somuch by 
working with the best people in the field. So, I felt very blessed because I had this kind of 
background
\dots
Hsu: Before we jump into Bell Labs more deeply, could you maybe explain-- talk about your PhD 
thesis,but try to explain it to somebody who, maybe like a museum goer who doesn't really know much 
about computer science and linguistics.

Aho: This is interesting. As I mentioned, Hopcroft told me, "Find your own research problem." He 
did teach a course in automata and language theory, so I got introduced to formal language theory 
and automata theory, at least, as it was known at that time. I was interested in programming 
languages and compilers. What I noticed was that a programming language has a syntax and a 
semantics. All languages have a syntax and a semantics. If you want to write a translator for a 
programming language, or even a natural language, you have to understand the syntax and semantics of 
your source language and the target language
\dots
Hansen: 1967, and you followed Ullman there. He had already joined Bell Labs before.

Aho: A few months before me.

Hansen: A few months before. And what group was it that you joined?

Aho: I was interviewed by a department head by the name of Doug McIlroy. He was an 
applied mathematician from MIT. He had been at Bell Labs for a few years before me. Amongst other 
things, he had co invented macros for programming languages and he's also in this class of one of the 
smartest people I've ever met.Jeff wanted to go to academia a little bit earlier than I did, like 
many years earlier. He stayed at Bell Labs for a few years and went to PrincetonUniversity where he 
joined the faculty of the electrical engineering department, but he would come and spend one day a 
week consulting at Bell Labs.His consulting stint was he would come Fridays and sit in my office 
all day.The conversations that we'd have would range over all sorts of topics, and sometimes he'd 
mentioned that he was working on a problem with a colleague atPrinceton, and after describing the 
problem, I might say, "You're kidding," and he said, "Oh, you're right. The solution is obvious, 
isn't it?" I don't know whether I would say dynamic programming or whatever, but several papers 
came out of this intense collaboration, and we got to the point where we could communicate with just 
a few words. We had a very large, shared symbol table.
\cite{aho_oral_history_2022}
\end{quotation}
\begin{quotation}
    But as Unix was being developed, Ken Thompson created the first two versions ofUnix using 
assembly language. He had joined Bell Labs at roughly the same time I had. He was there maybe six 
months or so ahead of us, and he had been assigned to work on the Multics project that BellLabs was 
part of with MIT and GE. When Bell Labs got tired of pouring money into Multics and not getting the 
operating system that it had wanted, it abandoned the project and left Ken Thompson to his 
own devices. Ken thought there were some good ideas in Multics. Being the genius that he was, he 
said, I can do it much more simply and much more elegantly. So he created a rudimentary version of 
Unix and thenkept writing and polishing it. Dennis Ritchie came on the scene. Ken had also created 
a programming language, B. The B was maybe the first letter of BCPL. Who knows? But when Dennis 
Ritchie looked at it, he said, what B needs is a decent type system. So he put a decent type system 
on B, and created theC programming language. Thompson and Ritchie wrote the third version of Unix 
using the newly createdC programming language. I became an early adopter of C, and I had C wired in 
my fingertips, so I could write C programs quite readily, and of course, there were all these neat 
tools that accompanied the programming environment on Unix. There were the text editors. I don't 
know whether you've ever heard of the ED editor or the QED editor that was at MIT as part of 
Multics. QED had regular expressions in it. This triggered my interest in regular expressions. Ken 
Thompson had written a program called grep for doing pattern matching on text files, and it had a 
very limited form of regular expressions when I encountered it.
\cite{aho_oral_history_2022}
\end{quotation}
\begin{quotation}
\textbf{Collaboration with Ullman}

Aho is best known for the textbooks he wrote with Ullman, his co-awardee. 
The two were full time colleagues for three years at Bell Labs, but after 
going back to Princeton as a faculty member Ullman continued to work one day a 
week for Bell.They retained an interest in the intersection of automata theory 
with formal language. In an early paper, Aho and Ullman showed how it was 
possible to makeKnuth's LR(k) parsing algorithm work with simple grammars that 
technically did not meet the requirements of an LR(k) grammar. This technique 
was vital to theUnix software tools developed by Aho and his colleagues at Bell 
Labs. That was just one of many contributions Aho and Ullman made to formal 
language theory and to the invention of efficient algorithms for lexical 
analysis, syntax analysis, code generation, and code optimization. They 
developed efficient algorithms for data-flow analysis that exploited the 
structure of "gotoless" programs, which were at the time just becoming the norm.
\cite{aho_turing_award_2020}
\end{quotation}
\begin{quotation}
\textbf{The Early History of Software, 1952-1968 101}

In the early 1960s computer science struggled to define itself and its 
purpose,in relation not only to established disciplines of electrical 
engineering and applied mathematics, but also in relation to—and as something 
distinct from—the use of computers on campus to do accounting, record keeping, 
and administrative work.58 Among those responsible for the discipline that 
emerged, Professor George Forsythe of Stanford's mathematics faculty was 
probably the most influential. With his prodding, a Division of Computer Science 
opened in the mathematics department in 1961; in 1965 Stanford established a 
separate department, one of the first in the country and still one of the 
most well-regarded.59
\cite{history_of_modern_computing_2003_ceruzzi}
\end{quotation}
\todo{Dragon book; all the books Aho, Ullman and others worked on together.}
\section{Compiler-Compilers}
\todo{Yacc and Lex made with Aho's help. then everyone started making mini languages.AWK. 
"Kernighan and Cherry developed a little language for specifyingmathematics called EQN using these 
tools"}
\begin{quotation}
    People started using the Kernighan and Lorinda Cherry EQN tool to specify mathematics in their 
documents and in the research papers that they were writing. They would feed the EQN specification 
into the typesetting program roff
\dots
Knuth adopted the EQN language to include in the TeX typesetting system, and in LaTeX. It's 
basically Kernighan and Cherry's way of specifying mathematics. These software tools had a great 
deal of influence, and Kernighan and Cherry enjoyed the fruits of parsing theory and formal language 
theory in using the tools Lex and Yacc to create their EQN typesetting language. Knuth has this 
saying that the best theory is motivated by practice and the best practice by theory. I internalized 
that with my early experience in the Computing SciencesResearch Center because I found that the 
theory that we were developing in computer science could be applied to document preparation systems, 
programming languages, compilers, and so on. It was really avery productive environment. I taught 
courses on compiler design at local universities, and then when I went to Columbia, I would teach 
the course on programming languages and their translators
\dots
I might point out that the first Fortran compiler developed by IBM in the 1950s took 18 staff years 
to create. In my programming languages and compilers course, I organized the students into teams of 
four or five. Each team had to create their own programming language, and then write a translator 
for it, and in all the time that I taught the course for almost 25 years at Columbia to thousands of 
students, never did a team failed to deliver a working compiler in the 15-week course, and I 
attribute that to the abstractions
\cite{aho_oral_history_2022}
\end{quotation}
\begin{quotation}
    Aho: Okay. AWK is a programming language that was created by me, Brian 
Kernighan, and Peter Weinberger.

Hsu: And it's your three initials that are in.

Aho: Yes. I'm the A in AWK. Weinberger is the W in AWK and Kernighan is the 
K in AWK.We thought that it was just a throwaway tool for us, nobody really 
would be interested in it. But it's amazing how much routine data processing 
there is in the world.The reason the language got to be known as AWK was because 
when our colleagues would see the three of us in one office or another, and when 
they'd walk past the open door, they'd say, AWK, AWK, AWK as they were going 
down the corridor. So we had no choice but to call it AWK because of the 
good-natured ribbing we got from our colleagues, and because at some Unix 
conference, they passed out t-shirts that had AWK,and the error message saying 
"bailing out on or near line five" on them.
\end{quotation}
\todo{Ratfor, AMPL, other Kernighan languages.}
\todo{continue with typesetting...}
\section{The Dragon Book}
\begin{quotation}
    Jeff had bought into this idea that it's good for your career to write a 
book about what you're working on. In the '70s, with all this work on Unix and 
C, there was a lot of interest in creating new programming languages and 
compilers. As with the algorithms book, what we did was we performed research 
on efficient algorithms for parsing and for some of the other phases of 
compilation, wrote papers on those  and presented them at conferences. But we 
took the important ideas that we developed and the community had developed over 
several decades and codified them into what are now called the dragon books. The 
first dragon book was published in 1977.We did have theorems and proofs in the 
book, and Jeff had this brilliant idea that the book should have a cover with a 
fierce dragon on it representing the complexity of compiler design,and then a 
knight in armor with a lance. The armor and the lance were emblazoned with 
techniques from formal language theory and compiler theory to slay the 
complexity of compiler design
\dots
In the 1980s, more was known about how to construct efficient compilers. We 
invited Ravi Sethi as a third coauthor, he was at Bell Labs at the time, to join 
us in creating the second version of the dragon book. In the first version, the 
dragon was in red. This second version, the dragon was-- sorry. In the first 
version it was in green. In the second version the dragon was in red. What was 
interesting about the red dragon book was there was a movie that was created in 
1995 titled Hackers with a young Angelina Jolie in it, and in the movie, there 
is the uber hacker that's explaining to the new hackers what you have to read to 
become an uber hacker. He shows them 10 papers and books that you must read, and 
one of them was the red dragon book. When my two children saw this movie, and 
they had seen the red dragon book at home, this is the first time they thought 
their old man was really something because he had one of his books in a 
Hollywood movie. It shows what you have to do to impress your kids these days. 
The red dragon book was 800 pages. In2007, we invited Monica Lam as a fourth 
coauthor to create a third version of the dragon book that had a purple dragon 
on the cover and it was close to a thousand pages. None of us had the heart to 
write a fourth book at this point because it just shows how much new knowledge 
had been created in the area of programming languages and compilers and their 
translators, and we continued to do research in this area to keep up with it.
\cite{aho_oral_history_2022}
\end{quotation}
\todo{Bjarne Stroustrup, C++ (1979); Dennis Ritchie, C (1972); Ken Thompson, B (1969); Brian 
Kernighan, AWK (1977), AMPL (1976), co-author of The C Programming Language (1978)}
\section{Commodification}
\todo{Bill Gates and Paul Allen (Microsoft) | Microsoft BASIC (1975) | 
Developed the first critical piece of commercial software for personal 
computers,establishing the doctrine that software should be a purchased, 
proprietarycommodity. Sun microsystems, each part of the company needed to sell 
to all the others,reason why their compiler was paid; proprietary Unix;}
\pagebreak
\section{Timeline}
\begin{figure}[h]
\begin{luacode}
    local start = 1965
    local _end = 1980
    timeline.draw_timeline({    
        start_year = start,    
        end_year = _end,    
        marker_interval = 5,    
        show_year = false,    
        line_always = true,    
        events = {        
        {1967, "Aho joins Bell Labs shortly after Ullman"},        
        {1972, "C"},        
        {1977, "Brian Kernighan, AWK",delta=-.5},        
        {1977, "The Dragon Book first published",delta=.5},        
        {1979, "Bjarne, C++"},    },})
tex.sprint(string.format("\\caption{TBD, %d--%d}", start, _end))
\end{luacode}
\label{fig:tbd-timeline}
\end{figure}


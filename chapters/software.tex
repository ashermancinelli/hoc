
\chapter{Software, 1960-1980ish}

Note that we will discuss some giants of computing history in this chapter who cast long
shadows over the field; however, we will primarily focus on those involved in the
development of compilers and programming languages.
In particular, we will not do justice to key figures from Bell Labs, such as
Brian Kernighan, Ken Thompson, Dennis Ritchie, Claude Shannon, and Doug McIlroy.
I can recommend \textit{The Idea Factory} for a more comprehensive account.

As Hopper and Backus had put it numerous times, programming was still exceedingly
painful at this point in time, and the relative cost of programming kept increasing.
As machines became cheaper and more powerful, the share of the cost of building and
running a program devoted to programming increased, and thus the importance of
ease of programming increased as well.

Michael Mahoney, doctor history and history of science, goes so far as to describe programming
and computer design in this way as early as 1945
\cite[The Structures of Computation]{the-first-computers-2002}:
\begin{quotation}
    The kinds of computers we have designed since 1945 and the kinds of programs we have written for 
them reflect not the nature of the computer but the purposes and aspirations of the groups of people 
who made those designs and wrote those programs, and the product of their work reflects not the 
history of the computer but the histories of those groups, even as the computer in many 
cases fundamentally redirected the course of those histories.
\end{quotation}

While this description was correct about the \textit{direction} compiler engineering
and programming language design were going, the statement was far too strong to make
about the year 1945.
Automatic programming was still scarcely more than synthetic machine code so the programmer
did not have to concern themselves with the details of the particular machine's instruction set
to the same extent, but they were still very much writing programs by writing specific instructions
instead of concerning themselves with the problem they were trying to solve and allowing the
compiler to turn that into proper machine code.

American historian of computing and aerospace Paul Ceruzzi describes the situation in 1968:

\begin{quotation}
    Despite great strides in software, programming always seemed to be in a state of crisis and 
always seemed to play catch-up to the advances in hardware. This crisis came to a head in 1968, just 
as the integrated circuit and disk storage were making their impact on hardware systems. That year, 
the crisis was explicitly acknowledged in the academic and trade literature and was the subject of a 
NATO-sponsored conference that called further attention to it. Some of the solutions proposed were a 
new discipline of software engineering, more formal techniques of structured programming, and new 
programming languages that would replace the venerable but obsolete COBOL and FORTRAN
\footnote{Readers may find it entertaining that Ceruzzi describes COBOL and FORTRAN
as obsolete \textit{by 1968}; both are still heavily used in some industries,
and one of the very first compilers to adopt MLIR (to be discussed in a later chapter)
is a Fortran compiler\cite{spickett_flang_levels_up_2025}.}. Although 
not made in response to this crisis, the decision by IBM to sell its software and services separately 
from its hardware probably did even more to address the problem. It led to a commercial software 
industry that needed to produce reliable software in order to survive. The crisis remained, however, 
and became a permanent aspect of computing. Software came of age in 1968; the following decades would 
see further changes and further adaptations to hardware advances.
\cite{history_of_modern_computing_2003_ceruzzi}
\end{quotation}

Backus describes this as part of the motivation for beginning work on \ftn{}
\cite[\textit{The Economics of Programming}]{hopl_backus_history_of_fortran}:
\begin{quotation}
    Another factor that influenced the development of Fortran was the economics 
of programming in 1954. The cost of programmers associated with a computer 
center was usually at least as great as the cost of the computer itself\dots 
Thus, programming an debugging accounted for as much as three quarters of the 
cost of operating a computer; and obviously, as computers got cheaper, this 
situation would get worse.
\end{quotation}

Thus this chapter is about the efforts to make programming easier and more productive
to address this crisis. Software rose in importance in industry, it became a field of
study in academia, and true research on compilers and programming languages began.

\section{The Software Crisis}
\todo{DEC PDP-8 and PDP-11; IBM System/360 and OS/360; Multics; Unix; C}As the 1960s progressed, 
the notion of 
\textit{software} became more established,and the programs being written served the authors to a 
much greater extent.Prior to the 1960s, programs were often tailor-made for a specific 
machine.There was no hope of re-using the program on another machine.For most of the users, their 
organization had spent a considerable portion of their budget on their system, and they were 
expected to use it for a long time. Retargetable compilers did not exist.

\section{Seymore Cray}
\begin{quotation}
    The CDC 160 and the Origins of the Minicomputer
    
    The Whirlwind (a computer prototype built at 
MIT) had a word length of only 16 bits, but the story of commercial minicomputers really begins with 
an inventor associated with very large computers: Seymour Cray. While at UNIVAC Cray worked on the 
Navy Tactical Data System (NTDS), a computer designed for navy ships and one of the first 
transistorized machines produced in quantity. Around 1960 Control Data, the company founded in 1957 
that Cray joined, introduced its model 1604, a large computer intended for scientific customers. 
Shortly thereafter CDC introduced the 160, designed
\cite{nothing_new_since_von_neumann_2000}
\end{quotation}
\section{The DEC VAX and the IBM System/360}
\begin{quotation}
    Through the 1980s the dominant mainframe architecture continues to be a descendent of the IBM 
System/360, while the dominant mini was the DEC VAX, which evolved as a 32 bit extension of the 
16-bit PDP-11.
\cite{nothing_new_since_von_neumann_2000}
\end{quotation}

\section{Bell Lab's Computing Science Research Center}

\subsection{Multics and Unix at Bell Labs}

There is no avoiding Multics and UNIX if we are to discuss the development of compilers at Bell Labs.
We will cover the history of the American Telephone and Telegraph Company (AT\&T)
in \textit{very} broad strokes to contextualize the development of the programming languages
and compiler tools developed in this time period.

In very broad strokes, American Telephone and Telegraph Company (AT\&T) was a government-sanctioned
monopoly that provided telephone service to the United States.
They became a monopoly by buying up smaller telephone companies across the country.
Over time, they found they needed quite a bit of fundamental science in order to keep up with the
requirements of servicing the entire country, so in 1925 they created Bell Telephone
Laboratories in New York. If they did not continue providing quality service, the government could
(and very often did) threaten to break up the company, so this need was existential.

During the second world war, they expanded to New Jersey, early 1960s they moved to Murray Hill,
in suburban New Jersey, where they had a few thousand employees.
The laboratory was responsible solely for research, and thousands of other employees
worked on applying the research and developing the process knowledge needed to bring the
research into the phone system.
This research led to numerous fundamental discoveries and inventions in physics,
materials science and many more domains; these discoveries include the transistor, lasers,
fiber-optics and solar cells, to name a small subset.
Patents were the laboratory's primary measure of success, and they produced many.
At its peak, AT\&T was the nation's largest private employer with over 1 million employees.
If you are interested in an account of Bell Labs in anything close to the proper detail,
I recommend \textit{The Idea Factory} by Jon Gertner; our focus if far too narrow to do
it justice here.

In the early 1960s, the labs spun off a new department, the Computing Sciences Research Center,
off of the mathematics group. This group had about 25 people tasked with software and computing
research, which included operating systems.
In 1961, Fernando Corbat\'{o} at MIT created a timesharing system called CTSS,
or the Compatible Time Sharing System, with relative success.
In 1965 MIT set out to make a successor system with all kinds of enhancements; they partnered with
Bell Labs and General Electric to create the operating system called \textit{Multics}.
The machine was developed at GE (Multics required a special machine), Multics was designed at MIT,
and Bell Labs was responsible for large parts of the software development,
which was primarily done in IBM's PL/I programming language.
Ken Thompson described it as "monstrously over-engineered" and "typical second-system syndrome"
\cite{kernighan_interviews_thompson_2019};
while they had a wonderful time-sharing system in CTSS, they got too ambitious with thier sophomore project.
Ultimately, this project ran long past deadlines and was slow to run and altogether not what
the labs was hoping for, so in 1969 they pulled out of the project.
Honeywell took over Bell Labs' share of the project.

This left a bad taste in their mouths, and Bell Labs leadership sought to avoid operating
systems work in the wake of this venture.
Some of the members of the Multics team had other ideas; they had gotten a taste for
operating systems work, and now they had not operating system to work \textit{on}.
These included Ken Thompson, Dennis Ritchie, and Doug McIlroy.

At the time, Ken Thompson was researching file systems on an outdated and underused
Digital Equipment Corporation (DEC) PDP-7 machine from the acoustics group.
\footnote{Because acoustics research was so core to Bell Labs' and AT\&T's mission,
	that department got just about anything they asked for in terms of computing resources,
	hence the outdated PDP-7 sat unused in their offices--they already had more powerful machines.}
He originally just used it for writing video games, until finally getting around to
his research on file systems.
He developed disc scheduling algorithms to maximize throughput on the PDP-7's disc drive.
To adequately test the performance of his file system algorithms, he needed some test
programs to stress the file system.
At some point in this process, he realized he was not far off from an
operating system of his own\cite{kernighan_interviews_thompson_2019}:

\begin{quotation}
	At some point, I realized with out knowing it up until that point, that I was
	three weeks from operating system, with three programs, one a week. I needed an
	editor to write code, I needed an assembler to turn the code into language I
	could run, and I needed a kernel overlay, call it an operating system.
	Luckily, right at that moment, my wife went on a three-week vacation to take my
	one-year-old to visit her parents in California. One week, one week, one week,
	and we had Unix.
\end{quotation}

Thompson's assembler was not a complete toolchain; there were no libraries, linkers, or loaders.
The entire assembly program had to be presented to the assembler all at once,
and the output was directly executable. Thus the \texttt{a.out} naming convention for
the default executable output of modern compilers was born (\textit{a}ssembly \textit{out}put).
Even after the system gained more toolchain components, the name remained.

Very early on, Thompson's system picked up very impressive users.
Two at a time, Dennis Ritchie, Doug McIlroy, Morris McMahon, and Brain Kernighan
all became users on Thompson's new system.
They sent a proposal to get a PDP-10, which was top-of-the-line at the time,
to port Unix to a more powerful machine.
This was rejected as soon as Labs leadership saw it was related to operating systems,
in spite of the fact that their request was well within budget.
Joe Ossanna (another colleague) came up with a proposal that
positioned their need for a PDP-10 as a research project to develop better typesetting
software for filing patent applications (notably missing any mention of operating
systems).
Ossanna developed the Nroff and Troff typesetting programs to fullfil their claimed
goal of helping the patent office.
The patent office loved their software, and they ended up buying the computing group
a PDP-11\footnote{While many resources will mention the PDP-11, they usually refer to the
	PDP-11\textit{/22}; very few of original PDP-11s (which the computing center worked on)
	were ever produced, but the refreshed models had many users.} sometime in the early 1970s.
Unix came up on the PDP-11 almost immediately.
Thus we have the basis for discussing the subsequent development of compilers
and programming languages at Bell Labs.


\section{Personal Histories}

The following sections will feature many of the same figures;
here we will introduce them so their interleaving stories may be told
uninterrupted in the following sections.
Readers unfamiliar with the history of Unix may find unfamiliar terms in these
personal accounts; these will be discussed in the following sections after
our characters have been introduced.

\subsection{Doug McIlroy, Joined 1958}

Douglas McIlroy was born in Fishkill, New York.
His father worked in electrical engineering and spent time at
MIT and ended his career at Cornel, having contributed to the RADAR effort in World War II.
His father collected maps, and Doug grew an interest in maps himself.
One day, while ill in bed with chicken pox, his father assigned him the task of drawing
a dam on a USGS map, and showing which regions would be inundated by the dam
\cite{doug_mcilroy_oral_history_2019}.
His mother, too, had a master's degree in physics from the University of Rochester,
which was very unusual at the time. She was forced to audit some of her classes there,
because "women can't take this class! But you can sit in on it, if you want."
Thus he was raised in an engineering-minded household.

Barbara, his wife, also studied mathematics and faced discouragement similar to his mother's.
They ended up meeting at the laboratory, where Doug recalled that some figures at the labs sought
to improve the situation for women, specifically Bernie Holbrook.

Doug joined Bell Labs in 1958 and shortly thereafter earned his
PhD in applied mathematics from MIT.
McIlroy became head of the Computing Techniques Research department in 1965, and he's regarded
as one of the most brilliant members of the staff by key members that readers may be more familiar with.

He is described by nearly everyone at the Bell Labs Computing Research Center
(who wrote or spoke about their time there)
as the most brilliant member of that team that no one has heard of
\cite{kernighan_interviews_thompson_2019}; Ken Thompson described him as
"the smartest one of all of us and the least remembered or written down of all of us."

He was relatively hands-off as a manager, preferring to peek into his employees' offices
with suggestions or interesting problems and wait for the employees to let \textit{him} know
what needed to be done, rather than assigning tasks directly.
His employees respected his tasted in technical and personal matters,
as Brian Kernighan describes\cite{kernighan_unix:_2020}:

\begin{quotation}
Unix might not have existed, and certainly
would not have been as successful, without Doug's good taste and sound
judgment of both technical matters and people.
\end{quotation}

\begin{figure}
    \centering
    \includegraphics[width=0.7\textwidth]{resource/software/unix/doug-1964-pipes.png}
    \caption{Doug McIlroy's 1964 memo proposing Unix pipes\cite{doug_mcilroy_origin_of_unix_pipes_1964}.}
    \label{fig:unix-pipes-mcilroy-memo}
\end{figure}

In 1964, he circulated a memo which led to Unix pipes\cite{doug_mcilroy_origin_of_unix_pipes_1964}
(see \ref{fig:unix-pipes-mcilroy-memo}) though it took him quite a while to convince
Ken Thompson that he really ought to implement them.
Aside from pipes, he wrote many common Unix utilites we still use today, like
\texttt{diff}, \texttt{sort}, \texttt{join}, \texttt{tr}, \texttt{echo}, \texttt{tee}, and \texttt{spell}.

He hired Alfred Aho in 1967\cite{aho_oral_history_2022}:

\begin{quotation}
I was interviewed by a department head by the name of Doug McIlroy. He was an applied
mathematician from MIT. He had been at Bell Labs for a few years before me. Amongst other things, he
had co-invented macros for programming languages and he's also in this class of one of the smartest
people I've ever met.
\end{quotation}

\subsection{Ken Thompson, Joined 1966}



\subsection{Dennis Ritchie, Joined 1967}



\subsection{Alfred Aho, Joined 1967}

\begin{quotation}
    One of the first people that I met at Princeton was a Columbia graduate by thename of Jeffrey 
Ullman. He had just gotten his undergraduate degree fromColumbia University and also had come to 
study digital systems in the EEdepartment at Princeton. So, he and I became close friends. When we 
graduated from Princeton, we both joined the newly formed Computing Sciences ResearchCenter at Bell 
Labs. There we developed a lifelong collaboration on subjects ranging from algorithms, programming 
languages, to the very foundations of computer science. I was very fortunate to have met some of the 
greatest people in the field and to have gotten to know them and work with them. You learn so much by 
working with the best people in the field. So, I felt very blessed because I had this kind of 
background
\dots
Hsu: Before we jump into Bell Labs more deeply, could you maybe explain-- talk about your PhD 
thesis,but try to explain it to somebody who, maybe like a museum goer who doesn't really know much 
about computer science and linguistics.

Aho: This is interesting. As I mentioned, Hopcroft told me, "Find your own research problem." He 
did teach a course in automata and language theory, so I got introduced to formal language theory 
and automata theory, at least, as it was known at that time. I was interested in programming 
languages and compilers. What I noticed was that a programming language has a syntax and a 
semantics. All languages have a syntax and a semantics. If you want to write a translator for a 
programming language, or even a natural language, you have to understand the syntax and semantics of 
your source language and the target language
\dots

Hansen: 1967, and you followed Ullman there. He had already joined Bell Labs before.

Aho: A few months before me.

Hansen: A few months before. And what group was it that you joined?

Aho: I was interviewed by a department head by the name of Doug McIlroy. He was an 
applied mathematician from MIT. He had been at Bell Labs for a few years before me. Amongst other 
things, he had co invented macros for programming languages and he's also in this class of one of the 
smartest people I've ever met.Jeff wanted to go to academia a little bit earlier than I did, like 
many years earlier. He stayed at Bell Labs for a few years and went to PrincetonUniversity where he 
joined the faculty of the electrical engineering department, but he would come and spend one day a 
week consulting at Bell Labs.His consulting stint was he would come Fridays and sit in my office 
all day.The conversations that we'd have would range over all sorts of topics, and sometimes he'd 
mentioned that he was working on a problem with a colleague atPrinceton, and after describing the 
problem, I might say, "You're kidding," and he said, "Oh, you're right. The solution is obvious, 
isn't it?" I don't know whether I would say dynamic programming or whatever, but several papers 
came out of this intense collaboration, and we got to the point where we could communicate with just 
a few words. We had a very large, shared symbol table.
\cite{aho_oral_history_2022}
\end{quotation}
\begin{quotation}
    But as Unix was being developed, Ken Thompson created the first two versions ofUnix using 
assembly language. He had joined Bell Labs at roughly the same time I had. He was there maybe six 
months or so ahead of us, and he had been assigned to work on the Multics project that BellLabs was 
part of with MIT and GE. When Bell Labs got tired of pouring money into Multics and not getting the 
operating system that it had wanted, it abandoned the project and left Ken Thompson to his 
own devices. Ken thought there were some good ideas in Multics. Being the genius that he was, he 
said, I can do it much more simply and much more elegantly. So he created a rudimentary version of 
Unix and thenkept writing and polishing it. Dennis Ritchie came on the scene. Ken had also created 
a programming language, B. The B was maybe the first letter of BCPL. Who knows? But when Dennis 
Ritchie looked at it, he said, what B needs is a decent type system. So he put a decent type system 
on B, and created theC programming language. Thompson and Ritchie wrote the third version of Unix 
using the newly createdC programming language. I became an early adopter of C, and I had C wired in 
my fingertips, so I could write C programs quite readily, and of course, there were all these neat 
tools that accompanied the programming environment on Unix. There were the text editors. I don't 
know whether you've ever heard of the ED editor or the QED editor that was at MIT as part of 
Multics. QED had regular expressions in it. This triggered my interest in regular expressions. Ken 
Thompson had written a program called grep for doing pattern matching on text files, and it had a 
very limited form of regular expressions when I encountered it.
\cite{aho_oral_history_2022}
\end{quotation}
\begin{quotation}
\textbf{Collaboration with Ullman}

Aho is best known for the textbooks he wrote with Ullman, his co-awardee. 
The two were full time colleagues for three years at Bell Labs, but after 
going back to Princeton as a faculty member Ullman continued to work one day a 
week for Bell.They retained an interest in the intersection of automata theory 
with formal language. In an early paper, Aho and Ullman showed how it was 
possible to makeKnuth's LR(k) parsing algorithm work with simple grammars that 
technically did not meet the requirements of an LR(k) grammar. This technique 
was vital to theUnix software tools developed by Aho and his colleagues at Bell 
Labs. That was just one of many contributions Aho and Ullman made to formal 
language theory and to the invention of efficient algorithms for lexical 
analysis, syntax analysis, code generation, and code optimization. They 
developed efficient algorithms for data-flow analysis that exploited the 
structure of "gotoless" programs, which were at the time just becoming the norm.
\cite{aho_turing_award_2020}
\end{quotation}
\begin{quotation}
\textbf{The Early History of Software, 1952-1968 101}

In the early 1960s computer science struggled to define itself and its 
purpose,in relation not only to established disciplines of electrical 
engineering and applied mathematics, but also in relation to—and as something 
distinct from—the use of computers on campus to do accounting, record keeping, 
and administrative work.58 Among those responsible for the discipline that 
emerged, Professor George Forsythe of Stanford's mathematics faculty was 
probably the most influential. With his prodding, a Division of Computer Science 
opened in the mathematics department in 1961; in 1965 Stanford established a 
separate department, one of the first in the country and still one of the 
most well-regarded.59
\cite{history_of_modern_computing_2003_ceruzzi}
\end{quotation}
\todo{Dragon book; all the books Aho, Ullman and others worked on together.}
\section{Compiler-Compilers}
\todo{Yacc and Lex made with Aho's help. then everyone started making mini languages.AWK. 
"Kernighan and Cherry developed a little language for specifyingmathematics called EQN using these 
tools"}
\begin{quotation}
    People started using the Kernighan and Lorinda Cherry EQN tool to specify mathematics in their 
documents and in the research papers that they were writing. They would feed the EQN specification 
into the typesetting program roff
\dots
Knuth adopted the EQN language to include in the TeX typesetting system, and in LaTeX. It's 
basically Kernighan and Cherry's way of specifying mathematics. These software tools had a great 
deal of influence, and Kernighan and Cherry enjoyed the fruits of parsing theory and formal language 
theory in using the tools Lex and Yacc to create their EQN typesetting language. Knuth has this 
saying that the best theory is motivated by practice and the best practice by theory. I internalized 
that with my early experience in the Computing SciencesResearch Center because I found that the 
theory that we were developing in computer science could be applied to document preparation systems, 
programming languages, compilers, and so on. It was really avery productive environment. I taught 
courses on compiler design at local universities, and then when I went to Columbia, I would teach 
the course on programming languages and their translators
\dots
I might point out that the first Fortran compiler developed by IBM in the 1950s took 18 staff years 
to create. In my programming languages and compilers course, I organized the students into teams of 
four or five. Each team had to create their own programming language, and then write a translator 
for it, and in all the time that I taught the course for almost 25 years at Columbia to thousands of 
students, never did a team failed to deliver a working compiler in the 15-week course, and I 
attribute that to the abstractions
\cite{aho_oral_history_2022}
\end{quotation}
\begin{quotation}
    Aho: Okay. AWK is a programming language that was created by me, Brian 
Kernighan, and Peter Weinberger.

Hsu: And it's your three initials that are in.

Aho: Yes. I'm the A in AWK. Weinberger is the W in AWK and Kernighan is the 
K in AWK.We thought that it was just a throwaway tool for us, nobody really 
would be interested in it. But it's amazing how much routine data processing 
there is in the world.The reason the language got to be known as AWK was because 
when our colleagues would see the three of us in one office or another, and when 
they'd walk past the open door, they'd say, AWK, AWK, AWK as they were going 
down the corridor. So we had no choice but to call it AWK because of the 
good-natured ribbing we got from our colleagues, and because at some Unix 
conference, they passed out t-shirts that had AWK,and the error message saying 
"bailing out on or near line five" on them.
\end{quotation}

\subsection{Jeffrey Ullman, Joined 1967}



\subsection{Brian Kernighan, Joined 1969}


\subsection{The First Unix Compilers}

Alfred Aho described this era of compiler history
as as cambrian explosion of programming languages and compiler tools
\cite{aho_bell_labs_role_in_programming_languages_2025},
and there were several reasons for that.
Tools were developed that made the description of compiler front-end
tools extremely simple, the computing center got access to machines with
more memory, and the theory of parsing context-free grammars advanced.
The Labs employees worked all of these together to produce many compiler
tools.

While Doug was the department head of the Computing Sciences Research Center,
he would regularly stop into the offices of his employees and drop ideas and requests
to them. In one case, he mentioned to Ken how convenient it would be to have
a tool for searching text files for patterns.
Ken already had a rough version of such a tool in his home directory,
so the next day he showed Doug his program and thus was born \texttt{grep};
many such tools came to be like this.

On another similar occasion, Doug walked into Ken's office to discuss the TMG language
(standing for TransMoGrifier),
designed by Robert McClure\cite{mcclure_tmg_compiler_compiler_1965}, one of his friends at the lab.
McClure had gotten defensive about this language, and when he left the lab,
he claimed it was proprietary and nobody else could use it.
TMG was a top-down recursive-descent compiler-compiler for context-free grammars
and procedural elements similar to Yacc, a tool to help write compiler
front-ends.
See this example from Doug's TMG manual\cite{tmg_manual_mcilroy_1972}
and notice the similarities to modern parser generators and Backus-Naur Form:

\begin{quotation}
	This simple program defines the translation of fully parenthesized infix
	expressions to Polish postfix for a stack machine.
	\begin{lstlisting}[frame=single]
    expr: <(>/exp1 expr operator expr <)> = { 3 1 2 };
    exp1: ident = { < LOAD > 1 };
    operator:
    op0: <+>/op1 = { < ADD > };
    op1: <->/op2 = { < SUB > };
    op2: <*>/op3 = { < MPY > };
    op3: </>     = { < DIV > };
\end{lstlisting}
\end{quotation}

Ken recounts\cite{kernighan_interviews_thompson_2019}
the story of how Doug brought a full TMG compiler \textit{written in TMG,
	on paper} to Ken's office. He then decided to feed the TMG program into his TMG compiler
\textit{by hand} to produce an assembly listing of the TMG program.
He then went over to Ken's keyboard and typed in the program that his TMG-TMG compiler
had produced, and with astonishingly few errors, they had a working TMG compiler on the PDP-7.

Ken's logical next step was to use this new TMG compiler to
write a \ftn{} compiler for the PDP-7 because
"no computer was complete without \ftn{}."
Now, the PDP-7 was only 8k of 18-bit words of memory and Ken's Unix system needed
about half of that just to run, leaving only 4k for user programs.
Once completed, the new \ftn{} compiler took far more memory than was available for
user programs, so he had to cut down the TMG program to get a program small enough to
fit on the machine but capable enough to facilitate a usable programming language.
Once he finally cut down his compiler to 4k, he called that programming language \textit{B}.
His ill-fated attempt at producing a \ftn{} compiler ended up being a very different language,
based more so on BCPL than on \ftn{}
\footnote{The origins are really unclear. Ken himself mentions \ftn{} as his only inspiration
	for B in a live interview\cite{kernighan_interviews_thompson_2019},
	however Dennis Ritchie describes it as a "language of his own" and
	"BCPL squeezed into 8K bytes of memory and filtered through Thompson's brain"
	\cite{development_of_c_language_chist_ritchie_1996}.
	He describes the name as either a contraction of BCPL or (less likely) a contraction
	of Bon, a language Thompson developed for Multics.}.
BCPL had been designed by Martin Richards at MIT in the mid-1960s, and thus had found
its way onto MIT's CTSS system, which Bell Labs employees were familiar with.

\subsection{BCPL and B}

BCPL, B and C may differ syntactically, but semantically they are very similar
and operate at a similar level of abstraction.
The BCPL compiler was allowed to use more memory than Thompson had on the PDP-7,
so it allowed for more complicated expression-oriented language features.
Dennis Ritchie points out the following two examples as only being possible
in BCPL and not in B or C due to memory constraints
\cite{development_of_c_language_chist_ritchie_1996}:

\begin{lstlisting}[language=c,frame=single]
let P1 be (*@\textbfit{procedure}@*)
and P2 be (*@\textbfit{procedure}@*)
and P3 be (*@\textbfit{procedure}@*)

E1 := valof ( (*@\textbfit{declarations}@*) ;
              (*@\textbfit{commands}@*) ;
              resultis E2 ) + 1
\end{lstlisting}

Because the entire program was held in memory, the BCPL compiler could make
procedure \textbfit{P3} available to expressions inside procedure \textbfit{P1},
but the B compiler was limited to a one-pass technique and could not provide such features.
On the other hand, some BCPL features were purposefully left out of B; for example,
to share data between separately compiled source files, BCPL programs had to use
a global buffer. To make a procedure or variable available to other source files,
the programmer had to manually associate the name with an offset into this buffer,
similar to the \texttt{COMMON} block in \ftn{} programs.
In more mature B compilers and later in C, the linker handles this automatically.

\begin{figure}[h]
	\centering
	\includegraphics[width=0.8\textwidth]{resource/software/unix/bell-threaded-code-figures.png}
	\caption{Execution of Threaded Code\cite{bell_threaded_code_1973}}
	\label{fig:bell-threaded-code}
\end{figure}

The B compiler did not emit machine code directly, but instead used
\textit{threaded code}\cite{bell_threaded_code_1973}, an early form of \gls{bytecode}
operating on a simple stack machine, which was then interpreted.
James Bell claimed that this threaded code needed no interpreter,
but really, the execution model for threaded code is just a specification
for an interpreter without outside the routines responsible for executing a given
bytecode instruction.
Today this would be considered a bytecode interpreter, but in the day when interpreters
were only known to interpret the source language directly, his ideas may have seemed
distinct from other forms of interpretation.
See Bell's depiction of the execution of threaded code \ref{fig:bell-threaded-code}.

Neither B nor BCPL had types; only the addressable size of data for the particular machine.
Since pointers were just integers representing offsets into memory, pointer arithmetic
was natural in both languages, and both had syntax sugar to facilitate pointer arithmetic
and dereferencing of offsets into arrays.

\begin{lstlisting}[language=c,frame=single]
/* Declaring an array of 10 integers */
let V = vec 10 /* BCPL */
auto V[10];    /* B    */

/* Accessing element i of array V */
V!i            /* BCPL */
*(V+i)         /* B    */
V[i]           /* B    */
\end{lstlisting}

After Ken had the TMG version of B working, he \gls{bootstrap}ped the compiler
by implemented it in B.
Dennis Ritchie recalls\cite{development_of_c_language_chist_ritchie_1996}:

\begin{quotation}
	During development, he continually struggled against memory limitations: each
	language addition inflated the compiler so it could barely fit, but each
	rewrite taking advantage of the feature reduced its size. For example, B
	introduced generalized assignment operators, using x=+y to add y to x. The
	notation came from Algol 68 via McIlroy, who had incorporated
	it into his version of TMG.
\end{quotation}

Dennis also recalls this about their language design and implementation decisions:

\begin{quotation}
	Although we entertained occasional thoughts about implementing one of the major
	languages of the time like Fortran, PL/I, or Algol 68, such a project seemed
	hopelessly large for our resources: much simpler and smaller tools were called
	for. All these languages influenced our work, but it was more fun to do things
	on our own.
\end{quotation}

By 1970, the team had been able to acquire the PDP-11, which they promptly ported Unix to,
along with the B \gls{bytecode} interpreter and a small assembler written in B.
As the number of PDP-11 users grew, so too did their B codebase, which now included
quite a few programs and subroutines, as a result of the tedium of writing assembly,
which was the only alternative.
B proved useful--however, the new machine revealed some serious inadequacies.
The only data type in BCPL and B was the \texttt{cell}, roughly equivalent to the hardware's
machine word.
The 18-bit PDP-7\footnote{18-bit might seem like an odd choice to the modern reader;
	most mainframe computers had 36-bit words, and the PDP-7 was marketed as a smaller system.
	Thus at 18-bits, it would be marketed as roughly half a mainframe.}
did not map cleanly to the 16-bit PDP-11.
BCPL on the PDP-7 had support for floating-point operations through special operators, but this
only worked because the floats could fit in a single word and this was not possible on the PDP-11.

\begin{quotation}
	For all these reasons, it seemed that a typing scheme was necessary to cope
	with characters and byte addressing, and to prepare for the coming
	floating-point hardware. Other issues, particularly type safety and interface
	checking, did not seem as important then as they became later.
	Aside from the problems with the language itself, the B compiler's
	threaded-code technique yielded programs so much slower than their
	assembly-language counterparts that we discounted the possibility of recoding
	the operating system or its central utilities in B.
	\cite{development_of_c_language_chist_ritchie_1996}
\end{quotation}

In 1971, Dennis Ritchie began extending B by adding a character type and directly generating
PDP-11 machine code.
Thus the creation of a compiler capable of generating efficient machine code on a system with
very little memory and the transition from B to C were simultaneous.
Dennis renamed the B compiler to \texttt{NB}, or \textit{New-B}, and then \textit{C}
shortly thereafter.

\subsection{C}

Dennis's new language now had types and new addressing schemes, and it evolved quickly.
\textit{New-B} existed so briefly that no full description of its syntax or semantics was
ever written.

The run-time overhead associated with BCPL and B's vector indexing semantics were
addressed in C:

\begin{quotation}
	Values stored in the cells bound to array and pointer names were the machine
	addresses, measured in bytes, of the corresponding storage area. Therefore,
	indirection through a pointer implied no run-time overhead to scale the pointer
	from word to byte offset. On the other hand, the machine code for array
	subscripting and pointer arithmetic now depended on the type of the array or
	the pointer: to compute \texttt{iarray[i]} or \texttt{ipointer+i} implied
	scaling the addend \texttt{i} by the size of the object referred to.
	\cite{development_of_c_language_chist_ritchie_1996}
\end{quotation}

C gained a type system, overlaying semantics onto the raw bits that BCPL and B users
were working with.
The type system was heavily inspired by ALGOL 68, though not necessarily in a way the
authors of the ALGOL language would have been happy with, as Dennis himself admits
\cite{development_of_c_language_chist_ritchie_1996}.
Many features were added to the language in a short time; some of the warts that persist in
even modern C were because of Dennis's desire to make porting programs from B to C as
simple as possible.

For example, the precedence of \texttt{\&} and \texttt{==} were made equal in C,
thereby requiring parentheses in the second expression below:

\begin{lstlisting}[language=c,frame=single]
if (a == b & c) { /* ... */ }
if ((c & b) == a) { /* ... */ }
/*  ^ Parentheses required here */
\end{lstlisting}

While many features were added to C in 1972-1973, the preprocessor was likely the most
significant. Dennis implemented it partly at the request of Alan Snyder
(who had also suggested that Dennis add the \texttt{\&\&} and \texttt{||} to make
boolean logic more explicit) at MIT
in \citetitle{snyder_portable_compiler_for_c_1975}, and
partly inspired by the textual inclusion mechanisms in PL/I and BCPL.
The first version could only \texttt{\#include} other files and do simple textual
substitution, though it was soon extended by Mike Lesk and John Reiser to handle macros
with arguments and support conditional compilation.

In the summer of 1973, the language and compiler were mature enough that the team was able
to rewrite the Unix kernel in C.
Ken Thompson had tried once before to port Unix to C, but the language did not support
structures at the time, making the task too onerous to continue.
The compiler was also ported to the Honeywell 635 and IBM 360 and 370, and commonly used
procedures coalesced into what eventually became the standard library.
The first of these major library components was Mike Lesk's input/output library, which
became the basis of the C \textit{standard I/O}.
Dennis and Brian Kernighan wrote the seminal \citetitlecite{the_c_programming_language_1ed_ritchie_kernighan_1989}.
Brian wrote most of the exposition and Dennis wrote most of the programs.
This would come to be common; Brian's gift for technical writing that was easy to understand
made him a common author of documentation and publications.

The subsequent period was filled with language features focusing on type safety
and portability as Unix and all the tools they had rewritten in C had to be ported to more and
more systems, and much of the codebase remained untyped, harkening back to its B and assembly
heritage. In the same vein, pointers and integers were mutually assignable in C,
absent of any syntactic indication that the programmer's intent had changed.
This practice fell out of style, and \textit{casts} were introduced to the language,
inspired by a similar \textit{but distinct} concept in ALGOL 68.
Similarly, the member-dereference operation did not care much about the type of the pointer
being dereferenced; \texttt{ptr->field} could be used regardless of the type of \texttt{ptr},
and \texttt{field} was taken to mean an offset from some pointer into a structure.
\footnote{
	Perhaps they should have preserved more of ALGOL 68's type system; these pointer rules
	plague optimizers working with C programs to this very day.
	The modern Linux kernel is \textit{not} compiled with a flag usually spelled
	\texttt{-fstrict-aliasing}, which allows the optimizer to take advantage of the C language's
	aliasing rules. These aliasing rules specify that a pointer cannot refer to multiple incompatible types,
	and programs that use such behavior are \gls{ub}.
	This flag would theoretically allow the compiler to better optimize the Linux kernel,
	however, strict aliasing rules are broken so often in the kernel's source that the flag
	results in a buggy kernel, so it cannot be enabled.
	Programs that make use of casting to and from \texttt{void*} or
	type-pun through unions are also common in C,
	and in general the aliasing rules are not very friendly to optimization when compared to
	Fortran, for example.
}

In 1978, Steve Johnson started work on a C compiler that was easy to retarget to other machines
while he, Dennis and Thompson worked on porting Unix to the Interdata 8/32.
While the language continued to evolve and some older practices fell out of style, Johnson
came up with the \texttt{lint} program, adapted from his \textit{Portable C Compiler} \texttt{pcc},
to search a set of source files for coding practices that might need to be updated.
This is one of the first (if not \textit{the} first) static-analysis tool, its namesake lives on
in today's tools; one still talks of a \textit{linter} remarking on a dubious construct
in one's code.

The third version of Unix, System III, was the first one to be primarily written in C.
As Unix matured and its System III and System V versions were distributed,
institutions outside AT\&T developed Unix distributions
(particularly at the University of California at Berkeley, yielding the Berkeley
Software Distribution of Unix), and C compilers were ported to many new machines,
C became a mainstay of the software world, as it remains today.

\subsection{Standardization of C}

Using pcc as a reference compiler became impractical.
The X3J11 committee formed in 1983 to standardize C.
Continue with \citetitle{development_of_c_language_chist_ritchie_1996}.
\todo{includes early criticism of C.}

\subsection{Regular Expressions: Grep, Yacc, Lex}
\label{sec:software-unix-regex}

Regular expressions were a large part in the explosion of new programming languages
and compiler tools at the labs in the early 1970s.

One such use of regular expressions was Ken Thompson's \texttt{s}
program ("s" for "search") to do pattern-based searching
in his home directory sometime after working on Unix on the PDP-11.
In one of those casual drop-ins, his boss McIlroy asked him to come up with a program for finding
patterns in files and searching through directories and these things, to which Ken responded
"ah, let me think about it overnight" knowing he already had the perfect program.
After a night of cleaning up his program, he gave it to Doug who loved the program
and moved it from Ken's home directory into the shared binary directory.
It was at this time renamed \texttt{grep} after the command in the \texttt{ed} editor
which performed the same function: \texttt{g/re/p}, for global, regular-expression, print.
Many small programs like this formed over time at the Labs.

On another occasion, Steven Johnson came into Alfred Aho's office to ask about
parsing algorithms for his Portable C Compiler. He was having trouble getting the C parser
working properly, so he went to Aho for his theoretical background in parsing algorithms
and automata theory and asked how he would go about constructing a parser.
Aho replied that first he would construct a grammar for the language, and then apply
Knuth's LAR parsing algorithm--Aho got some sheets of cardboard from the stockroom,
and did the sets-of-items construction for the grammar that Johnson had given him
over the weekend.

On Monday, he would give Johnson the constructed parsing table, Johnson would implement it,
and find issues with Aho's cardboard construction. After a few iterations of this, Johnson
asked Aho to explain the theory, which he implemented. This program went on to become the
\texttt{yacc} program, standing for \textit{Yet Another Compiler-Compiler}, which is used
for generating parsers from a context-free grammar.
Aho and Johnson continued to improve the performance of the program and it vastly decreased the
effort required to create a parser for a new language.
Johnson's Yacc program was especially useful because it would inform the user of shift-reduce
and reduce-reduce conflicts, so they knew if they had any difficult-to-parse constructs in their
grammar.
Yacc became one of the most widely used tools for creating parsers, even outside Bell Labs
in computer science courses.
\todo{Aho's egrep program that constructed the regexes lazily.}

In a mailing list, Johnson recalled that:
\begin{quotation}
	The name cames as a result of a comment by Jeff Ullman -- Al was
	telling him about the program,
	and he said "What? Yet Another Parser Generator?"
	So I started calling it Yacc.

	The first implementation was really slow.  We were all sharing a
	single PDP 11 with Unix on it, and
	Yacc could take 20 minutes to generate a 50-rule grammar table.
	Everyone would groan when  i
	started the program: "Ohhhh.  Johnson's running Yacc again".
\end{quotation}

After Al and Johnson's optimization efforts, the group was eventually able to
parse \Gls{F77}.

One of the fast regular expression algorithms that Aho had come up with also saw use in the
\texttt{lex} program, which is used for generating lexical analyzers from regular expressions.
Eric Schmidt, summer intern at Bell Labs but perhaps better known today as the
CEO and co-founder of Google, used one of Aho's algorithms and worked with Mike Lesk of Bell Labs
to produce the first version of \texttt{lex}.
Using these tools in tandem allowed for very tight iteration cycles for designing and
developing new programming languages and compilers. With Lex for the lexer and Yacc for the parser,
one could quickly develop the front-end of a compiler based only on the grammar for the language.

\subsection{Applications of Yacc and Lex: Eqn and AWK}

In 1974, Brian Kernighan and Linda Cherry developed one of the first programs
using Lex and Yacc to develop compilers.
It was called \texttt{eqn} and was used for typesetting mathematical expressions,
usually as a part of \textit{troff} documents.
At the time, troff was the state-of-the-art typesetting tool,
and it got lots of use at Bell Labs (recall that the Lab's primary output was publications).

Troff documents were marked by beginning and ending characters with a leading perdiod.
An Eqn equation in a troff document would start with \texttt{.EQ} and end with \texttt{.EN}\cite{kernighan_cherry_eqn_1975}:

\begin{lstlisting}[frame=single]
.EQ
left [ x + y over 2a right ]~=~1
.EN
\end{lstlisting}

The grammar for Eqn was defined by Yacc rules like so:

\begin{lstlisting}[frame=single]
eqn : box | eqn box
box : text
    | { eqn }
    | box SUB box | box SUP box
    | [ROMAN | BOLD | ITALIC] box
\end{lstlisting}

This would embed the equation $ [ \frac{x + y}{2a} ] \approx 1 $ in the document.
This was incredibly useful! Prior to typesetting programs, an author would need to deliver their
hand-written notes to a typesetter to produce a draft before sending the typeset drafts to
publishers and editors. Placing the typesetting programs in the hands of the authors themselves
allowed them to see more or less the final product as they wrote.

Usually, papers would not be typeset until they were already accepted and published.
When the Bell Labs employees started writing papers with Eqn and Troff, their papers
would be rejected by converences and editors because they "didn't want to publish anything that
had already been published elsewhere."
Of course these papers had not already been published, but they were so refined
that the reviewers assumed they had been.

Kernighan and Cherry wanted to be able to typeset equations in publications by typing
something similar to plain-English descriptions, similar to how one mathematician would
describe an equation to another over the phone.
Al and Steve helped Brian and Linda develop Eqn, and were their first users,
along with Richard Hamming and Doug McIlroy.
Many of Eqn's features were eventually incorporated into Donald Knuth's \tex typesetting program.

Al recalls that the first Fortran compiler took 18 staff-years to create at IBM,
while the folks at Bell Labs were able to produce programming languages and compilers left and right
thanks to his new tools and the developments in parsing theory they built upon.

Another significant use of Lex and Yacc that perhaps sees more use today than Eqn is the AWK language,
named after its creators Alfred Aho, Brian Kernighan, and Peter Weinberger.
At the time, they considered it a throwaway tool and nobody else would be interested in it.
It was designed to be a pattern-action language; programs describe a pattern and a corresponding
action to be performed on every instance of that pattern in the input.

\todo{Ratfor, AMPL, other Kernighan languages.}
\todo{continue with typesetting...}

\subsection{The Dragon Book}

\begin{figure}[h]
	\centering
	\includegraphics[height=0.3\textheight]{resource/software/unix/dragon-book-1ed.png}
	\caption{Cover of the first edition of the Dragon Book}
	\label{fig:dragon-book}
\end{figure}

Jeffrey Ullman and Alfred Aho continued working together on
developing compiler theory and parsing algorithms.
Other folks at the Labs would impress on Al that it was important to write
about what you were working on to build your reputation in the
scientific community, an idea that Jeff also bought into.
They decided they ought to write a book together about their parsing
techniques in light of how much interest the development of Unix and C
had generated. Lots of people were newly interested in developing new
programming languages and compilers, and they were well-positioned to
write about it; thus was \textit{The Dragon Book} born.

\begin{quotation}
	As with the algorithms book, what we did was we performed research
	on efficient algorithms for parsing and for some of the other phases of
	compilation, wrote papers on those and presented them at conferences. But we
	took the important ideas that we developed and the community had developed over
	several decades and codified them into what are now called the dragon books. The
	first dragon book was published in 1977.We did have theorems and proofs in the
	book, and Jeff had this brilliant idea that the book should have a cover with a
	fierce dragon on it representing the complexity of compiler design,and then a
	knight in armor with a lance. The armor and the lance were emblazoned with
	techniques from formal language theory and compiler theory to slay the
	complexity of compiler design.
	\cite{aho_oral_history_2022}
\end{quotation}

They published updates to the dragon book over time as well;
they co-authored the second edition with Ravi Sethi, another Bell Labs researcher,
in the 1980s. The second edition had grown to over 800 pages, and the third edition
was published in 2007 at close to a thousand pages.
At this point none of the original authors wanted to work on a fourth edition
because of how much the field of compiler design had grown since then.

The first edition (with the iconic green dragon in Figure \ref{fig:dragon-book})
had a very heavy focus
on lexing, parsing, and semantic analysis, all but omitting optimization
techniques. This makes sense because parsing algorithms were Al and Jeff's specialties,
but as this book went on to be the seminal compiler textbook for a generation,
many student's primary exposure to compiler design was almost entirely focused on
front-end design.
Of the 600 or so pages in the first edition, only about 100 were dedicated to
optimization and data-flow analysis\cite[Chapters 12, 13, and 14]{the_dragon_book_aho_ullman_1977}.

\subsection{Trusting Trust}

\citetitle{trusting_trust_1984}.

\subsection{C++}

Bjarne wanted simula 67 stuff in C. Started C with classes, built on the C preprocessor,
which allowed all the other Bell people to use his stuff since it dropped right into C.
Eventually needed to do more than the C preprocessor could handle, so he built a new
preprocessor \textit{cfront}, which grew to become a complete and distinct compiler.
\citetitlecite{bjarne_stroustrup_design_of_cpp_1995}.
David Macqueen interview had interesting perspectives on this.

\section{Commodification}
\todo{Bill Gates and Paul Allen (Microsoft) | Microsoft BASIC (1975) | 
Developed the first critical piece of commercial software for personal 
computers,establishing the doctrine that software should be a purchased, 
proprietarycommodity. Sun microsystems, each part of the company needed to sell 
to all the others,reason why their compiler was paid; proprietary Unix;}
\pagebreak
\section{Timeline}
\begin{figure}[h]
\begin{luacode}
local start = 1965
local _end = 1980
timeline.draw_timeline({ 
    start_year = start,
    end_year = _end,
    marker_interval = 5,
    show_year = false,
    line_always = true,
    events = {
        {1967, "Aho joins Bell Labs shortly after Ullman"},
    },
})
tex.sprint(string.format("\\caption{TBD, %d--%d}", start, _end))
\end{luacode}
\label{fig:tbd-timeline}
\end{figure}


\chapter{Crisis and Systems, 1960-1980ish}

\todo{shit got crazy; became important, needs more focus/attention.}

\section{The Software Crisis}
\begin{quotation}
Despite great strides in software, programming always seemed to be in a state 
of crisis and always seemed to play catch-up to the advances in hardware. This 
crisis came to a head in 1968, just as the integrated circuit and disk storage 
were making their impact on hardware systems. That year, the crisis was 
explicitly acknowledged in the academic and trade literature and was the 
subject of a NATO-sponsored conference that called further attention to it. 
Some of the solutions proposed were a new discipline of software engineering, 
more formal techniques of structured programming, and new programming languages 
that would replace the venerable but obsolete COBOL and FORTRAN. Although not 
made in response to this crisis, the decision by IBM to sell its software and 
services separately from its hardware probably did even more to address the 
problem. It led to a commercial software industry that needed to produce 
reliable software in order to survive. The crisis remained, however, and became 
a permanent aspect of computing. Software came of age in 1968; the following 
decades would see further changes and further adaptations to hardware advances.
\cite{history_of_modern_computing_2003_ceruzzi}
\end{quotation}

\todo{DEC PDP-8 and PDP-11; IBM System/360 and OS/360; Multics; Unix; C}

As the 1960s progressed, the notion of \textit{software} became more established,
and the programs being written served the authors to a much greater extent.
Prior to the 1960s, programs were often tailor-made for a specific machine.
There was no hope of re-using the program on another machine.
For most of the users, their organization had spent a considerable portion
of their budget on their system, and they were expected to use it for a long time.
Retargetable compilers did not exist.

Michael Mahoney made a strong statement to this end rather early on \cite[The Structures of Computation]{the-first-computers-2002}:
\begin{quotation}
The kinds of computers we have designed since 1945 and the kinds of programs we 
have written for them reflect not the nature of the computer but the purposes 
and aspirations of the groups of people who made those designs and wrote those 
programs, and the product of their work reflects not the history of the 
computer but the histories of those groups, even as the computer in many cases 
fundamentally redirected the course of those histories.
\end{quotation}

\begin{quotation}
Despite great strides in software, programming always seemed to be in a state 
of crisis and always seemed to play catch-up to the advances in hardware. This 
crisis came to a head in 1968, just as the integrated circuit and disk storage 
were making their impact on hardware systems. That year, the crisis was 
explicitly acknowledged in the academic and trade literature and was the 
subject of a NATO-sponsored conference that called further attention to it. 
Some of the solutions proposed were a new discipline of software engineering, 
more formal techniques of structured programming, and new programming languages 
that would replace the venerable but obsolete COBOL and FORTRAN. Although not 
made in response to this crisis, the decision by IBM to sell its software and 
services separately from its hardware probably did even more to address the 
problem. It led to a commercial software industry that needed to produce 
reliable software in order to survive. The crisis remained, however, and became 
a permanent aspect of computing. Software came of age in 1968; the following 
decades would see further changes and further adaptations to hardware advances.
\end{quotation}

\todo{1945 was too early for this strong of a statement;
how can one argue that software reflected the authors when it was so dependent on the hardware?}

\section{Emergent Theory}

\todo{Software (and compilers!) starts to become a real discipline!}

\begin{quotation}
\textbf{Collaboration with Ullman}
Aho is best known for the textbooks he wrote with Ullman, his co-awardee. The 
two were full time colleagues for three years at Bell Labs, but after going 
back to Princeton as a faculty member Ullman continued to work one day a week 
for Bell.

They retained an interest in the intersection of automata theory with formal 
language. In an early paper, Aho and Ullman showed how it was possible to make 
Knuth's LR(k) parsing algorithm work with simple grammars that technically did 
not meet the requirements of an LR(k) grammar. This technique was vital to the 
Unix software tools developed by Aho and his colleagues at Bell Labs. That was 
just one of many contributions Aho and Ullman made to formal language theory 
and to the invention of efficient algorithms for lexical analysis, syntax 
analysis, code generation, and code optimization. They developed efficient 
algorithms for data-flow analysis that exploited the structure of "gotoless" 
programs, which were at the time just becoming the norm.
\cite{aho_turing_award_2020}
\end{quotation}

\begin{quotation}
The Early History of Software, 1952-1968 101

In the early 1960s computer science struggled to define itself and its purpose, 
in relation not only to established disciplines of electrical engineering and 
applied mathematics, but also in relation to—and as something distinct from—the 
use of computers on campus to do accounting, record keeping, and administrative 
work.58 Among those responsible for the discipline that emerged, Professor 
George Forsythe of Stanford's mathematics faculty was probably the most 
influential. With his prodding, a Division of Computer Science opened in the 
mathematics department in 1961; in 1965 Stanford established a sepa-rate 
department, one of the first in the country and still one of the most 
well-regarded.59
\cite{history_of_modern_computing_2003_ceruzzi}
\end{quotation}

\section{Commoditization}

\todo{Bill Gates and Paul Allen (Microsoft) | Microsoft BASIC (1975) | Developed the 
first critical piece of commercial software for personal computers, 
establishing the doctrine that software should be a purchased, proprietary 
commodity. Sun microsystems, each part of the company needed to sell to all the others,
reason why their compiler was paid; proprietary Unix;}


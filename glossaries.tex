\makeglossaries

\newglossaryentry{bytecode}{
    name={bytecode},
    description={A compiler intermediate representation for the purpose of interpretation or execution instead of optimization}
}

\newglossaryentry{bootstrap}{
    name={bootstrap},
    description={The process of writing a compiler in the language that it compiles, such that an older version of the compiler can be used to compile a newer version of itself}
}

\newglossaryentry{ub}{
    name={undefined behavior},
    description={Source code constructs that are illegal as per the language's specification. Typically, undefined behavior assumed never to happen in a well-formed program, is used by optimizers when certain compiler flags are enabled (such as \texttt{-fstrict-aliasing})}
}

\newacronym{ir}{IR}{Intermediate representation, usually a file format or data structure used inside a compiler to represent the program being compiled}

% Use
% The \Gls{latex} typesetting markup language is specially suitable 
% for documents that include \gls{maths}. \Glspl{formula} are 
% rendered properly an easily once one gets used to the commands.
% Given a set of numbers, there are elementary methods to compute 
% its \acrlong{gcd}, which is abbreviated \acrshort{gcd}. This 
% process is similar to that used for the \acrfull{lcm}.

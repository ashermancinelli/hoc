\makeglossaries

\newglossaryentry{bytecode}{
	name={bytecode},
	description={A compiler intermediate representation for the purpose of interpretation or execution instead of optimization}
}

\newglossaryentry{bootstrap}{
	name={bootstrap},
	description={The process of writing a compiler in the language that it compiles, such that an older version of the compiler can be used to compile a newer version of itself}
}

\newglossaryentry{ub}{
	name={undefined behavior},
	description={Source code constructs that are illegal as per the language's specification. Typically, undefined behavior assumed never to happen in a well-formed program, is used by optimizers when certain compiler flags are enabled (such as \texttt{-fstrict-aliasing})}
}

\newglossaryentry{F77}{
	name={Fortran 77},
	description={The 1977 version of the Fortran programming language's standard}
}

\newglossaryentry{F90}{
	name={Fortran 90},
	description={The 1990 version of the Fortran programming language's standard}
}

\newglossaryentry{foss}{
	name={FOSS},
	description={Free and open-source software. This software is typically distributed under a license that allows users to modify and redistribute it freely, though different licenses imply different permissions. The two categories of open-source licenses are \textit{permissive} and \textit{copyleft}.}
}

\newglossaryentry{ftn}{
	name={Fortran},
	description={The Fortran programming language, standing for \textit{FORmula TRANslator}. Most renditions of the name of the programming language have only the first letter capitalized (\textit{Fortran}), however early versions were rendered as \textit{FORTRAN}. We attempt to use the proper name for each time period. \textit{Fortran} came to be used after the 1990 edition of the standard, while the 1977 standard and all prior versions were rendered as \textit{FORTRAN}.}
}

\newacronym{ir}{IR}{Intermediate representation, usually a file format or data structure used inside a compiler to represent the program being compiled}

% Use
% The \Gls{latex} typesetting markup language is specially suitable
% for documents that include \gls{maths}. \Glspl{formula} are
% rendered properly an easily once one gets used to the commands.
% Given a set of numbers, there are elementary methods to compute
% its \acrlong{gcd}, which is abbreviated \acrshort{gcd}. This
% process is similar to that used for the \acrfull{lcm}.

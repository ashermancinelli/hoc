\documentclass[12pt,openany]{book}
\usepackage{verbatim}
\usepackage{libertinus}
\usepackage{hyperref}
\usepackage[margin=1.0in]{geometry}
\usepackage{titlesec}
\usepackage{natbib}

% \bibliographystyle{plainnat}  % or abbrvnat, unsrtnat

\fontfamily{libertinus}
\titleformat{\chapter}[frame]
  {\normalfont\huge\bfseries}{\thechapter}{0.5em}{\Huge\centering}
\titlespacing{\chapter}{0pt}{0pt}{40pt}


\title{A Brief History of Compilers}
\author{Asher Mancinelli}
% \date{3/10/2025}
% \makeindex

\begin{document}

\maketitle
\tableofcontents

\chapter{Preface}

The introduction chapter contains a brief version of the entire book; the reader is encouraged to read it first.
My hope is that each chapter stands on its own, but that reading them in order will give a more complete picture.
The reader ought to be able to read a particular chapter that suits their needs at the time.
This book does not assume the reader is deeply familiar with compiler engineering or computer science.

\chapter{Introduction}

While the modern programmer may consider the term \emph{compiler} to be a specific one, it is specific only in the
sense that users of the word tend to mean the same thing when they say it.
The details of what a compiler does are often obscure to even professional software engineers.
This is especially evident when they feel the need to use a term like \emph{transpiler} to distinguish between a
"true" compiler and one that emits code another programming language.
Of course, to someone more familiar with the workings of compilers, this distinction is not so useful.
Brian Kernighan described compilers in this seemingly general way\cite{new_history_of_modern_computing}:

\begin{quotation}
\textit{
A compiler is a program that translates something written in one language into something semantically equivalent in another language.
For example, compilers for high-level languages like C and Fortran might translate into assembly language for a particular kind of computer; some compilers translate from other languages such as Ratfor into Fortran.
}
\end{quotation}

However, this is not sufficiently general.
Two key counter examples are \TeX and \texttt{METAFONT}, which are compilers that
transform text into PDF documents (usually) and renderable fonts, respectively.
Transforming text into an executable program is not the same process as
transforming text into a document or font, yet both are considered compilation.

\chapter{Dawn}
\chapter{Unices}
\chapter{Freedom}
\chapter{Codesign}

\bibliographystyle{plain}
\bibliography{refs}
\end{document}
